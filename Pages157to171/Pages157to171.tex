
\section{Third Section}

All the pieces included in this section originally appeared as articles, either in *Folkelbladet* or elsewhere, most of them toward the end of the eighteen‑nineties. The reason was here also the deep‑seated disagreements in the understanding of the nature of the Christian congregation and its appearance in the world, disagreements which have made themselves felt throughout the entire history of the Norwegian Lutheran Church in this country. The occasions were, as will be seen, various—partly official, partly unofficial statements from ecclesiastical quarters. The discerning reader will, however, always notice the author’s constant effort to point out the principles that lie at the foundation of what is intended. For this reason I have thought that these series of articles ought to be included here, even though several of them may seem to bear so local a character that they might more appropriately have been placed in the volume of the present work where it is intended to present a number of the author’s statements concerning the work and struggle of our Church. — The occasion for the publication of the individual articles is stated with sufficient clarity in the articles themselves that the reader will understand it without difficulty. The information deemed necessary has been added in notes beneath the text. — Ed.

\subsection{The Old Church Order}

Source: Editorial articles in *Folkelbladet* of July 21, August 4, October 20 and 27, December 1 and 15, 1897; January 19, February 16 and 23, and March 2, 1898; August 21, 1901. The last article is, as may be seen, of a considerably later date than the others, but is included here since it deals with the same matter. — Ed.

\bigskip

\subsubsection{What is the Meaning?}

We must assume that "The Lutheran" has reported Chairman Høyme’s words fairly and accurately in his report to the United Church’s annual meeting. Had we not read it in a paper that is "true to the truth in love"\footnote{"The Lutheran’s" motto. "The Lutheran" is the official ecclesiastical organ of the United Norwegian Lutheran Church in America. — Ed.}, we would have said that even a man like this had been treated unjustly. But now it stands to be read in "The Lutheran," and therefore we must assume that it is correct to the very nearest degree.

According to No. 25, Høyme said:

"We will not invent anything new. The old means—Word and Sacrament—and the old church order are good enough. No prophet has yet been raised up among us Norwegian Lutherans who can give us better means and a better order."

And again:

"The United Church stands firmly upon the ground of God’s Word and the Lutheran Confession and flies a pure flag. The old Lutheran doctrine and the recognized church order it will faithfully guard.\footnote{The quotations cited here after "The Lutheran" are found in the United Church’s Annual Report, 1897, pages 18 and 27. — Ed.}

The striking deliberations on this point at the annual meeting also show that the assembled pastors, with great delight, seized upon the new watchword that has now been given to the United Church for its work and struggle:

"T h e  o l d  c h u r c h  o r d e r."

But what does this mean? "The old church order" is placed in one part of the report side by side with the Word and the Sacraments; in another place it is set side by side with the old Lutheran doctrine. It must therefore be something of extraordinary importance, which also seems to follow from the solemn designation:

"The pure flag."

We will not attempt to guess what the United Church means by "the old church order"; we shall only confine ourselves to asking "The Lutheran" to inform us what is meant by this old church order, which in the United Church is placed entirely on the same level as the Word and the Sacraments and with "the old Lutheran doctrine."

For if this new addition to the church’s precious inheritance and ancestral possession is justified, then the entire Lutheran Church up to the annual meeting in St. Paul in 1897 must have drifted in ignorance and lived in deep darkness. Before that meeting, it never knew that anything else could, in truth and significance for the church, be placed alongside the Word and the Sacraments—or even alongside the old Lutheran doctrine.

What is meant by "the old church order"?


\subsubsection{Why Not Answer?}

Two weeks ago we ventured to ask “The Lutheran” what is meant by “the old church order,” which has now in the United Church become so significant that it is placed side by side with the Word and the Sacraments and with the old Lutheran doctrine.

“The Lutheran” answers the question in a manner which, in truth, is not an answer at all. It says first that the author of the question in “Folkelbladet” “presumably knows as well as anyone among us what is meant by old Lutheran church order.” That is not the point.

We know what we have in mind when Chairman Høyme and the annual assembly of the United Church speak of “the old church order.” Whether it is Lutheran is another matter. But in this matter, as in every other, we wish to be as fair as possible; therefore we would prefer, before saying anything about “the old church order,” to hear what the official organ of the United Church means by this catchword. “The Lutheran” would render both its own cause and us a service by a plain and straightforward explanation of what is meant by “the old church order.” Why be so reticent in this matter? Chairman Høyme and the annual assembly regarded it as very important indeed. “The old church order” is, after all, one of the great blessings which the United Church already possesses alongside the Word and the Sacraments and alongside Lutheran doctrine. “The old church order” is something the United Church will guard, says the Chairman.

From the Chairman’s statement, from the discussion at the annual assembly, and from the remarks of “The Lutheran,” it is evident that it is the Free Church in particular which is regarded as an enemy of the old church order.

The Free Church recognizes, so far as we know, the apostolic order in the congregations; perhaps that is too old?

Of the two matters to which “The Lutheran” points, and which are said to be found particularly in the Free Church—namely, that one grants women the right to speak and vote in the congregation, and that one seeks to tear down all congregational boundaries—the latter is entirely unknown to us; the former is something that is found here and there both within and outside the United Church. Whether it is right to grant women the right to speak and vote in the congregation can be discussed at another time. What we would like to receive a plain and exhaustive answer to is this question: What is meant by “the old church order”? Is it really so precious a matter, and so binding a law, as God’s Word and as Lutheran doctrine? If it is, why is it so, or when did it become so?


\subsubsection{How Old Is “the Old Church Order?"}

“Lutheraneren” has not been able—or has not wished—or has not dared—to explain what is meant by “the old Church Order.” We have asked for quite some time and have requested, very politely, an answer.

But no: “Lutheraneren” has replied that we knew it well enough ourselves. That is not a courteous answer; but perhaps it is as good as we deserve, and so we are content.

We must therefore attempt, without the assistance of “Lutheraneren,” to provide a small clarification—now and then—concerning this Church Order, which suddenly became so precious to the United Church when its chairman happened to hear one of these new things that are now being asked about: the congregation.

It does indeed sound rather warm and cozy with “the old Church Order.” The question, however, is how old it actually is, and whether it is so good because it is so old.

If that Church Order which is accorded such importance as the Means of Grace and the old Lutheran doctrine is old—so old that it is venerable by reason of age—then it cannot very well be a product of the United Church; for it has, at any rate, no age at all. Nor has the old Church Order been created by the Norwegian Synod, for it has scarcely, in ecclesiastical respects, worn out its childhood shoes.

If the Church Order is old, then it comes from the Norwegian State Church; and there we may stop. For to go further back would lead us over into Catholicism, and not even the United Church would be able to persuade its people to believe that it was any advantage to the old Church Order that it traces its lineage from the Catholic Church. In many respects it does indeed do so, as all who study a little church history know; but that is not what is being asked here. We wish only to seize this old Church Order at a point where it is for most of us easily recognizable and accessible, so that all our readers may see it, as it were, before their eyes and measure it with the measure of truth in the light of the New Testament.

And yet the state-church order has not been transferred altogether unadulterated and unchanged from Norway to America. It has, through the Norwegian Synod and “the old direction,” undergone certain changes which we would say were for the better, in the process of transfer. For the clerical aristocracy which in Norway is upheld by law and external power must in our country be upheld in another way; and this other way is certainly more harmful to the spiritual life itself and to the congregation than the external coercion of the law in the old country. Summed up in a single word, it is called church politics—the sum of all the methods employed to uphold the caste system and clerical superiority among us. And insofar as the means employed are more or less pure and honorable, this strange phenomenon called church politics becomes correspondingly more or less honorable.

We do not wish to go further into the question of the old Church Order on this occasion. We have arrived at a provisional answer that places the matter plainly before every person’s eyes: “the old Church Order” is the state-church order, with the distortion it undergoes when it is moved out of its old framework and must be upheld by new supports.


\subsubsection{On the Foundation of the State Church}

“The old church order,” which is the united church’s great and important treasure, its “pure banner,” is, as we have already demonstrated, the Norwegian state church’s office-aristocracy, with the peculiar cast this system has acquired through the church politics of the “old direction” and the Synod.

Despite all Chairman Høyme’s forceful expressions and blows upon the table, we intend now as before to retain the right to examine the Christian character of this order; indeed, we permit ourselves even to ask: is it conceivable, and can it reasonably be said, that the caste system of the state church should truly fit so well within an American free church?

The royal official in the Norwegian state church is a great man in the eyes of many small folk. Worse still, he is often an even greater man in his own eyes. And more often still, he is the greatest man in his wife’s eyes.

It has cost much money and long schooling to reach the desired goal — the office examination; it has required a long wait for appointment to the royal office. No one who has not seen it knows what labors of provision father and mother have borne before the son came through the costly school. No one who has not experienced it knows how many strange devices the student has had to invent in order to earn a little himself, so that he might sustain himself at the university.

When he finally came through and became “finished,” it was most often with debt, shattered health, an exhausted mind, and often with an empty heart. If he then received an office, it seemed to him no more than fair that he should receive compensation for what he had suffered. He ought now to enjoy the dignity of office, the people’s reverence for the pastor, the domestic happiness of a cozy parsonage — and rest and peace for his weary nerves. He had not studied in order to be fit for labor; he had studied in order to obtain a royal office.

And the royal office secures and protects him quite substantially. At least it did so some thirty to forty years ago. Many things have changed in Norway in recent times.

The poet Garborg knows something of how Norwegian peasants regarded the parsonage and its inhabitants. He says somewhere of the young peasant lad who steps into the parson’s garden and the parson’s house:

“Daniel had been with his father into the parsonage a couple of times, so he knew it. It was so clean and fine that one had to think they washed the floor every day; and chairs and tables were made of wood that gleamed, and there were flowers in the windows, fair as in Paradise, and curtains so light and white as the veils of innocence in heaven; and on the wall there was a picture so utterly unlike the Virgin Mary and Emperor Nicholas that they had at home that there was no resemblance at all; and shining brass handles on the doors, and glass cabinets with red and blue and gilded bowls, and a great strange instrument that they played upon so beautifully that one could weep, and a white marble head up on a shelf, and so many, many fine things whose names he did not know — and over it all a breath, something pure and gentle and heavenly, a fine fragrance he could not explain, a soft, wholesome, smiling warmth that does not oppress and that one does not grow drowsy from — ah, ah — it was not to be told.”


It has its appeal to flesh and blood. Many a scribbler has thought to himself: if only one were a pastor!

Can anyone be surprised if there are those who like this for their own sake and would gladly have it so, if it has fallen to their lot to become pastors, even if it were in America?

It is a quite natural thing that all those who have become pastors without being driven by the Spirit of God would gladly have it so. All those who have chosen the estate, the office, the whole way of life, would gladly have it so. Even some of those who truly from the beginning sought the blessed work and the holy service for God and his congregation succumb to the temptation of temporal greatness, honor, and enjoyment, when opportunity presents itself.

But add now to this what for many pastors of “the old church order” is the main thing: that they are the people’s religious guardians, those who have “the keys,” so that they open and shut the kingdom of heaven for human beings; it is no wonder if those who have first been seized by the thought of the exalted office will not give it up. For to have a religious power over human souls—that is in itself, apart from any outward advantages that may follow with it, nevertheless the greatest glory.

A red thread runs through the whole conception of the exalted office. The Latin school and the university and the examinations are the steps of the stairway that lift one up into this higher sphere, where only office-holders or the prominent people move about. The common man ought, if not exactly bodily then at least spiritually, to stand with his hat in his hand before this being. The pastor is the learned one, before whom the ignorant man bows. The pastor is the holy one, before whom the worldly man bows. The pastor is the one who has access to the highest sanctuary; the people must devoutly behold him from afar.

Such is “the old church order.” This is the United Church’s binding banner and “pure flag,” if we may dare to believe the chairman. There is something indeed comforting in many places in Norway connected with this order. For the pastors need only the royal appointment and the worldly salary in order to maintain their elevated position. And besides, there were many who quite simply looked up to “Father,” without reasoning much about whether “Father” truly was such an exalted being or not.

This comforting character falls away from “the old church order” in America. Under our free conditions there is no royal power that holds the office high above the people. Other supports must be found. Might not the extraordinary zeal for a great association have its root in this need for new supports for the old exalted office?

Would it not fit well with this purpose to revile those who ask after the congregation?

But there are also many other means that are used in order to attain the great goal. We need not mention them. Most of those who have followed the ecclesiastical development among us know a thing or two about this sad history.


\subsubsection{The Old Church Order and the Education of Pastors}

It is centuries of misuse gathered together in what is called “the old Church Order.”

Its influence therefore shows itself also in all areas of church life. And now, since this old Church Order has become a central concern for the great majority, it is necessary to illuminate it from various sides, if this might serve as an admonition. For it would be grievous if our Norwegian Lutheran Church in America were to walk blindly in dangerous paths, when yet the Lord in his Word has shown it a far better way, namely the congregation.

The old Church Order is the result of a long-standing secularization of the Church, or of an attempt to unite the Church and the world. Congregation and pastor were to be Christians without trial and without the cross. This is the inner corruption that gnaws at the marrow both in the papal church and in the state church.

We cannot cover the entire ground in a small newspaper article. But we may at least attempt to take hold of one side at a time.

According to the old Church Order, the pastor is a royal officeholder; and his upbringing is arranged accordingly. Now one cannot retain a royal officeholder in America; yet, in spite of these circumstances, one insists that pastoral education must be the same as in the state church.

One wishes to hold fast to Latin schools and universities according to the state-church pattern. If one cannot raise young men up into the royal office, one can at least, by means of Latin, raise them into “the higher strata.”

It is a lofty goal; for it is no small thing to be the representative of King and Church, who goes with head held high through crowds of men who, bowing, bare their heads before the venerable one.

Therefore the path upward is steep and abrupt. The many classes, the dead languages, the harsh examinations, and the solemn ordination are steps of a stairway, each of which is burdensome in itself, and all together almost insurmountable.

And because it is so hard to advance, all schools are arranged so as to awaken petty vanity and unscrupulous ambition and rivalry, in the hope that one might nevertheless attain good results, as each strains to come more quickly forward and higher up than the other.

In the process many become overstrained; most are quite satisfied with their labor once they have arrived at “the office.”

And in reality it is doubtful whether they are fit for much work at all when they finally, half worn out, arrive at the place where the work ought to be done. For all the while, as they believed they were clearing their way to the “higher ranks,” they were steadily removing themselves farther and farther from the people and from the congregation, and became strangers to both. They did not truly prepare themselves for work; they prepared themselves to pass examinations and obtain office.

By this we do not deny that the state church has capable pastors. We merely draw attention to how misguided their education has been, and how much they have had to relearn when they were to take hold of real work.

Neither do we deny that many men may become poor workers despite having received an education planned on a healthier foundation.

Nevertheless, it is in advance probable that the state-church pastoral education fits neither a free land nor a free church. When we are not to have royal officeholders, why then should we have their education?


The New Testament lays out the path of the Christian congregation in all essential respects, and the education of pastors constitutes no exception to this rule.

Christ himself trained twelve men whom he intended to send out to preach the Gospel. He was not satisfied with those who “sat on Moses’ seat”; he desired other and better workers in his harvest.

This education is exemplary. It is true that it does not show us how the disciples acquired the human knowledge which we find them to possess. But the spirit and character are presented in such a way that it leaves no doubt that this is an example to be followed.

Time and again we find that ambition—which is now so diligently fostered in all higher and lower schools, and which has become the dreadful Moloch that devours our children—also gained a foothold among the Lord’s first and closest disciples.

Driven by ambition, the disciples contended with one another over who was to be the greatest among them. They did not know that they were walking in the company of the Holy One who would soon be crucified before their eyes.

Seized by ambition, the sons of Zebedee asked to be allowed to rise into the higher ranks and sit at Jesus’ right and left hand when he came into his kingdom.

Was it then Jesus’ will that the disciples, by ambitious rivalry, should strain their powers to the utmost in order to surpass one another in knowledge and obtain better examination results than one another?

He answered by taking a little child and setting it in the midst of the ambitious and contending disciples, and admonished them to humble themselves as this child.

He answered instead by saying that the question is not about lifting oneself up to the highest place, but about descending to the lowest; it is not about desiring and obtaining, but about loving and giving.

And he holds an examination of a different kind than the universities do, when he asks Peter: Simon, son of Jonah, do you love me? And he gives testimony and credentials in his own way, when he says: 

Tend my sheep!

It would lead too far to go into details. We recognize the main point, which is this: that Jesus trains his disciples for something infinitely higher than being learned men, and for quite the opposite of ambitious heads. He does not merely want them knowledgeable; he wants them driven by the Spirit. He wants to make them humble bearers of the cross, eager to save souls.

Both the knowledge that is needed and the Spirit—without which knowledge becomes harmful rather than beneficial in the congregation—can be given only in and through the Word. Beside the Word, which is Spirit and life, all dead languages are without value in pastoral education. Only when the study of the dead languages stands in the service of the living Word of God does it gain value in this connection.

Immerse yourself in the Word! 

Immerse yourself in the Scriptures! 

This is the Alpha and Omega of Christian, congregational, and free-church pastoral education. Learn to know God’s revelation and God’s Spirit! Learn to love Jesus; learn to love souls! Learn to labor among people!

One will reply: yes, but no school can do this anyway. Each person must learn this for himself in the experience of life. So be it; and yet a school that leads away from the goal can become a great hindrance.

We believe that the most important source of all church conflict among us is precisely the lack of clarity at this chief point. Many feel that the state-church clergy do not correspond to the needs of free congregations. They look around to see how something better might be put in their stead. But because the poor Augsburg has dared to attempt a small change, it is therefore persecuted and threatened with destruction. Is it entirely certain that the old church order, in this respect, is exactly what fits a free church, and that it will therefore bring peace?


\subsubsection{Has the Congregation Been Harmed?}

We have previously pointed out that “the old Church Order” has done great harm to the clergy, to their relationship with the congregation, and to their education.

Has, then, the congregation suffered no harm from this old Church Order? Or is the congregation such a solid and invulnerable organism that nothing can damage it?

It is already more than enough harm for congregations that they have been brought into a distorted relationship with their own pastors. It is a fruit of the old Church Order that many congregations regard the pastor as a necessary evil, which, for superstitious reasons, they nevertheless do not dare to abolish. It is also a fruit of the old Church Order that in many places the few earnest men and women within the congregations look upon the pastor with mistrust, as a man who must almost certainly be counted among the opponents of spiritual life. It is likewise an effect of the old Church Order that very often there is a chasm between pastor and congregation that is insurmountable, because the pastors have received a schooling that is neither rooted in the people nor congregational in character.

But this sorrowful estrangement, under which both pastors and congregations suffer almost equally, whether they now feel it or not, is by no means all. One cannot even ask: at what point has the congregation suffered harm? One must rather ask: at what point has it not suffered harm?

For the fact of the matter is that under the old Church Order the congregation has become thoroughly hostile to spiritual life; so completely has it taken the nature of the world into itself.

In the state church the congregation is not gathered by the Word, by the invitation of the Gospel, but by outward coercion and worldly remuneration. And this has been practiced for centuries, ever since the time of Constantine the Great, until coercion has become a habit, which is now baptized with the fair name: Old Church Order.

But no old habit has ever been able to change human hearts. And however much it has become a habit to belong to the congregation in a merely outward way, neither true repentance nor living faith has become any habit, and can never become one.

And therefore we find that the old habit reigns among our people to a great extent; but true Christianity and spiritual life are rarely to be found; and in most places where they are found, they are exposed to the fiercest hostility of Christianity reduced to habit.

The sleeping and worldly church is the bitterest enemy of spiritual life. For the manifest world is not so afraid of losing its worldly pleasures as the sleeping church is afraid of being disturbed in its sweet repose and its rotten peace.

If, therefore, habit and long-established customs and ceremonies, in connection with a measure of servile fear and superstition, are the bond that in many places holds the congregations together, then it is not at all to be wondered at that the few men and women among us who labor for change are subjected to bitter persecution.

This sorrowful inheritance from the state church—that the congregation is more an outward custom than a result of the gathering power of the Word and the Spirit—is self‑evidently one of the chief reasons why there is so little Christianity within the congregations, and why the congregations exercise so little spiritual influence.

Here also lies the reason why so many ask with such anxiety: What shall we do to preserve the youth for the Lutheran Church? \textbf{If we go to the bottom of this matter, we shall find that the reason for the youth’s alarming inclination to depart from the Lutheran Church lies precisely in this, that the young people are not so bound by the habits of the state church and have no superstitious fear of breaking “the old church order.”}

If we had congregation—true congregation—with the Spirit and life of Christ in heart and home, in house and church, then there would not be such great difficulties in this matter.

But if, in this way, the formation of many congregations is of doubtful value in a spiritual sense, then it is not to be wondered at that we find many dark sides in congregational life under the old church order. And it will not be out of place to look a little more closely at the deficiencies under which we suffer; for only when we see the deficiencies is there any hope of having them remedied. It will do good to look at the difference between what the congregations ought to be and what they in reality are; perhaps someone might begin to long for a truer foundation for the congregation than that to which we have commonly accustomed ourselves.


Dr. Schmidt even appears now and then in “The Lutheran” and attempts to insert himself into the discussion of the congregation. That can only occur insofar as he departs from the core of the matter and serves up a multitude of inaccuracies.

The core of the matter is this, which Dr. Schmidt has not yet touched:
Is the New Testament’s teaching concerning the congregation just as binding upon Christians as what it teaches concerning doctrine?

For if it is not the New Testament, but rather the ordinances of the Church Fathers, that constitute law and rule for the Christian congregations, then the Lutheran Church is thereby condemned; for it has established the principle that Scripture alone is rule and guide, and the ordinances of the Fathers are not rule and guide.

And if Dr. Schmidt, together with Chairman Høyme, drinks old wine and believes in the old church order with indifference toward and disregard for, the New Testament’s presentation of the congregation, then all discussion with him is impossible; for then there is absolutely no guarantee as to what he may devise. Without a shared point of departure and common measure, every discussion is useless and harmful.

Come, therefore, to the core of the matter, which is the question whether the New Testament is, in the same sense, a binding rule for the congregation as for doctrine. Is this so, or is it not so?

At the same time we shall show what we mean by Dr. Schmidt’s misrepresentations by a single example drawn from an innumerable multitude, which would only unnecessarily weary the readers.

He says, for example, in “The Lutheran” of December 15, 1897:

“Chairman Høyme had during the annual meeting let fall a remark to the effect that our fellowship intends to hold fast to ‘the old church order.’ It was not difficult to understand that by this reference was meant all this new and foreign matter which the friends of the Free Church now insistently come forward with,” etc.

And a little further down he continues:

“But our opponent seizes upon this expression concerning ‘the old church order’ and seeks, by a multiplicity of direct accusations and indirect side-lights, to bring it about that our fellowship, under the name of ‘old church order,’ maintains and defends approximately all the wrong that has existed in the world before the establishment of any sort of church order whatsoever—whether in the synod or in Lutheran state churches or in the hierarchy of the Middle Ages or even in the Jewish church, both under Caiaphas in Christ’s time and under all the false prophets in the older times.”


This is unreliable and dishonest talk. “The old church order” was not a remark that Chairman Høyme “let fall” during the annual meeting. And Dr. Schmidt knows that this is not true. “The old church order” is part of Chairman Høyme’s official and written report to the annual meeting of the United Church. And he did not let this phrase “fall”; oh no—he hoisted and lifted it high and called it “the pure flag.” If the flag has now fallen, then it is certainly not Høyme’s fault; it is Dr. Schmidt who “lets it fall.”

Nor is it only Høyme who is responsible for “the old church order.” The annual meeting has adopted two resolutions, both of which concern this matter and which confirm Høyme’s words. The annual meeting decided “to endorse the chairman’s statements regarding the so‑called free church of Augsburg’s Friends,” which obviously, as anyone can see, includes also the statement concerning “the old church order.”

The annual meeting also decided: “By God’s grace the United Church will hold fast to the Confession and the church order that accords with it.”

The latter phrase was, as those who were present will remember, another wording for “the old church order.”

Oh no, Høyme was not alone in speaking of “the old church order,” and it did not “fall” at all during the annual meeting; it was supported both strongly and effectively, so that it remained standing. If anyone has let it fall, that has happened after the annual meeting.

Nor is it true that Folkelbladet has “seized upon this expression” and read into it what was not intended.

We have a solid record in this matter, and Dr. Schmidt knows that he is using dishonest language when he says the opposite.

We began by asking “Lutheraneren,” an organ that is certainly not unknown to Dr. Schmidt, whether it would tell us what was meant by the old church order. To this “Lutheraneren” replied that Folkelbladet knew that just as well itself. We asked again and waited a long time for another and better answer; but we received none. We were therefore left to conclude that our answer was good enough for “Lutheraneren,” and so we gave the answer ourselves. And then we stated that the old church order could not derive from the United Church; for then it would not be old. It must rather be sought in the Lutheran state churches; for back to the Catholic Church we would not go, even though there might be good reason to do so.

Can Dr. Schmidt now say what wrong we have done in this? Or is not “the old church order” precisely the old state‑church order, adapted to new conditions under which the poor old order has received some peculiar additions of highly doubtful value?

But if it is not this, then what is it?

If, namely, the men of the United Church are willing to explain the meaning of their grandiose slogan, then we shall be grateful, because at least we will have achieved that much. The people may then this year learn what their “pure flag” is, which the leaders raised last year.

