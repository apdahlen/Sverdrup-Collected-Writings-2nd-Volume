\section*{Foreword}

As the second volume of Professor Georg Sverdrup’s collected writings in selection, under the title “On the Congregation,” is presented to the public, it will be necessary to preface it with a few orienting remarks. Professor Sverdrup could, in a quite distinctive sense, make the Apostle Paul’s words his own: “But I speak concerning Christ and the congregation” (Eph. 5:32). His entire long and busy life had this as its great goal: the glorification of Jesus Christ through the enlivening and liberation of the congregation. And it is surely scarcely necessary to add here that from his innermost heart he believed and confessed: “But this is the work of the Holy Spirit.”

Nevertheless, Professor Sverdrup’s proper field of labor lay within the narrower sphere of the clergy. There he planted precious seeds in the hearts of his many disciples—seeds which have borne fruit and will surely bear still more glorious fruit. So fully absorbed was he in his work at the seminary that comparatively little time remained for longer written works on the important and far-reaching truths concerning which so many received oral instruction from his lips. He therefore had largely to restrict himself to taking up the pen when it was a matter of illuminating and exposing prevailing errors and clearly revealing threatening dangers to Christian life. But then he could also, like no other among us, cast the light of God’s Word upon the churchly struggles and strivings of the day. There will from time to time be those who, in the external circumstances here mentioned, will see a deficiency; others, however, will regard it as an advantage.

Those, namely, who approach the reading of this volume of Professor Sverdrup’s writings with the thought that they will find an abstract, scholarly exposition of the question of what the Christian congregation is, etc., will probably to some extent feel disappointed. What they will find instead are popular expositions of a series of principled questions concerning the congregation, and alongside this a considerable number of contributions to ecclesiastical discussion. But what is distinctive about these contributions is that the author always seeks therein to turn the gaze away from what is purely accidental and momentary and toward what is enduring and principled. In this lies their very greatest significance, and in this they differ from so many other contributions to religious and church questions, in that they cast the light of the Word and of history upon the prevailing—partly purely local—conflicts and difficulties. Thereby they are also highly suited to emphasize the exceedingly important truth that God’s Word has answers to all questions within the realm of church and Christianity for the one who, under the guidance of the Holy Spirit, seeks its answers, and that even the struggles and difficulties of the day have general significance and therefore should not be regarded merely as insignificant and purely accidental events of exclusively local interest and scope. Guided by the basic views here indicated, the author consistently aims to lift the gaze of his hearers and readers upward, in order to help them attain a broad and clear outlook upon the Lord’s will and ways.

It has not been without its difficulties to attempt to gather these many treatises, which make up this volume, into suitable groups, and I am by no means certain that I have succeeded even approximately. The selection included here also spans a period of thirty-eight years, and it is self-evident that repetitions will occur in part. I have also sought as much as possible to adhere to chronological order, so that the author’s own perspective through these many years might more clearly come into its own. Nevertheless, I have considered it best in these sections to assemble material from different periods of his active life.

From what has been said above it will be understood that it has not been well possible to avoid mentioning persons and ecclesiastical bodies, sometimes with censure or criticism. Where I deemed it necessary to include the relevant passages, I have not made substantial alterations. In this volume I have also followed—and intend to continue following—the principle which I stated in the foreword to the first volume (page VIII).

It is necessary here to point out that this volume includes, taken from Folkebladet, a larger number of articles which do not bear Professor Sverdrup’s signature, but which from time to time appeared as editorial articles in that paper. As is well known, Professor Sverdrup wrote over the years a series of editorial articles for Folkebladet, and it was only after I had convinced myself of his authorship through my own and others’ investigations that I included a selection of these articles. There is, of course, always a possibility—although, as I believe, a very remote one—that an error may have slipped in here. These articles are in each individual case marked as editorial articles in order thereby to indicate that they lack the author’s signature.

The historical notes which I have considered absolutely necessary to provide have been inserted at specific places, either as introductory remarks to the individual sections or in the form of notes beneath the text.

The headings chosen by the author himself have, with few exceptions, been retained. Where groups of articles deal with the same matter, I have added such a common title as the content seemed naturally to suggest.

The portrait of the author which accompanies the present volume was taken in 1895 and is considered one of the best from that period.

With regard to the publication of the work, I ask to inform the reader that a change has been made in the order announced in the subscription prospectus, in that the third volume will bear the title: Augsburg Seminary and the Lutheran Free Church.

With the wish for God’s blessing upon it, this second volume of Professor Georg Sverdrup’s collected writings in selection is sent forth to our church people on both sides of the sea.

Augsburg Seminary, January 25, 1910.

Andreas Helland



