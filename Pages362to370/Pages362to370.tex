%FIXME: Recast without the KJV resister.


\begin{center}
\includegraphics[width=0.9\textwidth]{OpenImage.png}
\end{center}

\subsection{The Significance of Mission for the Church}

Source: Account of the 100th Anniversary Assembly of the Lutheran Church, 1906, pp. 91–102. — Ed.

This section appears on pages 362–370 of the original volume. — Present Ed.

\bigskip

\begin{quote}
“I am come to cast fire upon the earth; and how gladly would I that it were already kindled!” Luke 12:49.
\end{quote}

This sigh of longing from the heart of Jesus has been fulfilled. His life was not spent in vain, and His death was not without purpose. The fire was cast upon the earth when the Spirit was sent and the Church was founded upon the unshakable foundation of Christ’s death and resurrection.

For the Church is not simply an association formed by the first Christians after Jesus had ascended into heaven. It is God’s work and His deed, an essential part of God’s revelation among men, and therefore an object of that faith which does not arise from human understanding, but from the Spirit of God. The tongues of fire which descended upon those who were gathered in the upper room in Jerusalem, when the Spirit from God came upon them, did not spring from the hearts of the disciples and were not the expression of a human fervor within them; the flames were kindled from heaven and were the expression of the Spirit’s light and warmth, which henceforth were to dwell with the believers.

For the Spirit was not given in order that He should again depart from the disciples. He was the new Advocate whom Jesus had promised to His own, the Spirit of truth and of love, who was to abide with them forever. Therefore the Church is not only the work and creation of the Spirit, but it is also the Spirit’s dwelling upon earth, according to the testimony which Paul bears to the Church: “Know ye not that ye are the temple of God, and that the Spirit of God dwelleth in you?”

But the Spirit of God is not idle where He dwells. He is the power of the living God, which ever drives those in whom He is found unto all good and God-pleasing works. As Samson was seized by the Spirit and driven by the Spirit to mighty deeds of power, so also that little flock in the upper room in Jerusalem was seized by the Spirit to carry out the great work of love among men: to gather them in unto the Father in heaven, and to unite them into one body with Jesus Christ Himself as Head and Lord.

For the Spirit is He by whom the love of God is poured out into the hearts of those who believe on Jesus the crucified. And since the love of God is world-embracing, Christ’s death is also for all, and an atonement for the sins of the whole world. Therefore it also comes to pass with those who believe on the crucified Saviour that the love of Christ constrains them to seek to win men, inasmuch as they judge this, that if One died for all, then were all dead; and that He died for all, that they which live should no longer live unto themselves, but unto Him who died and rose again for them.

Since, then, the Church has been brought forth by the Spirit of God upon the foundation of Christ’s death for all, therefore the Church, from its spiritual birth, bears within itself this inward impulse, that it must “seek to win men” by spreading among them the testimony of God’s world-embracing love and the salvation which He has prepared for all peoples. Therefore missionary zeal is not merely a virtue among other Christian virtues, more or less necessary for the believer; but it is an essential and peculiar manifestation of life of the Spirit of God in and through the Church.

Or can we forget the effect which the love of Christ had upon those disciples of the Baptist, who, when they had found the Messiah, must also find their brethren and friends and lead them to Him? There can be none who has the Spirit of truth and does not burn with zeal to bear witness to the truth; there can be none who has the joy in Christ and does not desire to share it with others; there can be none who has the love of God poured out in his heart and does not feel the inward compulsion to labour for the salvation and blessedness of all.


Here it is that the bond between the Congregation and the Mission, or the Kingdom of God upon earth, comes into view. For it is the divine appointment of the Gospel to spread the rays of salvation’s light over the whole earth, and to send them into the dark recesses and hidden depths of human life and of the human heart. And this Gospel has been entrusted to the Congregation, not that it should keep it to itself and bury it in the earth, that it might be the more safely preserved as a dead treasure, but precisely that the Congregation should let it shine both at home and abroad for the salvation of the human race. Thus it is in this, first and foremost, that the Kingdom of God consists. It bears within itself righteousness, peace, and joy in the Holy Ghost, and advances, not by oppressing and despoiling the nations, but by setting them free and giving them the heavenly gift which, in the fullness of humanity’s time, has been sent through the Son and the Spirit.

The Kingdom of God is a victorious kingdom of conquest, yet it wins its victories without the clash of swords and without shedding of blood. Indeed, wars and judgments from God often pass over peoples and kingdoms before the Gospel’s peaceful and gentle advance in the world; yet it may rightly be said that this is as the earthquake and the fire which upon Mount Sinai went before the still small voice that caused the prophet to cover his face with his mantle. For therein was God’s true revelation. Yet a conquest it is nonetheless—the most perfect conquest and the most complete victory, which makes of enemies friends, and of assailants the most zealous defenders. So it went with the snorting, wrathful Saul, who became the fervent laborer and Apostle Paul: he who before his conversion desired only to do evil against Jesus and His disciples, but who after his conversion was willing to suffer anything, if only the name of Jesus might be glorified, if only human souls might be saved through faith in the Crucified One.

No earthly kingdom can bar the door against the Gospel when the Lord’s hour to enter has come. No persecution can annihilate the Kingdom of God among men once it has found its way into a people. Though it may appear as though it is overcome and disappears, it is yet only as when the grain of wheat disappears into the earth in order to grow all the more richly thereby.

But this holy conquest—peaceful though it be—is nevertheless against the world. It disturbs the power of heathendom and bursts its bonds asunder. It makes souls free from the dominion of the world and of the flesh. It overthrows the tyranny of sin and disturbs the sleep of death. And the world awakens to bitter resistance and mortal hatred. Therefore the Mission, or the advance of the Kingdom of God upon earth, is also accompanied by great sufferings for the little flock which, in faith in Jesus and in His following, strives for the truth and for the peace of the heart. No one can take part in disturbing Satan’s strongholds and the world’s evil dreams without receiving blows and wounds in return. It would not be reasonable to expect otherwise; nor is there any other course. Yet he who prevails shall not regret that he was in the battle, for the crown is found where the wounds are.


If, after this glance at the biblical presentation of Mission and its connection with the Congregation and the Kingdom of God, we now look to the actual condition within Christendom, then we find— with horror, I should hope— that there are many Christians, at least in name, who do not appear to take any part in this warfare; they do not suffer its hardships, neither do they feel themselves thrilled with blessed joy at its victories. And worse still— comparatively few are the congregations which wholly and fully take part in the world-conquest of the Kingdom of God, and which make it a chief concern, a principal aim of their existence, to carry on missionary work at home and abroad.

It is therefore in the highest degree necessary to ask: What significance has Mission for the Congregation? Or does it perhaps make no difference whether the Congregation takes part in missionary labor or not?

And here it is not meant whether the Congregation contributes an offering or two to Mission during the year; rather, it is meant whether it is a chief matter for the Congregation to press the Kingdom of God further onward and the Gospel of God further out, so that it truly exerts itself in prayer and labor to that end. For I would note that it is a war against the powers of darkness of which we here speak; it is a struggle to save souls from death and to deliver them from the dominion of Satan. And this cannot be done without costing both exertion and sacrifice.

But does the Congregation, and the individual Christian, then gain anything by taking part therein?

To this question the answer would soon be given, if we could take the standpoint of Catholicism. For to it Mission is also a means of extending the power of the Church and increasing her glory; and naturally it is of great importance for every faithful son or daughter of the Church to be engaged in the work whereby the arm of the Church may reach unto the farthest regions of the earth.


Not so among us; the evangelical mission to the heathen and the home mission are not carried on in order that the Church might become world‑ruling, but in order that the Kingdom of God might be spread over all the earth, and that saved souls might be gathered from all peoples under heaven.

And thereby mission does not receive less significance for the Church, but rather greater. For in the evangelical Church it is not a question of winning power, but far more of preserving life and bearing the fruit of life.

God’s Church has been made a partaker of God’s life, and it is this which is at issue—to preserve it and to bear fruit from it. And to this end it is also necessary to take part in mission; for it is indeed God’s eternal counsel of salvation and His revealed will that all men should be saved and come to the knowledge of the truth. And this eternal will is carried out through the Church’s work of mission; and the Church is only then truly permeated by the Father’s eternal counsel and heart, when it is filled with zeal for the raising up of the human race from the Fall and for its return to the Father’s blessed communion. It is a matter of life for the Church to drink from this fountain of eternal love, and, strengthened thereby, to do the work of God.

And the Father’s will has first and foremost been realized in the world through the Son. His life and work, His suffering and death, His resurrection and ascension, all have this one purpose: to seek and to save that which was lost. And this which was lost is the whole sinful human race, and each individual sinner within it. And not only is it Jesus’ purpose to save the fallen race; it is also His command to all His disciples that they are to take part in this work. Being themselves saved, they are to labor for the salvation of others through the Gospel of Jesus Christ. And for the Church it is therefore a question of life or death when it comes to the question of being obedient to the command or not. He who died for all, that all should live, He wills that His body, which is the Church, should work toward the attainment of His purpose and the realization of His will. He who himself lives upon the ground of Christ’s death can preserve this life in strength and freshness only by bringing the Gospel of Christ’s cross out to those who sit in the darkness of death; for the love of Christ is preserved only in love toward those for whom He died.


And the congregation does not merely need to obey Christ’s command concerning mission; in this matter it must also follow the drive of the Holy Spirit. It is dangerous to resist the Holy Spirit. We know well that the Jewish people made themselves guilty of this trespass. They would not themselves believe in Christ; neither would they permit the Gentiles to believe in Him. In every way they sought to hinder salvation from reaching the hearts of men. We see their judgment and punishment before our eyes; shall we walk in their path? No—when God’s Spirit urges us to seek salvation for ourselves and to labor for the salvation of others, then let us follow the call, and God’s will and His work shall not be bondage to us but freedom.

It is not difficult in this way to recognize that the congregation does not attain full power in its life, nor complete devotion to its calling, until in the work for the spread of God’s Kingdom it comes into harmony with the saving counsel of the Triune God and its execution in the history of the human race.

And if we consider the very nature of the Kingdom and the manner in which it grows, we shall soon see that just as the congregation has great significance for mission, so also mission has great significance for the congregation.

Jesus has compared the Kingdom of God to leaven, which a woman took and hid in three measures of meal, until the whole was leavened. If it now appears that the meal is not leavened, it is unavoidable to ask whether the leaven has been sound. And if it appears that the evangelization of the heathen world proceeds slowly, it becomes a natural question whether there is the proper power in the congregation. In other words, where mission is weak, there responsibility inevitably returns to the congregation.

The Kingdom of God is, as we have seen above, a kingdom of conquest. If it does not advance in victory, the question is raised whether the army is awake and active. Necessarily, a vigorous missionary work must contribute to keeping the congregation watchful and militant. And each individual congregation is a troop within the great army as a whole, and we all know that the welfare and progress of the whole depend upon the strength and endurance of each small part. The congregations are therefore responsible first and foremost to God, but thereafter also to one another in this connection. If Christ’s work is to be carried out and victory won along the entire line, then the faithfulness of each individual soldier and each individual troop in battle and labor is decisive. Nothing sharpens the sense of responsibility more than lively participation in the great common work of all Christendom. And nothing can be more beneficial for the congregation than precisely this sharpened sense of responsibility.


It is a great and weighty cause that follows from this position of the congregation as a link in that conquering host which wages the Lord’s war in the world. Missionary labor is an essential part of the spiritual struggle for freedom that is waged against the tyrannical dominance of paganism and worldliness over the natural mind. If the congregation enters this struggle for the Spirit’s freedom in earnest, it will therein also find a mighty help in the work of its own liberation. And this is of surpassing significance precisely in our time and in the position we have, taken as a whole, been given within it.

If we rightly discern the signs of the times, there is more life than freedom. It is possible to live, in a certain sense, almost without freedom. But it is a wretched condition for the spirit to be held in the bonds of a legalism, half suffocated between the demands of the world and the compulsion of the Law. And yet there are not so few congregations that endure a miserable existence in precisely this manner. It is not possible here to enter into a more detailed explanation of this tangled and desperate condition. We can only say that there is an exceedingly great need for deliverance and, comparatively, little labor undertaken to relieve this need.

But here a vigorous missionary endeavor comes to the congregation’s strong aid. For if the congregation truly means in earnest the Christian conquest of the world by the Gospel, it will also of necessity come to the recognition that it is treason against Christ and the kingdom of God to nurture worldliness and pagan superstition in its own midst, while it fights to overcome these same enemies elsewhere. And nothing contributes more to the congregation’s liberation than that it loosens the bonds of materialism and at the same time gains greater room for love in heart and in deed.

These matters now mentioned are perhaps the most essential in this connection. Yet there is, besides, a multitude of things that are “added” for those who throw themselves with all their might into missionary labor.

It may be worth mentioning a few of them.

There is the fellowship between Christian persons and congregations without regard to the boundaries of confessions and church bodies, as they are set face to face with the immeasurable task of making all nations disciples of Jesus, and face to face with the nameless misery which paganism brings upon humanity. Mission continually brings us into fellowship with unnoticed Christian men and women of various confessions, with whom brotherly cooperation is often practiced under such conditions that it may be called unavoidable. For who, out in the dreadful darkness of paganism, would not rejoice in brotherly communion with the children of light?


There is another matter which stands in close connection with this; where Christianity and heathenism meet in a struggle of life and death, or where it is a question of rescuing souls from death, the Christian fundamental truths concerning atonement, forgiveness of sins, faith, spiritual life, and the hope of glory become so overwhelmingly important that the differences which exist among evangelical Christians are necessarily pushed back into a more subordinate position; and this prepares the way for a simpler understanding within the congregation and a corresponding greater concord among Christians with one another. And the more this inner, spiritual unity is fostered and advanced, the less significance do the outward separations and associations assume, so that we move ever nearer toward the goal which the Lord has set for His own upon earth: that they, gathered into congregations each in its own place, yet together constitute but one flock under one Shepherd.

Furthermore, the work of missions will place the congregation in its right relation to the question of nationality and race, and will set it free from the bonds by which it is bound in this respect. The truth which the Word of God sets forth—“Here there is neither Greek nor Jew, circumcision nor uncircumcision, barbarian, Scythian, bond nor free, but Christ is all, and in all”—has indeed been so grievously thrust back in the consciousness of the congregation. But mission will once more bring it forward and make it living for us through the fellowship in Baptism, the Gospel, and the Lord’s Supper.

Through the living and spiritual participation in the work and conflict of the Kingdom of God, the congregation will be delivered from the bonds of the nationalities with their partiality, injustice, and hatred. And freed from these wretched pettinesses, the congregation will feel the liberation which lies in the world-embracing brotherhood of the Kingdom of God. We shall learn—though but little by little and late—to stammer forth that glorious confession: Here there is neither black nor white, Norwegian nor Swedish, German nor Danish, Englishman nor American; but here is the eternal fellowship of our Lord and Savior Jesus Christ: “You are all brother.”

Perhaps some may think that we have had chiefly the outward mission in mind when we have here spoken of mission and its significance for the congregation. This, however, is not the case. We have here in our circumstances both home mission and heathen mission. For the goal of mission is the founding of a congregation, whether it be carried on at home or abroad. Mission has reached its goal where the congregation independently takes up Christian work of edification. What in Europe is called inner mission is another branch of the congregation’s activity, which one has in part with greater justice called rescue work. Both home mission and heathen mission spring from the same root, are driven by the same Spirit, are equally necessary for the congregation, and have at least in part the same value and significance for its spiritual insight and life.

For our own people and for all peoples alike we have the one and the same command from the Lord: Go ye therefore and make disciples of all nations, baptizing them in the name of the Father and of the Son and of the Holy Ghost, and teaching them to observe all things whatsoever I have commanded you. There is no difference. And no one can value too highly the significance of obedience to the Lord and to His command.

And therefore it may at last be said as a solemn admonition to us all: heartfelt and vigorous participation in the work of mission at home and abroad is an essential mark of the living and free congregation. The dead congregation bears the mark of self-interest; it labors for its own outward welfare, for it knows no inward; it labors for itself and its party, its outward greatness and power. The living congregation works in the self-sacrificing service of love for the salvation of souls, for the spreading of the kingdom of God, for the glory of God and the advance of His Gospel. Even though small and insignificant, it exalts its Savior through self-surrender; and in the unshakable certainty that when it gives up its life in the service of God, then it wins it; and when it becomes a grain of wheat that is laid in the earth, then it springs forth into imperishable harvest for the Lord.

May it be granted to us, brethren, even in great frailty, to set forth God’s living and free congregation among men, and to set forth ourselves in and through it as a living, holy, and God-pleasing sacrifice.

So be it, so be it, for Jesus’ sake! Bloodied and torn does the sacrificial bread appear upon the altar; yet if it be thus received by the Lord, the suffering and bleeding heart shall indeed find courage for every lack and healing for every pain with Him who Himself is the Man of Sorrows and of love, Jesus, our forever highly praised Savior. Amen.




