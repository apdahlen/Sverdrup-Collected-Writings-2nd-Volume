%FIXME: Recast without the KJV resister.

\subsection{The Pastoral Office and the Priesthood of the Laity}

The New Testament bears clear witness that the first Christian congregations possessed both a distinct teaching office and a free exercise of the Lord’s Word in their midst. To the Ephesians Paul writes: “And He gave some, apostles; and some, prophets; and some, evangelists; and some, pastors and teachers; for the perfecting of the saints, for the work of the ministry, for the edifying of the body of Christ” (Eph. 4:11–12). No one can deny that reference is made here to persons who, by their gifts, were qualified to be the congregation’s officers. Christ Himself gave them these gifts, raised them up in the midst of the congregation; the congregation chose them (Acts 6:5; 14:23) under the powerful cooperation of the Spirit of God (Acts 20:28), and laid hands upon them (Acts 13:2).

To the Colossians the same apostle writes: “Let the word of Christ dwell in you richly in all wisdom; teaching and admonishing one another in psalms and hymns and spiritual songs” (Col. 3:16). And again no one can deny that here the apostle speaks of each and every believer without any regard to congregational election or to any sort of “outward call.” All who have the Spirit of Christ are here exhorted to praise the Lord’s Name for salvation, in such a manner that it may be unto edification for those who hear.

It is therefore a matter of first importance for the Church of Christ to have an ordered office in its midst, filled through the election of the congregation. It is an equally serious requirement laid upon the congregation, that its members bear witness to their Saviour from believing hearts and from the lived experience.

Here, then, in our Free Congregations—where not ancient custom and usage alone may decide what is right, but where the Word of God must ever anew be the foundation of our congregational order—here it comes above all to lay the most earnest weight upon both these matters. What we must rightly give heed to is this: that the Office does not hinder the free proclamation of the Word, and that the free proclamation does not usurp the work of the Office.

There is a logical chain of reasoning which will always be brought forward against lay activity in the Free Congregation. It runs thus: “The congregation has the right to proclaim the Word and to administer the Sacraments. Since not all can exercise this right, it is entrusted to a single man (the pastor) by the congregation’s election. Thereby the one elected becomes the sole possessor of the right, and therefore no other may either preach or administer the Sacraments. Whoever preaches or teaches, apart from the pastor, is therefore a thief and a robber.” There is another chain of reasoning, no less logical, which will always be used to annihilate the distinctive position of the Office in the congregation. It is this: “God’s people are a people of priests, called to proclaim the virtues of the Lord; since no one can take this calling away from God’s people, the election of a pastor is no transfer of proclamation to a single man; but when one is chosen in particular, it is only that he may be ready to preach when necessity does not fall upon any other, and for the sake of order he should perform the ‘ministerial acts.’ Therefore, if the pastor preaches and teaches in the congregation as one who has the right and duty thereto continually and as a daily calling, then he is one of those who would lord it over the Lord’s flock.”

Both of these chains of reasoning appear excellent, and yet they both end by coming into conflict with the Word of God. It is against God’s Word that a single man alone should wholly possess the right to bear witness and to teach in a congregation; for where then is the exhortation that Christians should “teach and admonish one another”? It is against God’s Word that the one who is chosen as teacher in the congregation should not thereby have received a particular calling which the other members of the congregation do not have; for where then is the Word of God that Christ appointed some as apostles, and so forth; and the commandment “that they which preach the gospel should live by the gospel”?

The error in both of these opposing chains of reasoning lies in this, that they do not sharply distinguish between two different things: the task and duty of the congregation, and the task and duty of the individual Christian. It is the task of the whole congregation to proclaim the manifold wisdom of God (Eph. 3:10); it is the duty of each Christian to proclaim His praises, who hath called them out of darkness into His marvellous light. The individual cannot seize for himself the work of the congregation, and the congregation cannot deprive the individual of his work. If a state owns a common property, then no single man may take it as his own; but conversely, no state has the right to rob the individual of his property. What is common property we may entrust to an office-bearer to administer on our behalf; but what is our private property, that we each administer ourselves.

When, therefore, a congregation chooses a man made fit by the Lord to be its public teacher, he is thereby set apart for a particular calling; he becomes the congregation’s office-bearer, and it shall be his life’s task—and therefore also his means of livelihood—to be a preacher of the Gospel, a steward of the Sacraments, a servant of God, who without hindrance can devote himself to the study of God’s Word and to its dissemination. He shall daily penetrate into the whole counsel of God and become a true scribe, instructed for the kingdom of God, able to bring forth out of his treasure things old and new. He shall be an instrument of God to set forth from God’s Word Jesus, the crucified Saviour, the true constitution of Christ’s Bride, and the true nature of the individual Christian. He shall be the shepherd of the congregation, who leads it upon the right pastures; the watchman of the congregation, who by God’s Word fights against sin within it and deception round about it. He shall, so to speak, be the conscience of the congregation, in that by God’s Word he reproves sin and encourages faith, being an ever-living reminder of what the congregation is and of what a Christian is. Yet it is so far from the case that he should be the only one who lives and works in the congregation, that on the contrary there is great danger that he will not be what he ought to be if he becomes alone in life and labour. As a heart cannot beat the pulse of life in a dead body, so with great difficulty can a pastor be a true pastor, if his activity and labour do not give birth to activity and labour all around him.

It is therefore the congregation’s great common labor with the Word of God, inwardly and outwardly, which is most immediately entrusted to the pastor; yet it is very far from being the case that thereby all the individual members’ Christian duty is laid upon him. The duty of Christians, each in his own place, to bear witness to the Savior in word and deed thus remains resting upon them. There are pastors who would gladly be alone in the activity and life of the congregation, who would gladly have their congregation become a dead mass that neither lives nor works; for that is the easiest thing for the pastor. And there are congregation members enough who would gladly have the pastor both live and work in their stead; for that is the easiest thing for them. But this is in any case nothing other than sheer Catholicism, and it ought not to find spokesmen within the Lutheran Church. Holy Scripture describes the individual Christians to us as both living and working, witnessing and prophesying, and it is altogether inadmissible to let all this be done on one’s behalf by a salaried pastor in the name of the office.

It therefore stands fast that the congregation’s call of a pastor does not abolish the Christians’ activity for the building up of Christ’s body, but rather confirms it. The pastoral call places in the midst of the congregation a living center for the work of the congregation, and around the labor of the pastoral office there gathers the individual Christians’ work of edification. The office is like the firm trunk of the work, which unceasingly is active in bringing nourishment to the other branches of labor. The work of the office is regular and fixedly prescribed and orderly determined; the work of the individual Christians is more dependent upon occasion and circumstance, upon the Spirit’s particular gifts and impulses. It is the concern of the congregation to watch over this: that the work of the office be carried out with zeal, and that the activity of the individual members take place unto edification and not unto disturbance in the assembly of God.

But how far, then, does the individual Christian’s duty of witness extend? Can any law or rule be set for it? The Word of God gives this general rule: “Let all things be done unto edification.” And the Word testifies that each one shall use his gift for the edification of the congregation in the one and the same Spirit. It is exceedingly dangerous to draw other boundaries. Naturally, it says itself that under extraordinary conditions of distress the duty of the individual may extend so far that he must directly seek to perform the work of the office, although this in free congregations is almost unthinkable, for the reason that the congregation will then intervene and seek to remedy the distress in a regular and orderly manner. But under ordinary circumstances, when the office is carried on in the right spirit and according to the Word of God, there is no other rule to be given than that which is set by the diversity of the gifts and the unity of the Spirit in connection with the various earthly callings. For there arises a necessary and justified limitation upon the individual members’ work of edification from this very fact, that each has his own daily labor with earthly things, whereby he must and shall seek his daily bread and the necessary means to maintain the office and the congregation among them.

We cannot, however, refrain from mentioning that both congregational meetings for the governance of the congregation and edification meetings for the congregation’s growth in faith are so natural and regular a form of the exercise of the gifts that it is exceedingly difficult to conceive of a free congregation without them. That congregational meetings are essentially a work of laypeople admits of little doubt, even though there are many congregations in America where the congregational meetings as well are only a form of the pastor’s sole activity. Yet congregational meetings have of themselves become unavoidable everywhere.

Edification meetings, in which laypeople both pray and exhort, on the other hand seem to be the object of bitter hatred on the part of many pastors. And yet it is utterly incomprehensible how anyone can find room for the gifts of the Spirit without such gatherings being held. Whether the pastor is present or not; yes, even if the pastor is opposed to them, it nevertheless appears altogether necessary for a true congregation to seek to employ all its powers in this way as well.

It is at any rate certain that the gatherings of the first Christians were essentially of this character, and it is a known fact that at all times in the history of the Church, whenever there has been any spiritual life, the need for these gatherings has been so strong that neither clerical prohibitions nor police laws have been able to prevent them. From such gatherings the Lord’s witnesses have been brought forth before tyrannical judges and have been banished from the land and cast into prison and slain; and yet again and again the people of God have always found time and place to assemble for the purpose of praying with one another and for one another, and of being strengthened by brotherly exhortation and instruction.

Strange indeed would it be if Christian people, when they assemble, should not be permitted to assemble around God’s Word and converse with one another and instruct one another concerning the wondrous works of the Lord! We must therefore rejoice with joy over every congregation where such gatherings are held, and we must grieve and lament over every assembly of people that calls itself a congregation and yet has no gift of the Spirit which it can in this manner employ for its own edification.

It cannot, however, be concealed that just as there are pastors who are not true shepherds in the congregation, so there are also laymen who speak the Word of God from purely fleshly motives. Yet if the ungodliness of certain pastors does not give us the right to abolish the office itself, neither do the failings of certain lay preachers give us the right to forbid the free proclamation of the Word in the congregation by laymen. Let us do what we can to stir up the gifts of grace and to promote their right use; and when we have done what lies within our power for that cause, then by that very fact we have also gained the authority to reprove abuses. But he who does nothing to enable the congregation to build itself up and to labor and work in every direction for the spread of the kingdom of Christ—let him himself rather be silent than attempt to lay a muzzle upon others.

The Norwegian-American congregations are exposed to danger from persons who seek to press their way into the pastoral office without possessing even the most elementary qualifications for it. When such men enter the office, they are exceedingly prone to untimely lust for domination and to a desire to hinder all lay activity. On the other hand, there is danger from such laymen as, out of sloth or pride or for the sake of gain, will travel about and hold edifying meetings instead of working with their own hands. If these first gain some following, they will readily despise all congregational order and seek to awaken suspicion against the pastors wherever they go, perhaps in order to be chosen as pastors in the place of those who waver. Against both of these dangers there is one remedy: the life and freedom of the congregation. Earnest godliness and manly independence in the congregations, and a true, sincere cooperation between pastors and laymen—this is the safeguard against the disorders which bad pastors and bad lay preachers will bring about under our free conditions. It is a matter upon which we ought all to be able to agree in the Free Church, that the authority of the congregation ought to be able to set boundaries wherever it becomes necessary, both against the encroachments of the office and against the errors of the layman. But the rule of the congregation is this: Let all things be done unto edification!

\subsection{The Diaconal Office in the Congregation}

It is impossible for us to pass from speaking of the individual local congregation in the Free Church to speaking of the association of congregations within a fellowship, without first treating separately of an exceedingly important office in the congregation, namely, the diaconate. It has already so pressed itself into our ecclesiastical consciousness that we can scarcely imagine a free congregation as organized if it does not have its deacons. Yet hitherto so little has been written concerning this ministry, and so few clarifying accounts have been provided as to how it is exercised in the various congregations and within the different fellowships, that one must generally be content with the brief and dry legal provisions in the congregational constitutions when one seeks to form a conception of the place the deacons truly occupy in our congregations. Certain it is, however, that on the one hand, in many a congregation in America, much faithful diaconal service is rendered in secret and before God, while on the other hand the diaconate in many places only very imperfectly fills the place it ought to have.

It belongs to the Christian congregation’s confession of sin that it both knows and feels with pain that it is imperfect in all its work; but when we must especially emphasize that there is often much lacking in the care with which the work of the diaconate is carried out, we mean thereby more than that imperfection which clings to all the congregation’s work on earth. There is, to be sure, often lacking the proper understanding of what the work of the deacons truly is, both among the congregations who elect, and among the men who are elected to this office. Therefore those men are not always chosen to whom the Lord has given the gift to carry out this service; neither can the gifts rightly be developed, trained, and cultivated where there is lacking a clear and thorough insight into the true significance of the ministry. In part this has its cause in the fact that this office is not known from the church of the homeland, nor has it been taken up again through an independent development within Norwegian congregations, but must rather be regarded as transferred from American congregations to ours. It may therefore be of some benefit to attempt to provide certain explanations of what this office is, and how it may best become a blessing to congregational life.

Deacon means a servant, and the name is therefore simple and humbling to our flesh; yet it has a wondrous sweetness and beauty for the Christian, for whom it is a name of honor, since the Son of God voluntarily took it upon himself and came not to be ministered unto, but to minister, and to give his life as a ransom for many.

It is the greatness of all Christians to become small and the servants of all, in order to win them for the eternal life, if it were possible. Thereby they walk in the footsteps of their Lord; they follow him in humility and suffering; they love with him, labor with him, strive with him, devote themselves with him, that the glorious blessing of reconciliation may, both by word and by deed, be brought near unto as many hearts as possible. It belongs especially to Christians to practice the service of love where need and misery have made the sinful earthly life truly wretched, where sin has brought great distress and suffering, and where suffering has often softened hearts, so that they are receptive to the right remedy against both sin and need and death—the precious Gospel of our Lord and Savior. Just as the Savior himself found no entrance among those who were full and rich and had abundance and lacked nothing, but was all the more welcome among those who suffered and were afflicted, so also have Christ’s believers in a special manner become the servants of the suffering and the poor and the outcast, if indeed they have become faithful in the calling wherewith they are called.

The narrative in Acts, chapter 6, shows us clearly how the diaconal office was instituted. And it emerges quite plainly from this account that the deacons were chosen by the congregation to carry out a service which in no way was of a different kind from that which all Christians are bound to show one another. Yet here we find the same circumstance which we earlier touched upon when we spoke of the priestly office and lay activity. All Christians are to be priests and witnesses, and yet the congregation as a whole must have its priest, who is an executor of the whole congregation’s duty of witness; likewise all Christians are to be servants or deacons in the footsteps of Jesus Christ, and yet the congregation as a whole will be able to exercise its duty of service only through particular deacons, chosen thereto. It is also noteworthy that the account in Acts, chapter 6, shows us that this office has its ground in murmuring and dissatisfaction, that it is thus conditioned by imperfection among Christians; from this we may better understand what the Apostle means in Ephesians 4 when he says that the gifts and offices in the congregations are to serve “for the perfecting of the saints.” Through its offices and officeholders the congregation exercises self-control, self-cleansing, and self-edification.

That which threatened to bring about a division in the first congregation, and which gave rise to the diaconal office, was a complaint that the Greek-speaking widows were being neglected in the daily distribution. It was thus manifestly a question concerning the proper assistance of those in need. And the service which was to rest upon the deacons is said to be this: to serve at the tables. We recall that the first congregation held its property in common. They ate together, and these common meals were therefore to be overseen, ordered, and provided for by the deacons, who were also to see to it that no one was wronged in any respect. From this there followed, as a matter of immediate necessity, that if any member of the congregation could not come to the common meal because sickness or distress hindered him, then the deacons were bound to see to it that such persons also were fed and helped. Thus we find that later in the history of the Church, when the common meals ceased, the direct care of the poor and the sick became the official ministry of the deacons.

It is, however, self-evident that just as far as the pastoral office does not exclude the activity of the laity, so far neither does the diaconal office exclude private charity. We must here remember that in those times the world was full of poor people, and that especially the great cities were, to an extraordinary degree, crowded with the needy and the helpless.

The apostles prescribe in Acts 6:3 that the congregation is to choose men who have a good testimony, who are full of the Holy Ghost and of wisdom. And we find that the effect of this new order in the congregation, which brought the complaint to an end and gave the apostles occasion to devote themselves exclusively to the ministry of the Word, was great and blessed. “And the word of God increased; and the number of the disciples multiplied greatly in Jerusalem; and a great multitude of the priests were obedient to the faith.” We also find that some of the deacons took a vibrant part in the proclamation of the Word. From this we may conclude that the apostles regarded this office as an exceedingly essential and important link in the development of the congregation. It is evident that they desired the best men of the congregation, and that in those who were chosen they found a great support in their work, and that the congregation reaped a great fruit from thus having the right men in their right place.

Later, when the Church consisted of many congregations, and the other congregations imitated the congregation in Jerusalem in its organization, we find in 1 Timothy 3:8–13 a description of what a true deacon ought to be. Paul writes concerning this as follows: “Likewise must the ministers of the congregation (deacons) be grave, not doubletongued, not given to much wine, not greedy of filthy lucre; holding the mystery of the faith in a pure conscience. And let these also first be proved; then let them use the office of a deacon, being found blameless. Even so must their wives be grave, not slanderers, vigilant, faithful in all things. Let the ministers of the congregation be the husband of one wife, ruling their children and their own houses well; for they that have used the office well purchase to themselves a good degree, and great boldness in the faith which is in Christ Jesus.” These words, which ought to be read and pondered again and again in our congregations, which each year choose deacons, give the most excellent guidance for choosing the right men; and we all know that in the free congregation it is not a matter of many offices and lofty titles, but of the right men in the right place.

If we now consider our own circumstances and our congregations, it is striking to see how in many ways we find the experience of the apostles confirmed, that not many rich, not many mighty, not many wise after the flesh are called. Among us also there is, in part, poverty and need within the congregations. And yet there is a vast difference, inasmuch as we in this land can form no true conception of the utter helplessness in which thousands of the poor in the apostles’ time were placed. What poverty we see here is as nothing compared with the misery and wretchedness which the great cities of the Roman Empire hid within their walls. A natural consequence of this is that our deacons have comparatively little work with the direct care of the poor, but are instead especially directed to think upon the sick, the dying, and the afflicted as that field of labor to which the Word of God points them.

We must also take note that in the apostolic congregations there was commonly a chosen presbyterium alongside the deacons. We do not choose several pastors, but only one pastor; and from this it will follow quite naturally that our deacons, who often together with the pastor form the congregational council, also come to deal with various matters which in the earliest time belonged more properly to the presbyters. In this there is nothing improper. For those who are fit for the office of deacon are also fit for the office of presbyter; and where the care of the poor recedes so far into the background as it does in our congregations, it would not be necessary to have a separate congregational office devoted exclusively to that matter.

Most of the regulations governing the activity of deacons in our congregations are directed toward this end: that the deacons are to be the pastor’s assistants in the spiritual care of the congregation; that they are to visit the poor, the sick, and the afflicted; that they are thus to be the congregation’s chosen and commissioned servants, appointed to carry out, on behalf of the congregation and with responsibility to it, a work which in private also rests upon every individual Christian. But since the charity and help of the individual in every kind of distress always has something accidental about it and cannot be relied upon to reach everywhere, and may even lack the proper wisdom and insight, the congregation also, as a public body, must strive to have this ministry rightly performed through chosen men.

If, then, great care was required in the apostolic congregations in the choice of deacons, this is surely no less the case in our own day. Or have our congregations advanced further in faith in Christ and in fitness to serve one another? We must rather lament that sin and unfaithfulness in the work have increased within the congregations. It entails a great responsibility to possess the freedom and right of election of the apostolic congregations; for if we do not choose with insight and with the guidance of the Holy Spirit, we hinder the kingdom of God from coming to us, in that we obstruct the proper application of the gifts to labor in the congregation. But if we rightly esteem the Lord’s manifold gifts and the workings of the Spirit in the various persons, then we know that small means can accomplish wondrous things in the kingdom of God, provided they are used in the right place and in the Lord’s service.

It is a serious matter in the free congregation to elect all its office-bearers, and it is a very burdensome task to be an office-bearer in the free congregation. This applies especially to the pastor and the deacons. What should in particular be the deacon’s striving is the imitation of Jesus as helper and comforter in the manifold distresses which congregational life presents. Even if poverty is not precisely in these days the greatest outward distress within our congregations, it is nevertheless far from the case that life has thereby been freed from misery. In a thousand forms the misery of sin confronts us also among ourselves. There is sickness; there is drunkenness; there is family discord; there is unbelief and despair; there are temptations of every kind. Who can count all the forms of sorrow and distress, who can count the tears that are shed, often where no one suspects it? But to enter into the house of sorrow and the dwelling of need, and to receive grace to wipe away a tear, to comfort a heart, to guide one who has gone astray, to admonish one who is disorderly—this is the lovely work which the congregation lays upon its deacons. May the Lord grant that there be many who are fit for this work, many who are faithful in its often heavy exercise.

For the pastoral ministry the diaconate is an indispensable support in our congregations. They can give notice of sickness and spiritual distress here and there; they can be with the sick when the pastor cannot. They can assist in edifying services; they can conduct worship when the pastor is hindered. They can pray with him and give counsel and help where a solitary man may often be both at a loss and helpless. It is immediately evident that it is precisely the same now as in apostolic times: if the pastor is to be the holder of all offices and perform all works in the congregation, then his chief work is neglected; he receives so many tasks that one hinders the other. Such a condition is equally unhealthy for congregation and for pastor; regrettably, congregations in many places would gladly have it so, and in still more places pastors themselves would gladly have it so. Yet many there are also among us who are faithful and discerning pastors, who recognize that the freedom and life of the congregation, and the heavy responsibility proper to the pastoral office, require a division of labor. May they become more and more among us; then they would also, through steady labor for this cause, experience that the congregations even in our day do not lack the gifts of the Spirit for their own edification. But where nothing is done to test the congregation’s strength, one may be certain that the strength and the gifts will hide themselves; yet the congregation will incur guilt, because it buried its talent in the earth.
