%FIXME: Recast without the KJV resister.

\subsection{The Ministry in the Free Congregation}

There is scarcely any question within the Christian Church that has occasioned so many difficulties and so much strife as the question of the ministerial office. Personal passions, zeal and envy without understanding, temporal pride and spiritual immaturity have continually and persistently exerted their influence so as to cast confusion into a matter which concerns not only the upbuilding of the Kingdom of God in the land, but also many personal interests of a purely temporal kind. It was unavoidable that, since ministers are human beings, they should be subject to the frailty of earthly conditions; and when temporal interests became so inextricably bound up with the work for the Kingdom of God, many frictions were bound to arise.

The free church community has no way of escaping this difficulty. For although the state church may have made the question more complicated and rendered its proper solution impossible, the real struggle concerning the right place of the ministry nevertheless begins in earnest only within the free church. In the state church the ministry is once and for all placed in a thoroughly distorted position, and there is no one who expects or demands that it should be otherwise; but in the free church the demand arises that the ministry shall be in its proper place, and since this is a most burdensome place, the struggle comes both from hostile ministers and from dead congregations against the very nature of the ministry and its authority.

There is nothing in the Word of God that lends any support to the fleshly mind. And although both bishops and popes have labored to interpret the Word of God to their own fleshly advantage and in favor of a proud clerical aristocracy, it is nevertheless just as certain that there is no room in the Word of God for a great and mighty priestly estate that claims a special divine authority and a peculiar position between the congregation and God, different from that of any other Christian. But if the Word of God serves no clerical estate, it all the more strongly emphasizes the necessity of the ministry, of the service of the Word, of the work of proclamation, which is to be carried out by the congregation through such Christian persons as are fitted for it by the gift of grace which the Spirit of the Lord has given them. Thus, just as far as Roman priest-exaltation is removed from faith in Christ and from the truth, just so far is the Quakers’ priestlessness removed from the sound form of God’s congregation.

We thus soon see that here the way of truth is so narrow, so sharp as the edge of a knife. And it is clear that conflicts were bound to arise here. In America these have assumed such a form that on the one side the claim has been made that minister and congregation are bound to one another as husband and wife—naturally with the understanding that the minister is the husband, and that he thereby enjoys several advantages which the wife, or the congregation, does not possess. Likewise, from the same quarter, the greatest efforts have been made to suppress so-called “lay activity” within the congregation. On the other side there has been advanced the sinful assertion: “The minister is the servant of the congregation,” and with that one has thought to have decided the whole matter without further trouble. Both of these views, which in fact may be summed up in the question, “Is the minister the lord of the congregation or its servant?” are equally unfit to give the ministry the significance and the position which the Word of God assigns to it.

In the State Church, the Office and the Priest are at once placed in the most erroneous relation to the congregation, in that the priest is a “royal civil servant” and is appointed by the King without any manner of participation on the part of the congregation. No arrangement can be conceived by which one could come farther away from what is Christian truth in this matter than this. It is absolutely torn loose from every root in the Word of God; it stands so completely outside all connection with congregational life that nothing else can be expected than that it must do harm. And harm it does.

For the priest comes to the congregation as a stranger, remains there as a stranger, and is “transferred” to a “better” calling, without the congregation receiving any other impression than that it has now been the man’s means of livelihood for a time, until he could obtain “something better.” That this is not a practice favorable to the work and fruit of the office, anyone can see.

And if the priest himself is also a man who labors as a “royal civil servant,” as he according to the demands of the State Church ought to do, then it is clear that such a man first becomes the King’s servant, then the Church’s servant, and then perhaps, if it can be managed, a very little bit of the congregation’s priest. It is another matter that there are here and there priests in the State Church who, independent of all human ordinances, occupy the place which the Word of God assigns to the priest; but they are few, and they are by no means genuine “state priests.”

He who will be a good “royal civil servant” in the priestly calling must first pass through a godless schooling and thereafter through a highly dubious student life, wherein all sharp edges, all independent conviction, all living apprehension of the truth are ground away; and then, when he has learned to bow and bend both himself and the truth, he is “finished” and can discharge his office by the King’s grace with just as much propriety as a bailiff or a magistrate. If he has become sufficiently polished, he can also confidently reckon that the ladder of advancement will be tolerably easy for him to climb, step by step, until he obtains a “quiet calling” in his old age.

But the whole arrangement is in its very essence a complete caricature of God’s order and manner for the congregation, and therefore fit only to bring priest and congregation into a false relation to one another. Even where the priest, as sometimes happens, is an upright man, and the congregation harbors some serious thought, even there this perverted relation—that the priest is the King’s civil servant instead of the congregation’s office-bearer—works in a high degree to the detriment of a united cooperation for the upbuilding of the Kingdom of God.

In the free Church the matter is naturally set right at once; for it corrects itself of necessity. Thus it has come to pass among the Norwegians in America, that since no one sent them pastors, the congregations themselves were compelled to choose them. In this respect it was a piece of good fortune that the Norwegian Church is a state church, and that it is the King who sends pastors to the congregations there. For had it, for example, been the case that an ecclesiastical authority in Norway had sent out pastors, then it would naturally also have sent pastors for the emigrants, and it might have taken a long time before matters had come into their proper order. For the Norwegian King there could be no access to send pastors to a foreign land. Nevertheless, it was not altogether avoided that both people and pastors in the beginning regarded themselves as a branch not only of the Lutheran Church, but also of the Norwegian State Church; and even to this day the Norwegian Synod is often simply called “the State Church.” It is also related that a congregation was organized somewhere in America with the definite understanding that it should belong directly under the Norwegian Church Department, which sufficiently shows that it did not at once dawn upon all that it was truly a Free Church which was in the process of forming.

Those times, however, are now almost past, and it has become a recognized matter that the congregation among us possesses the right of election, and possesses it alone. To be sure, it is used rather frequently, especially in the Norwegian Synod, that the right of calling is entrusted to individual persons or to the church council; yet it is willingly acknowledged that this is an emergency condition, which ought not to exist. For our part we can by no means find that such necessity is compelling, and we scarcely believe that in our entire fellowship there is any congregation which will do this; and we are assured that none ought to do it. For although it may occasion difficulties for a congregation to obtain a pastor when it must itself extend the call, yet we are also exceedingly apprehensive when we hear that someone wishes to take labor and responsibility from us; for we know that they always take freedom along with them. Therefore we oppose all who would take this “burden” from the congregation, because it cannot be taken away without the congregation at the same time suffering loss in its most precious rights.

A truly congregational election is thus the manner in which our ministers are called, and there has surely never been anyone who has attempted to deny that this is a rightful calling. But by this the question is far from settled whether the minister is lord or servant within the congregation. The circumstance that he, in full accordance with God’s Word, is chosen by the congregation, does not yet determine his future position. It has indeed been shown that the Free Church, with ministers chosen by the congregation, has been able to go to the greatest extremes on both sides. In particular, hierarchical tendencies readily unfold themselves within the Free Church.

There are two ways in which the clerical calling easily presses its way into God’s Church where it is left entirely to itself. The minister and the clergy may strive after power either from the basest motives or from the noblest motives without true insight into the nature of the congregation. There are, namely, ministers who seek power for base gain’s sake. Their number has unfortunately always been great. In the Free Church these have an excellent opportunity. They employ as their means the most pitiable form of politics. They establish associations in which clerical agreements secure them against the possibility that the congregation might obtain any other minister than the one upon whom the ministers themselves have agreed. When they are thus secure from that side, they then secure themselves against the congregation’s ability to depose them by inducing the congregation to accept such provisions concerning the removal of the minister as in fact render this impossible. Then, by personal means, they secure for themselves certain “fixed men” in every congregation, and then the demands begin gradually to rise. That in this way matters may be driven quite far is evident enough to all who know how things stand among large portions of our people. And when such a minister at the same time suppresses all lay activity within the congregation, seeks to dampen every awakening wherever it is perceived, and with a blind eye passes by dancing and drink and debauchery within the congregation, so that in the end the religion of the people consists in going to church and paying the minister, then it is clear that the people, while asleep, lose their Christian freedom, because they have no use for it.

But if we turn ourselves, with the utmost abhorrence, from this kind of seduction of office, which alas in the Free Church is like a corroding poison that all too quickly leads the people partly into the hands of sectarians, partly into brazen denial of God, there is yet another form of priestly domination which appears so good that we are almost tempted to wish for more of it than is found in America. For there are pastors who, penetrated by the consciousness of their great responsibility and their holy calling, labor with untiring zeal for the salvation of souls, but who forget the congregation’s own responsibility and the congregation’s calling to labor with the gifts which the Lord has given it by His Spirit. They proceed on the assumption that the congregation is no congregation at all unless it consists of nothing but, or almost nothing but, living children of God. And in order to know who these children of God are, they establish a mold of them according to their own experience, and then measure others by this measure. When it then appears that this uniform does not fit all, there is accordingly only a poor congregation left, and the pastor becomes ever more zealous to draw as near as possible, by his influence alone, to the goal of uniformity. This kind of hierarch also has a decided enmity toward the freedom of the congregation, toward lay activity, toward all activity in the congregation that is not directed by himself. They indeed labor for the Christian life according to their understanding of it; but since they hinder the true movement of life, they readily kill the life of the congregation by a suffocating uniformity. For this kind of endeavor the Free Church also affords a very wide field, and it invites this sort of zealous labor, because for a time it appears exceedingly fruitful and edifying. It is, however, precisely along this path that the Catholic Church has come to its perfect priestly dominion.

These two forms of priestly rule might seem altogether to repel one another and to be incapable of agreement. Experience, however, shows the contrary. The Free Church has at all times had both forms, and it is the lust for power that reconciles the sharpest oppositions. There are certain principles in which these two kinds of pastors agree exceedingly well. They can, namely, very easily agree concerning the concept of office. They are both agreed that it is directly from God; for the one sees in it a support for his temporal advantage, the other sees in it a help for his spiritual work. They are both agreed that the will of the pastor ought, as far as possible, to be law in the congregation; for thereby they could both most easily advance their plans. They are both agreed that all spiritual activity which they cannot control is exceedingly dangerous, each for his own reasons. And thus there are woven, by these two kinds of pastors, certain remarkable ropes and bands with which they would bind the congregation. But in truth it is only when they have cut off the congregation’s true relation to God that they both attain their goal. Like a Samson with his head shorn, the giant is then easily led into bondage.

The Free Church is thus in the very greatest danger of being tyrannized both by worldly priests and by one‑sided, though earnest, men. And the doctrine that the relation between pastor and congregation is as that of the husband to the wife is indeed in many respects appealing enough, but it lacks foundation in the Word of God and therefore leads to bondage.

But the Free Church also has another danger, which is expressed in the saying: the pastor is the servant of the congregation. The saying is indeed correct, if it is rightly understood—that is, if it is taken as an expression of the simple truth of God’s Word, that all the pastor’s work is a ministry of reconciliation, to draw souls to God; a service in the congregation and for the congregation; a service in His footsteps who came to serve all. But there is another understanding of this saying, according to which it simply means that the pastor shall do whatever the congregation commands, whether it be one thing or another; and in this sense the saying becomes purely and simply a denial that the pastor is the servant of the Lord and can do only what God has commanded, for the service of the congregation.

This understanding—that the pastor in every respect stands in the same relation to the congregation as a hired servant to the one who has hired him—necessarily produces hirelings who willingly bend truth and justice for a morsel of bread. And yet there may be more congregations of this sort, who gladly hire a hireling, than anyone could suppose. This is a dreadful danger for the Free Church; for where the congregation not only governs its own affairs, but also masters the Word of God and induces the pastor to assist it in carrying through things that are unchristian in their inmost root, there the whole congregation becomes without salt, and soon an utterly rotten tree, which must fall under the rapid development that free conditions bring with them. When the wolf comes, the hireling flees, and the sheep are scattered and torn. It is scarcely possible to deny that there are in the Free Church several things which may soon entice congregations into such a view of pastor and office. Once a worldly spirit and tone gain dominion in a congregation, and the pastor does not all the more firmly maintain his office as a service of God, then now one consideration and now another may soon make him a slave instead of a servant, and congregation and pastor alike are soon equally far from the Lord’s right way—to their own destruction and to the offense of the congregation of God.

There is nothing so heart-rending as these distorted caricatures of the order of God’s congregation within the Free Church, for there is nothing so lovely and delightful, both before God and before all the people, as a truly Christian congregation, where the fellowship of brethren flourishes, where the Word of God has its proper place, where the Spirit of the Lord governs both shepherd and flock, so that together they go out and come in and find pasture. Would that there were spirit and grace to paint such a picture, that it might melt all cold hearts, both among pastors and congregations, so that they might no longer stand against the Lord, who so willingly desires to form a true congregation among us! But if the picture is not to stand on paper alone, but become a living reality around us and within us, then pastors must cease standing in the way of the congregation’s spiritual freedom, and congregations must cease standing in the way of the proper administration of the Office. Alas, we must lament both things among our beloved people.

The Office is God’s service. It is God himself who sends the true pastors in the ministry of reconciliation. He sends them with the gifts of the Spirit; he makes them fit to bear witness to the way of life and death, to cry aloud with full voice concerning sin, and to lift up and call sinners to the Savior. The Word of God is not to come to men only in a printed book, but according to the Lord’s ordinance it is also to be borne forth by witnesses who themselves have experienced that the doctrine is of God. God himself lays the work upon the hearts and shoulders of his servants; and by his power even the least of his witnesses shall speak. To him the pastor is accountable, and in him also is the pastor’s strength. And the man and the congregation who do not first and foremost acknowledge this—that the Office is the proclamation of God’s Word, and that God himself is the one who will demand an account of how the pastor carries out his work—that man is no pastor, and that congregation no Christian congregation.

Therefore this is the question which is the most important of all for every pastor and every congregation: Is that which is spoken and preached the Word of God in its crushing severity and life-giving power? If this is so, then bow yourself, you sinner, and be raised up, you who are broken, for it is the Lord who speaks to you through the mouth of his servant! But if it is not so, then let it become so; the grace of the Lord stands ready, and he is willing to send it to you again, if only you will open your heart to him.

The ministry is the congregation’s service in this Word of God.
Here is the other side of the matter. The congregation has the right to demand
that the pastor truly, in spirit and in truth, shall “minister” to it with the
Word of the Lord. The body of Christ is to grow by the same Word through which
the life of God has been awakened. Not for his own advantage or his own honor
is the pastor to preach, but for the edification of the congregation. As
independent of all fear of men and favor of men as the pastor must be in the
administration of the office, so humble and willing to serve must he be when it
comes to preaching the Word in season and out of season. Always willing to
admonish the rebellious, to instruct the erring, to comfort the afflicted, and
to bear with the weak, he must not shrink back from becoming the least of all.

It is with the pastoral office in the same way as with a Christian’s position
in the world: there is a height and power of the Spirit in it, because it is the
Spirit of God who sends him to cry out in Christ’s stead: “Be ye reconciled to
God!” But there is also the humility and poverty of the Spirit, because there is
unceasingly to be borne the contradiction of sinners, the sorrow of the erring,
and the humiliation of the proud. Yet through it all it is love that gives
strength and wisdom, endurance and patience for the work of ministry.

It is therefore necessary to hold fast with unwavering faithfulness, if pastor
and congregation are to stand in a right relationship to one another, that the
office is God’s service in the Word and the congregation’s service in the same
Word. If the one is misunderstood or the other forgotten, the relationship
becomes perverted. Let the pastor stand manfully and undaunted upon God’s truth,
and let the congregation bow beneath the Word; and let the pastor always be bent
to serve all, and the congregation rightly eager to demand his service with the
Word which the Lord has entrusted to it—then all stands well in the house of
God, which is His congregation. Then shall the Lord’s faithful promise, that
His Word shall be food for the hungry, a hammer to shatter hearts, a balm to heal
the wounded, prove itself as living truth in the congregation; and there shall
be a fellowship of love which with irresistible power shall draw more and more
into itself and unto God.

If this, then, is the proper relationship of the office to the congregation—that
it is the proclamation of the Word which the congregation by its call entrusts
to him whom God has sent and sends, and that it is both God’s service and the
congregation’s service in the Word—then the question arises: Is the congregation
now bound to be edified only by the pastor, or is it its duty and right also to
edify itself? In other words: Is lay activity abolished by the congregation’s
calling of a pastor, or is it confirmed? We shall in what follows speak somewhat
more closely of this matter, which is so exceedingly important in these days.

