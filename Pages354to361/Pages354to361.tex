
\begin{center}
\includegraphics[width=0.9\textwidth]{OpenImage.png}
\end{center}

\section{The Conditions for the Formation of Congregations}

Source: Report of the Lutheran Free Church’s 9th Annual Meeting, 1905,
pp. 86–95. — Ed.

This section appears on pages 354–361 of the original volume. — Present Ed.

\bigskip

The intention in proposing this theme is doubtless to obtain an answer to the question: When is it right, timely, and justifiable to undertake the organization of a congregation in a given place, and which persons ought one to seek to include in the organization?

To a question of this kind, it is self‑evident that no answer can be given which lays down a rule applicable to all places and at all times, unless it be given in the form of a general principle, which must then be applied as appropriately as possible in the individual cases. We cannot say that a missionary pastor must always labor for a year with preaching before he seeks to have a congregation organized; neither can we say that he must first of all have a congregation established before he has the right to preach the Word of God to the people.

The question is of such a nature that only general or principial answers can be given.

And if it is this which is to be given here, then it is essential to hold fast that the Free Church, from the very beginning, has maintained that the question concerning the congregation and its nature must be clarified out of the divine revelation itself in the New Testament, because it is what one in the Lutheran Church calls a doctrinal question, in which the Word of God is the sole rule and guiding norm.

And the proper organization of the congregation stands in such a close and intimate connection with the congregation’s essence, that one must necessarily return to the question of the congregation itself, if one would gain a handle on how the congregation and congregations are formed.

Now, since the Congregation provides the New Testament with clear and abundant instruction—for indeed the whole revelation that is given in the New Testament after Christ’s Ascension speaks of the Congregation and of its nature and its work. The Acts of the Apostles, the Epistles, and the Revelation of John speak continually of the Congregation, or, which is the same thing, of the work of the Holy Spirit; so that those who in our days are opponents of the Congregation and will hear nothing of the necessity of returning to Scripture when the question concerns the Congregation, cannot at any rate appeal to the claim that there is so little in the Bible about the Congregation that one cannot there receive sufficient light upon the matter.

Yet, as the situation actually stands among us, it is in fact distorted in so many and manifold ways that it is altogether reasonable that there should be a constant and continued discussion of the subject, in order, if possible, to gain more light concerning it. And since all our labor is dependent upon a right apprehension of the Congregation, there is nothing more important than to pursue this investigation with all strength and all vigor, so that the clearest results may be attained as soon as possible.

We are Lutherans, brethren, and this means first and foremost that we stand unshakably upon the ground of Scripture; therefore it is altogether right and genuinely Lutheran to seek the true instruction concerning the Congregation in the Holy Scripture itself.

And indeed Luther himself has shown the way also in this matter. There is no man for whom it has succeeded better to gather his thoughts into a brief chief summary than Luther. His Small Catechism is in this respect the greatest masterpiece that exists within the Christian Church. As with a sunbeam it is written there, what stands in the Catechism, that it is “the Holy Spirit who calls, gathers, enlightens, sanctifies the whole Christian Congregation on earth, and preserves it with Jesus Christ in the one true faith.”

This is plain speech, and it is scriptural in the highest sense. It agrees not merely with the letter of Scripture, but in truth with its Spirit; and it is not attested by a single passage here or there, but the whole Scripture bears witness to this, that it is precisely the goal of all God’s work, leading, governing, and revelation for and with fallen mankind to raise up His kingdom upon earth, or, as it is said in Scripture, “to purchase for Himself a Congregation with His own blood,” and that this Congregation is that which the Spirit calls and gathers, enlightens and sanctifies from His day of Pentecost onward, when He came with the power of the storm and the fire of love upon the disciples of Jesus and bound them together into the body of Jesus, “which is the Congregation.”

It is plainly evident from the testimony concerning the Church’s beginning that it is the work of the Spirit in Jesus’ believing disciples. And where the Spirit, through the Word, is permitted to carry out His work, there the Church is formed and founded and constituted “that very day,” so that there is no long waiting. There arise Baptism and the Supper and brotherly fellowship and common prayer as the marks that set forth the Church as visible before all eyes, so that it may be perceived and sensed both as a distinct society and as a living and drawing power, which continually gathers more of those who desire to belong.

Who then became members of the Church, and who decided whether they should be included or not?

Scripture says: “They that gladly received Peter’s word were baptized,” and thus they were added to the Church. But Scripture also says: “And the Lord added to the Church daily such as should be saved.”

Behold here two testimonies concerning the same matter; these we must hold together, and the more we unite them in a spiritual manner, the more clearly will they also agree in that which Luther says: It is the work of the Holy Spirit to call, gather, enlighten, and sanctify the Church; for how else could this be fitting, that those who received the Word were added, and that the Lord added them, except precisely in this way—that the Lord worked through the Word, and the Spirit drew them by the calling Word and the regenerating Baptism into the Church, “which is the Body of Christ.”

This is the natural, and therefore the only right, way of forming the Church. And as it began in this manner, so it also continued in the apostolic age. When “Samaria received the Word of God” through the preaching of the scattered Christians, there too the Church came into being. Likewise in Galilee, yes, even as far as Antioch in Syria, where Christians from among the Jews and Christians from among the Gentiles were for the first time united in one Church and in one Body of Christ, in that “the middle wall” was broken down; and where therefore the new society received its own new name, in that the disciples there and then were called “Christians.”

And from this Antioch there also went forth the missionaries to the Gentiles; for from this time onward the occasion for the formation of congregations became as wide and spacious as the whole world; and indeed congregations were formed through Paul’s preaching, in an astonishingly short time, in a multitude of cities both in Asia and in Europe.

Always in the same manner: the preaching of the Gospel itself created the necessary presuppositions; for the congregations were formed of those who “became believers,” who “received the Holy Spirit,” who “were baptized into Christ’s death.” Always through the Spirit and through faith, which, when viewed from the outward side, manifests itself as complete and full willingness.

\textbf{Willingness}

One might indeed gladly call it self-determination, provided that it is understood that thereby no opposition is intended to the Christian truth that this self-determination is, in its innermost ground, a divine determination; for those who “gladly received the word” are those whom the Lord has called and chosen and given the obedience of faith, so that their willingness is not of themselves, but of the Holy Spirit.

Here, then, the apostolic principle for the formation of congregations is clearly expressed: it is willingness—the willingness of the Spirit and of faith. It presents itself exactly as it is written in the Book of Revelation: “And the Spirit and the Bride say, Come. And let him that heareth say, Come. And let him that is athirst come. And whosoever will, let him take the water of life freely.”

As already said, it is impossible to express the apostolic rule for the formation of congregations better than in Luther’s words: “The Holy Spirit calls, gathers, enlightens, and sanctifies the Christian congregation.”

With this willingness there is thus laid upon those who come to the congregation—called and drawn by the Spirit—the whole responsibility for the Kingdom of God, that is, the responsibility for the life and flourishing of the congregation, for its labor and its honor. For they are those who entered into the congregation not because other people judged them to be Christians, but because they judged themselves, and because they confessed the faith in the Redeamer\footnote{The original text reads den forstaaede ("the understood"), likely a typographical or OCR error for Forsoneren ("the Redeemer"). The latter is more consistent with the surrounding context of confessing faith in Jesus Christ as Savior. — Present Ed.}, who was preached to them. To join oneself to the congregation and to be baptized was the same as to confess Jesus Christ as one’s Savior from sin, death, and the kingdom of Satan; and with this it followed of itself that he who knew the Savior in Christ was also willing to “pay unto the Lord his vows” and to be a witness of Him “who hath called us out of darkness into His marvelous light.”

It lies in the very nature of the matter that, as the congregations grew older and increased in number and esteem, and as many different reasons arose which might move a person to attach himself to the congregation, the latter was bound to seek to guard itself against the danger that lies in the reception of such members as are not drawn to the congregation by the Spirit of God, but by fleshly motives. All that lives and grows bears within itself the impulse of self-preservation, and seeks to keep at a distance and to remove what is called “foreign matter.” It is therefore also self-evident that the highest and most indispensable of all organisms, the Body of Christ or the Church, was careful both with regard to admission into and exclusion from the congregation. And partly through its officers, and partly through the meetings of the congregation, the Church exercised that oversight which it recognized as its holy duty toward the Lord and His Body. Yet this did not conflict with the principle of freedom; it merely imposed a necessary and justified testing of the nature of that freedom, whether it was the work of the Spirit or of the flesh.

It is these spiritual truths concerning the formation of the congregation which the Free Church seeks to revive in the footsteps of Luther. And there is great need of this; for although Luther knew full well and continually affirmed that the Church is the work of the Holy Spirit and therefore a matter of freedom, yet we all know that the Lutheran Reformation was robbed of so very much of its spiritual character by being fastened in the arms of princes and politicians. And quite particularly there was left no room for freedom or for the work of the Spirit in the formation of the congregation. The Church became merely a part of the state administration, and the congregation only one aspect of national life. And as coercion always rules in the state, it naturally came to rule also in the state church.

The consequence of this was a grievous confusion of concepts, the sum of which, in the sphere of the question of the congregation, may be described thus: that Christianity came to be regarded as a “national religion,”\footnote{The word Folkefit in the original text is an evident OCR or printing error. Based on the comparison to "pagan religions" and the critique of the state church, the intended word was likely Folkereligion ("folk religion") or Folkeskik ("national custom"). The translation "national religion" reflects the author’s argument that Christianity had been reduced from a spiritual choice to a mere cultural or ethnic inheritance. — Present Ed.} like the pagan religions among the pagan peoples; and bonds harder than those of such as religion are scarcely to be found, especially among a people such as our own.

Now when these children of the State Church feel Christianity upon them as compulsion and bonds, this brings with it, for many, a bitter hatred of Christianity. It appears to them to lie obstructively in the way, partly of the satisfaction of their lusts, partly of the free development of thought. Especially when, through a spiritual awakening, the demands of Christianity press heavily upon the consciences, there is awakened also this human defiance and craving for freedom, which says: “Let us break their bands asunder, and cast away their cords from us.”

And not a few there are who, upon the crossing to America, tear asunder both the bonds of the National Church and the rope of the imposed Christianity, and they walk in the destructive opposite of the state‑church system: fleshly freedom.

But then, on the other hand, there is a very great multitude of people who here in America cling all the more zealously to the state‑churchly custom, as they feel that there is danger in the severing from the law of the State Church. And it can hardly be denied that many Norwegian congregations in America are formed upon the foundation of the Norwegian National Church, with some admixture of American township politics.

But amid all confusion and mismanagement there still stands the Lord’s firm foundation, and it bears the same old inscription: The Lord knoweth them that are his; and, Let every one that nameth the name of Christ depart from iniquity.

Therefore it is not hopeless to labor for the restoration of the congregation and, little by little, to set forth a free‑church practice which returns to Luther and the New Testament, and brings clarity into the confused notions of the nature and organization of the Church.

The task is to come away from compulsion and politics in the formation of congregations and onward to true, spiritual voluntariness. But when we loosen ourselves from the coercion of the State Church and flee from its distorted image of the Church, it is necessary to be on our guard lest we also depart from Lutheranism and from the truth.

We must unreservedly go back to the principle of voluntariness, which is this, that “the Holy Ghost calls, gathers, enlightens, and sanctifies the congregation,” and at the same time be watchful that we do not put ourselves, or one or several persons, in the place of the Holy Ghost, so that we imagine we can call and gather, build up and form a congregation, provided only that everything proceeds according to our own head.

Therefore let the invitation have its voice; let the Gospel ring freely and fully, calling men to come in through this one true Door, that the house of the Lord may be filled.

This is the responsibility of the missionary pastor or the evangelist; this is his calling and his work. Lay then the responsibility upon those who hear the call and the invitation. They are accountable before God both to follow the invitation and to follow it rightly, that is, in the same spirit in which it is issued. Let it be said to them that they must come, and that they must put on the wedding garment when they come; let it also be said to them that in this matter they bear responsibility before the Lord and not before men. Let it be borne witness loudly and clearly that he who is part of God’s congregation thereby lays claim to being a true Christian, a child of God, and that both the congregation and the world have the right to expect of him that he conduct himself as a Christian, to the honor and praise of God’s name among men.

It avails nothing to suppose that we can set our judgment concerning a man in the place of the Lord’s judgment. He himself has the keys and opens and shuts in the one perfect manner; but upon us rests the responsibility for the unwearied labor of preaching the Word and bearing witness to the truth in season and out of season, and thus in time it will dawn upon the consciousness that the congregation and its outward ordering are a holy matter, that it is a temple of God which no one may dare to corrupt.

Under such fear of God and under the sense of this mighty responsibility must all organization of the congregation be undertaken; and it must be clearly testified that free as the congregation is, it is also obedient to God’s will and Word. And whoever feels within himself that this becomes an unbearable constraint, let him hasten to seek the Lord while he may be found, that from the Lord himself he may receive the freedom of the Spirit, which consists in that love which is poured out in our hearts through the forgiveness of sins and the blessed deliverance.

If then the question is asked: Shall we wait long before we organize a congregation, or shall we wait until we are certain who are believers, so that we may take them into the congregation and leave the others outside?—then the answer has already been given: Follow the Lord’s Word, and do not delay forming a congregation of those who are baptized into Christ and confess faith in him, whether they be many or few, and wait upon the Lord, who shall give the increase as it is pleasing to him.

It is not the number that matters; it is the Spirit. It is not our judgment that matters; it is the Lord’s. And he who comes bears himself the responsibility for his sincerity, when the invitation has sounded to him in undiminished fullness: Come to rest, come to labor; come to the Supper, come to the wedding; come to the vineyard, come to the harvest work; come to the Cross and come to the Crown!

