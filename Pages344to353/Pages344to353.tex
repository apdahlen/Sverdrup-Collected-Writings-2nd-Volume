
\begin{center}
\includegraphics[width=0.9\textwidth]{OpenImage.png}
\end{center}

\section{Sixth Section}


As a conclusion to this volume, there has in this section been included a series of lectures which all— with the exception of the last—were delivered at the annual meetings of the Augsburg Confessors and the Lutheran Free Church. The final lecture—The Unity of Believers—was written for a church gathering in Hatton, North Dakota, in the winter of 1907. As it was not possible for the author to be personally present, the lecture was read aloud by one of the participants in the meeting. It was later published in “Luthersk Tidskrift.” —Ed.


\subsection{Congregation and Congregational Life}

Source: Report of the Fourth Meeting of the Augsburg Confessors, 1896, pp. 59–70. —Ed.

This section appears on pages 344–353 of the original volume. — Present Ed.

\bigskip


\textbf{The Congregation}

\begin{quote}
“Who hath heard such a thing? Who hath seen such things? Shall a land be born in one day, or shall a nation be brought forth at once?”
Isa. 66:8.
\end{quote}

The wondrous work which Isaiah thus proclaims is the birth of the Church through the outpouring of the Spirit on the feast of Pentecost. The Church did not come into being through any human exertion of strength or clever calculation; it came as a gift of grace from God, a glorious fruit of the death and resurrection of Jesus Christ; for the ascended Savior fulfilled the promise which He gave the disciples before His death, concerning the Comforter whom He would send, and the Spirit of truth whom He would give them; and when the Spirit came, and when the storm rushed with mighty power, and when tongues as of fire rested upon each one of the disciples, and when the living testimony of the Spirit concerning the great works of God resounded in the many tongues—then the Church stood there, “a whole nation born at once.”

But the Congregation is not without its preparation. This new “dispensation in the fulness of times,” this “mystery of Christ,” was hidden in God from the foundation of the world, and was the object of His eternal counsel, His “good pleasure which He purposed in Himself.”

Therefore it was also proclaimed and promised by the prophets throughout the whole Old Testament age; “for the Lord GOD doeth nothing, but He revealeth His secret unto His servants the prophets” (Amos 3:7). From Moses to Malachi the voice of promise sounds forth through all the prophets, concerning the day that is to come, when the dwelling of God shall be with men, and He will dwell with them, and they shall be His people, and God Himself shall be with them and be their God.

It is of this day that David sings in his last words: “And he shall be as the light of the morning, when the sun riseth, even a morning without clouds; as the tender grass springing out of the earth by clear shining after rain” (2 Sam. 23:4). And again: “Thy people shall be willing in the day of Thy power; in the beauties of holiness from the womb of the morning: Thou hast the dew of Thy youth” (Psalm 110:3).

Of the congregation of God the prophet Micah speaks, when he says in the fifth chapter: “And the remnant of Jacob shall be in the midst of many people as a dew from the LORD, as the showers upon the grass, that tarrieth not for man, nor waiteth for the sons of men. And the remnant of Jacob shall be among the Gentiles in the midst of many people as a lion among the beasts of the forest, as a young lion among the flocks of sheep: who, if he go through, both treadeth down, and teareth in pieces, and none can deliver.”

— Of the congregation Zephaniah prophesies, when he says in the third chapter: “I will also leave in the midst of thee an afflicted and poor people, and they shall trust in the name of the LORD.” And to this afflicted and poor people it is said: “The LORD is in the midst of thee; thou shalt not see evil any more. The LORD thy God in the midst of thee is mighty; He will save; He will rejoice over thee with joy; He will rest in His love; He will joy over thee with singing.”

Time would fail us, were we to recount all the good words with which the Lord has proclaimed the congregation in the Old Testament; only this word of Zechariah shall we set down here:

“Awake, O sword, against My shepherd, and against the man that is My fellow, saith the LORD of hosts: smite the shepherd, and the sheep shall be scattered: and I will turn Mine hand upon the little ones. And it shall come to pass, that in all the land, saith the LORD, two parts therein shall be cut off and die; but the third shall be left therein. And I will bring the third part through the fire, and will refine them as silver is refined, and will try them as gold is tried: they shall call on My name, and I will hear them: I will say, It is My people: and they shall say, The LORD is my God” (Zech. 13:7–9).

If we then would consider how precious, chosen, and beloved the Church of the Lord is unto Him, it is set before us that the eternal counsel concerning the New Testament dispensation, which is the Church, has been made manifest through the revelation and working of the love of the triune God throughout all generations.

With the purpose of bringing forth the Church in the fulness of the times, God gave the promises, called Abraham, chose Israel, and led His people in wondrous ways; kingdoms arose and kingdoms fell, idol temples were raised and again sank into ruins, the nations exerted their strength in vain attempts to build an everlasting kingdom and to gather imperishable wisdom and power; yet the Lord had His Kingdom and His Church in view in all this, and when the power of the world had been proven to be impotence, and its wisdom folly, and its joy nothing but transience, then the hour of the Lord had come to offer the exhausted and poor world the riches of grace through the Son and the Spirit.

And if even this is great and precious to behold, how the whole governance of the Father over the world points toward the Church and its coming into being and growth, infinitely greater still does the worth of the Church become when we consider that it has been purchased with the blood of the Son. For so God loved the world, that He gave His only-begotten Son over unto death for its salvation, and so has the Son loved, that He came, through a life of suffering and through His death upon the cross, to purchase for Himself the Church with His own blood. The birth, death, and resurrection of the Son have this one purpose, that the Kingdom of God, which is the Church, should come, unto the honor and praise of God, and unto the everlasting blessedness of human souls. For He Himself, the only-begotten Son of God, would be the grain of wheat which falls into the earth and dies, in order that it might spring up and bear much fruit—a fruit which shows itself across the earth in the Church, which is the field of God, whose glorious harvest is the souls that are saved by the love of God and by the grace in Christ Jesus.

For that the Church might come into being, Jesus went unto the Father, that from him he might send the Holy Ghost. The Spirit came, and the Church was born. The word spoken to Nicodemus received its fairest fulfilment: “The wind bloweth where it listeth, and thou hearest the sound thereof; but canst not tell whence it cometh, and whither it goeth: so is every one that is born of the Spirit.”

It is the holy work of the triune God that stands before us when we contemplate the wondrous coming-into-being of the Church. Sin had disturbed God’s work of creation; Death had become lord over the fallen creature; God’s wrath over sin laid loveless suffering upon the human being who was created for blessedness. God’s eternal counsel concerning the dispensation of the fulness of times was a counsel of love and of grace; for it was concerned with restoring all the damage which sin had wrought; it was concerned with abolishing the power of sin and of death and of condemnation over humankind; it was concerned with giving back to death’s bound slaves life and incorruption. Therefore no Church could be born, no fellowship of God’s liberated children be created, without atonement through Christ’s death, life from the dead through Christ’s resurrection, union with God and His Son through the Holy Ghost. God’s free and living people upon earth have as their nearest and indispensable presupposition the death and resurrection of the Son of God, and union with him through the Holy Ghost.

It is for this reason that the Church is Christ’s body; the ascended Saviour dwelleth, through the living faith wrought by the Spirit, in his believers. He was dead, and behold, he is alive for evermore. Thus must it also be said of every one who in spirit and in truth would belong to his body: This one was dead, and behold, he liveth. That life of Christ which pulses through the Church by the Holy Ghost is the life of the risen One; it is life after death, the eternal life which dieth no more. None is partaker thereof save he who himself is dead and raised up with him. The living faith is a death from the world upon Christ’s cross, and a being made alive unto God through Christ’s resurrection.

This is Christ’s mystery and the Church’s holy riddle: dead with Him and raised with Him, God’s Church is not of the world and yet is in the world. And because the Church is thus fellowship with the risen and ascended Savior, therefore this fellowship is independent of earthly circumstances and conditions. The Church is not an earthly people, such that it depends upon having the right human forefather, or the right color, or the right earthly fatherland, or the right language. Nor is it a particular caste or class or age or school of cultivation. For the Church is not a higher form of worldly life, nor a more or less developed stage of civilization; but it is a new life of God among men, a life which can be found only where a dying away from the world has taken place. Human conditions come into consideration only insofar as it is difficult for those who are something and have something in the world to die away from the worldly and receive the life of God. Therefore not many wise, not many rich, not many noble have been brought into the Church; for it cost them too much to lose all in order to win the Kingdom of God. Therefore also within the Church there is neither rank nor caste, higher or lower strata. There is the essential equality that follows from having the same Father, while there is the necessary difference that follows from one being young and another old, one having one gift for labor, another a different gift. There is the essential unity that belongs to the body of Jesus Christ by virtue of the one Spirit. There is also the difference and manifoldness that arise from this, that the one Spirit distributes His gifts and gives to each individual believer the gifts as He wills.

But if the Church is thus life out of death, or the life of the risen Lord Jesus Christ, then it follows that the Church can die no more. Its true and essential life is the life of Jesus Christ, which has already passed through death and therefore dieth no more. Not only does the Church possess vitality enough to exist so long as the earth stands and the times endure; but through all times and all generations and all eternities God is revealed and praised and honored through the Church, which is His Son’s, our Lord Jesus Christ’s, living body, which indeed can suffer, be tormented, and be assailed by the world, but which, despite all mistreatment, nevertheless lives the risen One’s eternal life in the midst of the world of transitoriness.

Therefore, since the Congregation is the goal of the Father’s providence, of the Son’s birth, death, and resurrection, of the outpouring of the Spirit; since the Congregation is the chosen race, the royal priesthood, the holy nation, the people of God’s own possession, the dwelling of God in the Spirit, the city set upon a hill and the light upon the lampstand—so it is not strange that Paul, who was cast into prison and laid in chains for the sake of this Congregation, writes in the consciousness of this great responsibility to those who were a congregation and called themselves a congregation: “I therefore, the prisoner of the Lord, beseech you that ye walk worthy of the calling wherewith ye are called.”

What we here mean by congregational life is nothing other than that manner of walk which corresponds to the high calling of the Congregation. For this calling rests first and foremost upon each individual congregation in every place where Christians gather around the Word of God and Baptism and the Lord’s Supper. If an assembly of believing men and women has taken upon itself the holy name of congregation and begun its holy work with Word and Sacrament in a place, then it has thereby also taken upon itself the calling and the responsibility; and God will judge them accordingly, as Paul writes to the Corinthians: “Know ye not that ye are the temple of God, and that the Spirit of God dwelleth in you? If any man destroy the temple of God, him shall God destroy; for the temple of God is holy, which temple ye are.”

There are so many who seek to escape the great and dreadful responsibility which rests upon the Congregation, by thinking of the whole Christian Church instead of the small individual congregation; and by fixing their attention upon this great and far‑reaching fellowship, the single congregation becomes so insignificant that it seems as though its life and walk and mutual dealings could not be of such great importance, and thus it might more easily and cheaply escape from its calling and responsibility. But nothing can be more corrupting than such a way of thinking. Just as little as the individual soul can be saved in the crowd and mass of Christians without being Christian itself, just so little can the individual congregation escape its calling and its responsibility in the mass of congregations. Where Christ is, there He is either whole, or He is not there at all. Where Christianity is, there it is whole, or else it is not there at all. And each individual congregation by itself has precisely the same calling and task and responsibility as the whole Christian Church; and so heavy is the responsibility, as though it were the only congregation and the entire gathering of congregations.

Let us, in our consideration of the Congregation, try to hold fast to this, so that no one may have occasion to evade the calling with the false consolation that there are so many involved that it is not so exacting with each one.

How then shall the Church walk worthy of its calling?

The high calling to be the Body of the risen Saviour and the dwelling-place of God’s Spirit demands the deepest humility; for only in the consciousness of our own unworthiness and incapacity is there room for God’s Spirit and for God’s glory.

The all-embracing calling to gather all souls at the cross of Jesus Christ, that they may be cleansed by the blood and reconciled with the Father, demands the greatest love; for only in Christ is salvation, and therefore it is only the love of Christ that can constrain us to seek the salvation of souls.

The one great goal—to gather all into the fellowship of Christ and to present every human being complete in Christ Jesus—demands one Spirit and a diversity of gifts, so that, despite the many ministries and powers, there may yet be no confusion or distraction, but that the many small abilities and gifts may, through the uniting and ordering power of the Holy Spirit, work toward the same goal.

If, then, we attempt to orient ourselves somewhat more closely within this activity of the Church, in order to answer to its name and its calling, it may perhaps be most convenient to divide it into two branches, which nevertheless stand in the most intimate connection with one another:

* The inward growth of the Body of Christ unto its edification in love.
* The outward growth of the Body of Christ, to embrace all human beings, the whole fallen race.

For there is room for growth or advance in both directions. In its earthly existence before the return of Christ, the Church is never perfect or complete or finished. It is perfected precisely at the return of Christ; therefore it is not perfect at any earlier point in time. Until that time it is continually in progress and development.

In the inner life of the Church there is always this deficiency in the earthly Church: that not in all who are drawn into the Church and attach themselves to it is there a true transition from death to life; many dead and sleeping persons follow along into the outward Church, or grow up in its midst. This is an unavoidable calamity, for which the Redeemer Himself has prepared us through His revelation. It therefore constitutes no excuse, either for those who enter the Church with a worldly mind, nor for those who even dare to employ fleshly and worldly means of persuasion in order to move anyone to join the Church.

There is in no one an absolute and complete severance from the world and the flesh. Therefore there is also in no one a full and perfect participation in the life of the Risen One. With believers in all ages it fares as it did with Paul, who says: “So then with the mind I myself serve the law of God; but with the flesh the law of sin.”

Therefore at all times there is room for growth in the Church. Ever onward is the watchword of Christians and of the Christian Church, until the coming of Christ.

And in outward regard there is always, both at home within Christendom and beyond its borders, a great multitude of souls who may be gathered into the Church through the Gospel. Along the highways and by the hedges there are many lost and wretched people, many prodigal sons and daughters, who ought to be compelled to come in. And think, think of the millions of heathen who have not heard and do not hear the blessed Gospel of Christ! There is room for growth, room for advance, room for struggle, and room for victory.

But if a body is to grow, then two things in particular are required, namely nourishment and organs.

God has provided for both. The Church has received Word and Sacrament. Through these it receives nourishment and vital strength; power of resistance against the world and sin and death, so that in spite of all the contagion of corruption it may nevertheless keep itself living and sound; strength for labor, so that it need not sink powerless beneath the toil and burden of the mighty work.

The Church is also equipped with many members, all of which are organs through which the Church receives Spirit and carries out its work. The misunderstanding that the Church has only a single organ, namely the pastor, stands in opposition both to the letter of Scripture and to its spirit. All Christians, men and women, adults and children, are called and appointed and gifted to be workers in the vineyard. There is rest and peace in God’s Church; there is also busyness and life and activity. There is nothing but movement and freedom in the Church. Many are fearful of such freedom in the Church; but this is because they do not know the working and power of the Holy Spirit; for God’s Spirit is mighty, through God’s Word, to give to each member its own work.

Therefore, if the proclamation of the Word becomes truly powerful and Spirit-filled, scriptural and full-orbed, and if hearts are kindled and the members of the congregation become zealous in word and in deed, then the congregation shall grow with the growth of the body, both inwardly and outwardly. Knowledge and understanding shall increase, and love shall become more inward and more comprehensive. And the congregation shall gain greater adherence and draw more to itself, when all its members live for this one thing alone: the salvation of souls and the spreading of the kingdom of God.

The congregation’s inward growth will therefore show itself in a stronger separation and distinction from the world, in greater renunciation and self-denial, and in a more self-sacrificing love for souls, so that on the one hand the light stands out sharply from the darkness, and on the other hand the light draws and lifts those who still walk in the coldness and darkness of the world. Yet this is a work that is first of all exercised within the congregation. There the baptized children are to be reared, instructed, and enlightened, that they may remain upon the narrow way of renunciation and faith; there the seeking are to be borne; there the believing are to be strengthened and guided, encouraged and admonished. And then outward, in the heathen world, there is work enough to be done in order that the Gospel may be proclaimed. And the congregation that does not engage in mission is no congregation except in name; for the Spirit of God is a Spirit who drives to labor for the spreading of the kingdom of God unto the ends of the earth.

When the Christian congregation thus becomes conscious of its calling and begins to live therein, it shall ever more experience that if it lives in love, it lives under hatred. Not only shall the living congregation be repaid with the hatred of the world, because it is an offense to the world to labor for its salvation. But the living congregation will, if possible, be hated even more by the false and dead church, which in the awakened life of the congregation will see its own worldliness judged. Furthermore, all those who have worldly advantage from the congregation’s death and ignorance and inactivity will become bitter enemies of the free and living congregation. But those whose delight it is that souls be saved and life be brought forth, they will with joy greet the free and living congregation and rejoice to be allowed, if it be given them, to go before it in its struggle, labor, and tribulations. That is the place of the priests in the free congregation.

There are those who fear that, when the congregations awaken, the space will become too constrained for the pastors. This fear is unfounded, provided the pastors understand their task and their calling. When awakening comes to a congregation, the faithful shepherd and caretaker of souls will find both labor and sympathy, both afflictions and joys. But the worldly-minded pastor will indeed encounter many difficulties where the work of God’s Spirit begins in the congregation, and perhaps the room will become too narrow for him.

The Church lives in hope; for the consummation cannot fail to come; the Life of Christ cannot slumber without attaining unto its eternal blossoming in glory. It is the water lily which, from the turbid depths, shoots upward toward the clear sunshine beneath God’s heaven. That Life which springs forth from Christ’s dark grave in His resurrection from the dead; that Life of God and of His Spirit’s eternal love, which is lived amid hatred and persecution, tribulation and contempt—this Life at last reaches unto eternal glory.

Then the Church is consummated, when all her faithful and true members are gathered before the throne of the Lamb, and from the depths of their hearts the song of praise surges forth, mighty and full-toned as the sound of many waters. Then shall the New Jerusalem, the holy city, descend out of heaven from God, prepared as a bride adorned for her bridegroom; and then shall the tabernacle of God, in eternal perfection, be with men, and He shall dwell with them, and they shall be His people, and God Himself shall be with them and be their God. Amen.

