%FIXME: Recast without the KJV resister.


\subsection{The Fellowship of the Congregations}

Of all the tasks which the free Church in America has to resolve, there is perhaps none so difficult as that of finding the right form for the mutual connection of the congregations. It is something concerning which all are agreed, that the individual congregations must unite with one another in fellowship. This rests in part upon the fact that they have one Lord, one faith, one baptism, one God, one Spirit; and this inward bond impels them also to form outward fellowships. In part it arises from this as well, that there are certain ecclesiastical affairs which can scarcely—indeed, we may readily say, cannot—be advanced without the united strength of several congregations. But on the other hand, throughout the whole of church history it has shown itself that the “Church”—by which we then understand the outward society of many congregations—has almost constantly been such an enemy of the congregation and of its freedom, that this must counsel us to the utmost caution in our manner of proceeding in this matter, lest we once again fall under the old yoke of bondage. To this must be added that in this respect we are far from having the same clear and definite guidance in the Word of God as when the question concerns the nature and right of the individual congregation.

We have already pointed out that the word “Church” is unknown in the New Testament. It would be incorrect if anyone were to draw from this the conclusion that there was therefore no Church in the days of the apostles, and that for this reason we ought to have no Church either. The New Testament uses, as we have already shown, the word “congregation” both of the individual Christian local congregation, which was formed everywhere the Word of God was preached, and of the whole great multitude of Christians, regardless of where they might be found. In such passages as, for example, Matthew 16:18: “Upon this rock I will build my congregation,” the word congregation therefore becomes approximately the same as the Church, which in our Lutheran Confession is described thus: “The Church is the congregation of saints, in which the Gospel is rightly taught and the Sacraments are rightly administered” (Augsburg Confession, Article VII). In this sense there has naturally always been—and perhaps especially in the days of the apostles—a holy, catholic, Christian Church. In the apostolic age, when the congregation stood in the glow of its first love, all congregations were inwardly bound together by a living love both toward one another and toward the Head, Christ, so that they were more than at any later time one people with “one heart and one soul.” There has scarcely ever been greater and more intimate fellowship among all Christians than precisely then.

If, on the other hand, by “the Church” one understands an outward organization of several congregations under a common external constitution, then it must be frankly and openly admitted that we find no such thing in apostolic times. There is no pope, no presiding chairman, no synods or annual assemblies. There is no confession, no constitution; in short, there is absolutely no organized church body existing outside of or above the congregation. The congregation is the only organization that exists.

When, in very recent times, a high-church periodical in Norway has begun to speak of a “common governing body” in Jerusalem, which was said to possess power and authority over all the first Christian congregations, this is so utterly unprovable a claim that it would be an insult to our readers were we to begin demonstrating that no such governing body ever existed. Anyone who has read his New Testament knows how matters stood. The apostles traveled about and preached and founded congregations, and these congregations ordered themselves in approximately the same manner everywhere, with their presbyters and deacons. And the apostle who was their founder bore them upon his heart; and if he saw any danger drawing near to the newborn ones, he was immediately among them, either with speech or with letter, in order to guard the Word of the Lord and preserve the souls from deception. But there is no sign that anything other than the pure and free impulse of love moved him to speak, or the congregation to listen.

The apostolic office confers no power or authority beyond that which lies in the truth-power of God’s Word itself. That power Paul has; that power Peter has; that power every—even the least—believing witness upon earth has, and as stewards of God’s gifts they are to be esteemed. But no constitution or form of government has given them that power, nor can any human law take it from them. No other power or honor or right or freedom did the apostles ever demand or ever possess. Where the power of the Spirit and of the Word is great and living, as it was in the apostles, there it is both superfluous and wrong to add anything to it or take anything from it by means of human arrangements and constitutions.

The apostolic Church, then, is the congregations scattered throughout the Roman Empire, bound together not by an outward constitution, but by the unity of the Word and of faith, of the sacraments and of the Spirit. From this the definite conclusion may be drawn that the Church is not an order higher than the congregation, that it is not an authority set over the congregation, that it is not a society of a higher kind than the congregation; for if it were so, it would have had to exist from the very beginning. Later church history cannot add anything new of a higher kind than what already existed; it can only develop and unfold, gather together and unite, what was already present from the beginning. The church body can only be the fellowship of congregations; Church and congregation are of the same kind, and the difference is only this, that by the Church we understand a union of several congregations.

On the other hand, it would be quite wrong if anyone were to suppose that, since the apostolic congregations did not form any outward society united by definite societal laws, it is therefore a superfluous—and therefore a harmful—thing for congregations to unite with one another in such an outward manner.

We find, already before the apostolic congregations, clear signs that a connection between the congregations was both desirable and necessary, and that it therefore had to come forth, in its own time, as a mature fruit of development. We shall here mention only a couple of examples, so as not to become overly prolix.

The first remarkable feature is the deputation sent from the congregation in Antioch to the apostles and the elders in Jerusalem, of which we read in Acts, chapter 15. It was the very great question of the relation of the Gentile Christians to the Mosaic Law that was at stake at that time. The congregation in Antioch was divided and of differing mind in this important matter, and it found that it would gain greater certainty if it could hear the counsel of others. Above all, it desired to hear from Jerusalem how one there, in the very center of the old covenant, thought concerning the ceremonies of the old covenant. A congregational meeting was held in Jerusalem, and after much disputation some of the apostles took the floor, and their judgment became the prevailing one and was adopted by the congregation; wherefore the congregation also sent men with word of this decision to the sister congregation in Antioch.

Another example of how the congregations were in need of one another’s support and help we have in the collection of money which, under the leadership of Paul, was undertaken in the congregations for the relief of the brethren in Judea. We find it mentioned in 1 Corinthians 16 and 2 Corinthians 8–9. From these two examples, which are the most prominent and best known, we see that already in apostolic times there was within the congregations a need for mutual assistance, both in spiritual and in temporal matters. The apostles themselves had, in this period, to serve as a bond between the individual congregations and to knit them together.

If it did not immediately come to an outward and formal union, this was because there were so many both inward and outward reasons which made such a constitutional association impossible. For in the first place each congregation had so exceedingly much work with its own affairs that time would not suffice for much more; and in the second place we must remember that communication between the various parts of the world was not nearly so easy in that time as it is now. For a congregation in Jerusalem to get word to a congregation in Rome would, in the most favorable case, take as much time as it would take for word to go from here to Norway, not to mention that persecution could hinder it altogether; and such circumstances must naturally have contributed greatly to the fact that it neither went easily nor quickly to establish a lasting connection between the congregations among themselves.

We therefore believe that we may dare to assert with full certainty that it is altogether unwarranted to claim that the congregations ought not to form outward associations, since we do not find any such thing in the apostolic age. On the other hand, we emphasize with equal assurance that it is altogether unwarranted to make “the Church,” or the outward union of congregations, into a society of a higher kind, of another nature, than the congregation itself. And we shall briefly show how the notion that the Church is different from and above the congregation has been a pernicious falsehood in the Church of God, which has corrupted both the Church and the congregation wherever it has become dominant in the constitution of the Church.

For it was not long after the days of the apostles before outward persecutions and inward distresses drove the Christians to seek a closer union among themselves; and chiefly on account of doctrinal controversies one began to hold church councils, which were intended to bring clarity and peace to the disputed points. But in the persecutions, in the doctrinal conflicts, and at the church councils, it was the bishops—who at that time were approximately the same as what we call pastors—who especially came to the fore. The persecutors preferred to kill the bishops in order to strike the shepherd and scatter the flock; the bishops took part most prominently in the doctrinal disputes; the bishops could more easily than anyone else attend the church assemblies. Thus it became a natural consequence that all eyes were directed toward them, and they attained a prominent position not only within their own congregation but also round about in all the congregations. They became the representatives of the Church in a manner that no one else was. They became the “leading men” in the Church and the spokesmen of Christianity over against heathendom. They often possessed great learning and had often undergone a long period of study, and thereby they gained that influence which always accompanies insight and clarity.

All this was indeed perfectly in order, and a necessary matter for the outward and inward welfare of the Church. But there soon arose the thought that the bishops were not merely the representatives of the Church, but that they were in fact the Church itself. Under the many difficult and intricate conditions in which the congregations found themselves through doctrinal controversies and other offenses, there arose a continual question: who shall decide the dispute, who shall judge the matter, who can speak with authority, so that peace and rest may again be established, and the truth may have free course against falsehood? And it lay near at hand to answer: “the Church”; and if one then asked where the Church was, the answer was given: it is the bishops, or the assembly of bishops. And thus the notion soon made its way that the Church was an authority over the congregation; and if one would find this Church, it must be found in the clergy. The Church thus became a new society above the congregation, and it naturally came to consist of the learned and initiated, in contrast to the unenlightened and the unlearned. This is the beginning of the system which ended in the Papacy.

Yet it remained only a thought so long as there was still some real Christian congregation. For life was still strong enough to hinder the corrupting consequences of such views. But then came the time when the Roman Empire had to bow before the Cross, and Constantine passed over to Christianity. The bitter fruit which the Church reaped from this was that the congregation was destroyed. In one great, unceasing stream the pagan masses poured in over the Church; and when one came to look at what had been bound together, the whole field of the Church lay strewn with gravel and sand and great stones, and only here and there did a little blade of grass sprout, dry and withering, between the stones. But instead of the living, little plantings of the Lord, which Christianity had created round about in the Roman Empire, instead of the living congregations built of living stones, there arose great, proud church buildings of dead stones, wherein the great pagan masses could gather together to be “influenced” by the Word which the clergy spoke. The congregation had vanished, and over its grave splendid monuments had been raised; that which from henceforth was called by the congregation’s holy name was, in its great majority, a dead mass, within which the clergy carried on their mission. What the congregation had formerly been in relation to the pagan world, that the clergy now became in relation to the “dead congregations.”

From this time onward the corruption is unceasing. From now on it is an established fact and a manifest reality that the “Church” is one thing, and the Congregation something altogether different. The Church is the clergy; the Congregation is nothing at all. One soon ceases even to speak of it. It is dead and buried. The clergy have inherited all its power, all its right, all its glory, all its divine truth and force. “The people” — they have become “the laity.” It is superfluous to recall how matters proceeded further along this slippery path, until clergy and laity alike, by this false relation, stood equally far removed from Christianity and its power. Only this must always be remembered: that the heaviest responsibility here, as always, rests upon the blind guides of the blind.

The Lutheran Reformation also brought clarity to this question concerning the Church and the Congregation. And for Luther it became a settled certainty that the Church was nothing other than the Congregation. And had it not been for the grace of the Lord, which once more set the light of the Word upon the candlestick, we should even now not have known the name of the Congregation, but would still be clinging to the old Catholic confusion concerning “clergy” and “laity,” which to this day still haunts the minds and hearts of so many.

But as clear and definite as the fundamental thoughts of the Reformation were on this point, and as directly and firmly as they are expressed in the Confessional Writings themselves, so little did the Lutheran Church succeed in carrying the confessed truth through into life. It is of little use here to ask: why? There are many things that speak in mitigation of the weakness of which the Lutheran Church made itself guilty, and perhaps none of us would have shown greater courage; but the naked truth is this: that the Lutheran Church fastened itself in the arms of the princes, and the princes gladly received the increase of power which they thereby obtained; and what was not given them willingly, that they later took in spite of the Church’s protest.

In this way it came to pass that, after a brief kindling of the thought of the Congregation’s freedom and of the true nature of the Church, the consciousness of the Congregation again sank down into the depths of sleep and forgetfulness, and once more the old notion of “the Church” arose as an authority over the Congregation — with this sole difference, that it was now the king and the royal officials who were the Church and exercised its power, whereas before it was the pope and the bishops who were the Church and had robbed the Congregation of its power.

We therefore have little ground for boasting in this respect over against the Catholics. For if wrong there must be, there is nevertheless more ecclesiastical thought in having a pope chosen by the men of the Church to govern the Church, than in having a king who often has no greater interest in the Church than a thirst to occupy such a place. It is another matter that kings perhaps have not always mishandled the congregation as much as the popes; this does not arise from the fact that royal dominion is more justified within the Church than papal dominion, but from the fact that in the Protestant lands it does not go over so well to treat the people in the same manner as there where Catholicism has made them into an ignorant mass, and the Lutheran doctrine has within itself a counterweight against excessive abuses. But otherwise the royal power has even fairly frequently treated the congregation in a manner which by no means yields the papal power any notable advantage. In the midst of our enthusiasm for the Lutheran Church’s clear assertion of God’s truth, we must cast our eyes to the ground with shame when we consider how matters stand with the freedom and right of the congregation in the Lutheran lands. In Norway we have a fairly striking example thereof in the treatment that was meted out to Hauge. If since that time things have become better in our fatherland, this is due not to the king but to the people.

The notion that “the Church” stands over the congregation, and that it may command the congregation according to its own arbitrary will, is to this day the prevailing one among most Norwegian theologians; and it will no doubt take a good deal of time before it becomes a fully settled matter, no longer open to dispute, that the Church is the communion of saints, in which the Gospel is rightly preached and the Sacraments rightly administered, or that the Church is the congregation, and that therefore the congregations’ union into what we call a church body always is and remains a fellowship of congregations. No higher fellowship comes into being thereby when the congregations unite, such that the congregations should thus be led into bondage; rather, by forming a fellowship the congregations should unite their powers in order thereby to advance their interests, preserve their freedom, and guard their right.

The misfortune, however, is this: that the history both of the Papal Church and of the State Church stands before us as mighty terrors whenever the question is raised of forming a society; nor is the history of the Norwegian Synod a less frightening example for the formation of societies within the free Church. For although the Norwegian Synod is small and its history short in comparison with the two aforementioned forms of church, it is for that very reason so much nearer to us, and it furnishes the striking proof that a free Church (that is, independent of the State) and free congregations do not always go hand in hand. No wonder, therefore, if the long history which shows us “the Church,” whether Papal or State Church, as the oppressor of the congregation, still works with a paralyzing effect upon the development of a free church society. Yet on the other hand we must rejoice deeply that we have this history to look back upon, and are therefore exceedingly cautious and willingly somewhat slow in our development; for the dearly bought lesson that lies therein shall, by the grace of God, become for us a strong guidance in our future work.

We may gather the result of our consideration into these two propositions: that the Christian congregations, by their very nature, must form societies and therefore always have formed societies; but that, on the other hand, the society of congregations, or the Church, is not a new authority over the congregation, but the voluntary union of congregations for mutual help and strengthening.

It is a thing known to us all, that union gives strength; and there is surely no one who does not perceive that if the Church of God on earth can attain greater strength through the union of congregations, then it is not only their right, but also their duty, to form a society.

But the question is how this increased strength may accrue to the congregation and to no other; how the union may become such that it truly works toward the same goal which is the congregation’s own, namely, the edification of the congregation upon the foundation of the apostles and prophets, and the spreading of the Kingdom of God over the whole earth; for upon these two things all congregational labor turns, and if the society does not serve this end, then it has no significance for the congregation.

Yet we add at once, that the strength which is strong enough to do good is always also mighty enough to do harm. If therefore the society is to be of benefit, then precisely the same thing is required in the society as in the individual congregation: the Word of the Lord and the Spirit of the Lord must be leading and reigning in all things; otherwise all becomes harm instead of benefit.
But when congregations now form the fellowship, then it must first and foremost be grounded in unity of faith and confession, and thereafter there must be certain ecclesial aims which are sought through the union. Without a common ground upon which to stand, no church fellowship can be built, and without common aims it can have no permanence.

The common ground which is to unite us as Lutheran congregations is the Lutheran Confession; the common goal toward which we must strive is the congregation’s own edification and the spread of the Word unto the end of the world. These two tasks alone are those which God’s Word sets before the congregation, and the fellowship cannot aim at anything else; otherwise there will continually arise a misdirected labor, a false goal, and a false and destructive striving.

Yet one principal question must still be pointed out before we proceed further, namely this question: Who shall form such a fellowship? It is so deeply ingrained and firmly rooted a prejudice among many Northerners, that it is only the pastors who are workers in the Lord’s vineyard, and that the “laity” are but a dead mass, that to many it seems altogether natural that, just as the pastors are to carry out all other labor in the congregation, so it is also they who are to form the fellowship, they who are to constitute the church body. If this superstition is to be removed, then one must begin at the root; it must be preached again and again that every Christian is a priest, that every Christian is called to be a worker, that no one is a Christian unless he takes the most living part in the work of edifying the congregation, according as God grants him the gift of the Spirit and the measure of faith. If it were first to be rightly and vividly acknowledged, by the light of God’s Spirit, that every member of the congregation is called to perform a service for the Lord, that every member of the congregation individually ought to be a living stone in the Lord’s temple, that every single Christian is a witness of the Lord upon the earth, then it would soon also be acknowledged that it is the congregations which are to form the fellowship, if the fellowship is to be an evangelical fellowship, and not a catholic dominion of priests.

Only there where the congregations truly awaken and form fellowship—only there will there arise a true church body; only there will there be the right goal, only there the right labor; only there will there be that breadth and weight in the work which God’s Word itself describes; only there can there, in the long run, be question of striving unceasingly toward this goal: the coming of God’s kingdom in us and around us. Therefore it is also only there, where the congregations truly form fellowship, that there will be the right permanence and firmness in the fellowship.

The fact of the matter is this: however much an association of pastors may, for a time, be wholly and entirely devoted to its calling, however much it may uplift a community to have none but Spirit-driven men as pastors, yet it is a sorrowful experience that such times are both few and short, and that therefore there soon come times when the lust for dominion, for gain, and for honor becomes a stronger driving force than the Spirit of God and the call of Christ—now in fewer, now in more. At times the community is then torn asunder by strife and factions; at times the whole is swept away in an unceasing stream of corruption. There is a church body that has attempted this; it is the Roman Catholic Church. It soon made the experience that the priesthood degenerated; then it resorted to associations of laymen in monastic orders; but these too degenerated in the same manner. There is only one association that has the promise of standing forever, when it is built upon the Rock; it is the congregation of God. And if the church body is not formed from the congregations, then it will never stand amid the storms of time and against the devil’s cunning assaults.

Short as Norwegian-American church history is, it may nevertheless already offer to the attentive observer many proofs of the corrupting nature of the fact that the formation of church bodies in almost all cases has been almost exclusively a matter of the clergy. In part the church bodies are drawn, consciously or unconsciously, toward the old church road; in part they hover in an unbroken danger of being torn apart and split asunder by personal passions and clerical quarrels. There is presumably no need of any further demonstration of either the one or the other; let each church body lay its hand upon its own breast and see whether it is not true.

It is therefore certain that if we Norwegian Lutherans in America, few as we are, are truly to be able to form a free church body, then it must more and more come to this, that it is the congregations which form the church body. Toward this end there must be labor, early and late, by pastors and by members of the congregations, by laymen and by the learned, in a true brotherhood and in full confidence. And we are more than sufficiently many, if we truly serve this goal as a calling from the Lord, and go to the work with that strength which the Lord willingly gives to those who follow His call and stake their lives upon it.

We have already mentioned both the foundation upon which such a fellowship must be built and the goal toward which it strives. It is proper at once to say that when it is declared of such a fellowship that it rests upon the ground of the Lutheran Confession, this is because the congregation builds upon God’s Word, and upon that alone; and it thus already stands, through the Word and the Sacraments, in fellowship with the whole Christian Church throughout the entire world, which is built upon the same foundation. But when Lutheran congregations are to form fellowship with one another, then the common bond which unites precisely this fellowship is the Lutheran Confession.

All Christian congregations are agreed upon the great common foundation of the whole Church, God’s Word and His Sacraments; the Lutheran congregations which unite themselves into a fellowship are also agreed in the truth that the Lutheran Church has attained the deepest and fullest understanding of the content of God’s Word in its Confession. As long as Christ’s Church has been upon the earth, it has always had to contend with falsehood and error; and just as it lies in the very nature and character of faith that it must bring forth testimony and confession, so it also belongs necessarily to true and genuine faith that it not only bears witness to the truth, but also bears witness against error, against falsehood. Thus it is also the case that the Lutheran Church, through its Confession, bears witness both for the truth which it has recognized, and against the error which it has seen within Christendom; and it is this Confession which is the bond of union of the Lutheran Church fellowship.

It is therefore a necessary requirement of Lutheran congregations which wish to form fellowship that they truly, and with full conviction, adhere to the simple Lutheran Confession. When, in the Norwegian Synod, one has wished to go beyond this and to demand agreement also in doctrinal propositions which go far beyond, and in part far away from, the Lutheran Confession, this is a deviation which doubtless arises from the fact that the entire fellowship is so essentially a clerical fellowship that it forgets that the congregation cannot make its own confession, and cannot adopt doctrines, which a handful of pastors agree upon at an annual meeting or a pastors’ conference. That which is to bind the congregations together must be the old truth, well tested in the living experience of the Church, which through catechetical instruction has also become the personal experience and spiritual possession of each individual Christian. To attempt to bind any other bond upon the congregation and the congregations—this becomes bondage and inward untruth, however much one may paint upon the chains: “This is the pure doctrine.” The whole matter is and remains a complete misunderstanding, both of the significance of the Confession and of the nature of the congregation and of the fellowship.

If there is, then, to be an outward church fellowship, there must be a definite and recognizable Confession which binds the fellowship together; and if a congregation cannot give its assent to this, then it cannot belong to the fellowship either. For there is then lacking that inward unity which is necessary if an outward unity is to endure.

If it is now granted that we must be in agreement concerning the Lutheran Confession, or—which is precisely the same—our Lutheran Catechism, then it is thereby also said that the free church fellowship has no right to exclude from itself any congregation that holds seriously and firmly to this Confession, even though such a congregation should be unwilling to acknowledge opinions and doctrinal views which either an individual man or a plurality within the fellowship embraced and regarded as matters of controversy. There must, in a fellowship which would avoid becoming a party, be a renunciation of all coercion, and there must be full freedom and equal right for every Christian, and for every Christian congregation, to expound and interpret the Word of God according to the measure of its faith and the gift of the Spirit; for the opinion of a single man, or of an accidental majority, concerning a point of doctrine always becomes a straitjacket whenever it is made binding upon all by the bonds of fellowship.

It is necessary, if the fellowship is to be preserved in the truth, that there be room for personal conviction, room for personal experience, room—in a word—for life; and therefore it must be borne with, even though it is an imperfection belonging to the conditions of the Church in its state of conflict, that within a fellowship there are differing opinions in many matters, and that there is open access to the exchange of views, without one party at once beginning with judgments of heresy against the other and expulsion from the fellowship. This relation also can be attained only through a genuine congregational fellowship; for clerical societies have always shown themselves wholly unfit to preserve that balance which the congregation possesses by the fact that it ever rests with calmness and faithfulness in the Word and the Sacraments, and asks first and foremost for life in the simplicity of the Catechism, not for lofty and grand doctrine. Therefore it tolerates no departure from the Catechism or the Confession; but in return it is not so narrow as to pronounce judgments of heresy like one who, filled with an imagination of his own wisdom, scents heresy in everyone who is not of his opinion in every matter.

Just as important as it is to be fully clear concerning the foundation of the ecclesial fellowship, so that it is made neither too broad nor too narrow, so equally urgent is it to be fully clear concerning the goal that is set before us; for only thereby can we be preserved from corrupting directions and from the most painful experiences. And only when we have the right goal can we also, with clarity, choose the right means. That goal for which we labor together must likewise determine what is to be a matter of the society and what is to be a matter of the congregation.

We have said that the goal is the congregation’s own edification and the spread of the kingdom of God over the whole earth. We cannot adopt any lesser or any narrower goal.

It is evident that all societies will claim that they pursue this goal; and if they truly did so, then all churchly conflict would rest either upon mere misunderstandings or upon disagreement over trifles and useless questions. But alas, we often find even of ecclesial bodies that “they say it well, but do it not.” It therefore becomes necessary more closely to explain what we mean when we say that the goal is the edification of the congregation into a living temple for the Lord, or the growth of the congregation upward unto its Head, Christ.

It is not our meaning that the edification of the congregation into God’s temple consists in an outward constitution, whereby the Church is transformed into a well-ordered assembly of people who bow to certain outward church laws, and by a determined external order set each one in his place from the highest to the lowest. For with all such outward order one may indeed set a splendid building in place; but it lacks—everything; for it lacks life. God does not dwell therein. It is then not the Spirit of God who joins the whole together, but outward commands and ordinances.

Nor does the edification of the congregation consist in this, that it establishes higher and higher dignities in its midst, beginning with priests who constitute a higher kind of people than “the laity,” then setting bishops over the priests, archbishops over the bishops, and finally a pope over the whole. This too is an “edification,” indeed a sheer “piling up,” until it becomes “a tower whose top reaches unto heaven”; but it would be quite peculiar to call this “the edification of the congregation.” It is indeed growth upward, but not unto Christ; it is self-exaltation.

Neither does the edification of the Congregation consist in this, that the Congregation puts on a uniform, so that the congregations become like small troops in an army. If this uniform becomes nothing more than an outward garment, which otherwise has nothing to do with the man himself, then it may indeed serve to produce a well-ordered party that presents a fair appearance before men; but when the Lord comes to muster the army, there may perhaps be none whom He knows; for His mark is not the outward, but the inward, the holy seal upon God’s Congregation: “The Lord knoweth them that are his.” It is precisely the same whether the uniform is a particular form of doctrine or a particular form of life; whether it consists in a lesson that is learned by rote, or in a habit that is practiced. All such “edifications” of the Church and the Congregation we ought rather to combat.

The edification and growth of the Congregation is the work of the Word and of the Spirit; it therefore cannot be promoted, but is hindered, by all human thought and all human work that does not have its ground in the Word and is not driven by the Spirit.

The edification of the Congregation into God’s temple, and its growth up unto the Head, is first and foremost wrought by God’s Word and His Sacraments, and by nothing else. Therefore it must be the highest striving of the Congregation and of the fellowship to preserve these pure and unadulterated. Faith is the proper fruit of the means of grace, and it is that wondrous power which separates the Congregation from the world and binds it to God. Through it Christ becomes ours, and the image of God is renewed within the heart. Christ takes up His dwelling in us, and the Congregation becomes the new race of God’s children, which indeed walks despised and smitten, yet also liberated and saved through a world held in bondage. This, then, is what we mean by the edification of the Congregation: its renewal through faith and its holy union with Christ, its growth in purity and virtues, its increase in freedom and life.

In this growth there takes place, through the power of the Word and of faith, both a continual purification and a continual sanctification. The Congregation is cleansed from sin, and indeed also separates from itself those who once clung to the Congregation, but more and more closed their hearts to the Lord’s Word. The Congregation is sanctified, in that it is ever more deeply immersed in the Word and penetrates into its depths, is permeated by the life of God, united ever more inwardly with Christ, and more and more presents itself as the race of God’s children, where there is the freedom of the Spirit and the love of the Spirit poured forth.

To this inward edification the outward answers. Thus the congregation becomes the dwelling of God in the Spirit, the city of God, which is perfectly fair, perfectly ordered, where there is no disturbance. God’s house is the home of order, where each knows his place; God’s city is the fortress of peace, where all things have their proper rule; God’s people are a host of warriors, where each hero stands at his post and on his watch. Yet the chief matter is always this: that the inward edification in Spirit and faith of God’s people must be the first, and the outward form must follow therefrom.

Ah yes, you will say, but this goal we never attain in the land of corruption. Yes, that is true; it is heavy. Yet herein also lie the power and the victory, that we have a calling, a goal, a crown that is lifted high above corruption and grave and death. Herein lies the strength of the Church and of the fellowship, that it boldly lays hold of its anchor within the veil. There first is the harbor, there is the rest, there is the crown. If the goal be not so high, it is not fit for God’s Church. She must lift her gaze so far and so high that it reaches beyond all the greatness of the earth and all the falsehood of the world. If she takes to herself a lower goal, then she perishes with the world, then she is lost with the world, then she falls with the goal which she pursues here below. The Church strives toward the goal of becoming God’s people; but John says that this goal is reached only then, when the new Jerusalem descends from heaven as a bride adorned for her bridegroom; then shall the tabernacle of God be among the children of men, and he shall be our God, and we shall be his people.

If the Church has set the right goal for her own edification, then thereby she has also the right goal for her work outward. If we cannot rest until we reach the heavenly Jerusalem, and only then deem the Church’s task here fulfilled, when God dwells in his people in everlasting glory, then neither can we halt until the gospel of the kingdom has been preached to all nations. Further and further onward, farther and farther out must the message of salvation press, until the testimony of the crucified fills the earth, until it sounds from sea to sea, from pole to pole.

God’s Kingdom overthrows all the kingdoms of the world; it fills the whole earth. Every Christian community must bear within itself this nature of God’s Kingdom, otherwise it is of no use in the Lord’s household. One often calls the work “a great community,” and it is right to call such work great, if it is not a spiritual work, a work in the spirit of the world and with the world’s means. Yet it also belongs to God’s calling in the lands to labor in the inward love of Christ for the salvation of all men. That community which settles down in the thought that we are many enough; that which, in particularity and pride, will have no others with it than those whom it can mark with its own mark and stamp as its own; that community which will have only those with it who possess a certain stature and a certain experience and a certain learning, which it itself has taken as the measure and standard of Christianity; that community which will not cast its net as wide as the sea and draw the net, even though there should be rotten fish in the catch—that community does not have the true love of God and the perfect mind of Christ. Toward the crown and out to the ends of the world the Christian community, like the Christian congregation, casts its gaze; and even though it knows that death lies between us and the crown, and that heathendom and the coldness of the world lie between us and the goal of victory, yet it is calm and unafraid—in faith.

He who sets himself a small goal becomes small; he who takes a great goal becomes great by his goal; but he who dares to stake everything upon the eternal goal, who dares to give up all earthly goals in order to attain the heavenly, he becomes greater than all. The Prince of our salvation has, through the suffering of death, bound to himself a name above every name; and he is also the one who leads his children into the same glory, if they dare all, surrender all, sacrifice all in order to win him.
