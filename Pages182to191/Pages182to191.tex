%FIXME: Recast without the KJV resister.

\begin{center}
\includegraphics[width=0.9\textwidth]{OpenImage.png}
\end{center}


\subsection{Do You Want a Congregation?}


\textit{Source: Editorial articles in *Folkebladet* for September 29, 1897; March 16, 23, and 30, 1898. — Ed.}

This section appears on pages 182–191 of the original volume. — Present Ed.


\bigskip


\subsubsection{Is It Pride?}

In “Lutheraneren,” J. R. S. writes: “Yes, I understand that it is a lawfully great portion of tempting pride and censorious judgment when the ‘Free Church’ now makes it its watchword: ‘Do you want a congregation?’—as though there had been no congregation among us before, but one first had to be brought to us by the emissaries of the ‘Free Church.’”*\footnote{“Lutheraneren” of September 8, 1897. The quotation is found in an article entitled “Compulsion or Vitality,” which forms a reply to an editorial article in *Folkebladet* of September 1, 1897 bearing the same title. The question, “Do you want a congregation?” was raised during the discussion of the topic “Congregation and Congregational Life” at the annual meeting of the Augsburg Synod in Fargo, N. Dak., in 1896, and in a certain way became “a charged phrase.” — Ed.}

J. R. S. maintains that it is pride and censorious judgment to ask: Do you want a congregation? because such a question would only then be reasonable if there had previously been no congregation among us.

By this, J. R. S. can of course not mean that in one out of a hundred—or in ten out of a hundred—places there exists a congregation. For what existed in some places could not, after all, help the other places where it did not exist.

Or if we were to ask a sick man: Wilt thou be made whole?—could the sick man then put on an offended countenance and answer: Thou speakest as though there were no healthy people among us?

J. R. S. must, if indeed he means anything other than to be coarse, mean this: that the question, Will ye have congregation? is superfluous everywhere, because everywhere there is congregation among us. There is no need for change, for everywhere among the majority of the people the congregation is truly congregation; it answers to its name; it is mere “pride and censoriousness” if anyone thinks that the congregations are not at all—perhaps never—everything they ought to be and could be.

Then it must surely also be “pride and censoriousness,” what stands written in our catechism concerning “the Christians”:

“The name indeed they all have, but the Spirit and manner of the anointing most of them lack.”

The world has always fended off the demand for true and living Christianity precisely with this word of the majority: We have all this already in order; it is only pride and censoriousness that make some think that we are not all good Christians together.

This is the false prophecy of the worldly church, at all times and in all places like unto itself. Here is peace and no danger. Our Christianity is good, our doctrine is pure, our Christians are good Christians, and our congregations are good congregations. Dear people, sleep in peace, sleep in peace!

But what says the true prophet? From the Lord he says with piercing earnestness: Ye heal the hurt of the daughter of my people lightly, saying, Peace, peace; when there is no peace. And of himself the true prophet says: Oh that my head were waters, and mine eyes a fountain of tears, that I might weep day and night for the slain of the daughter of my people.

Let us look a little at the meaning of that for which the Free Church labors; then perhaps it will show itself that the question after congregation is not unjustified in our midst, and that there is more pride in boasting that all is in order than in acknowledging that we are sorely lacking in being what we ought to be.

Perhaps, if J. R. S. were allowed to see what congregation is, he would no longer come with foolish jests about “the Free Church’s emissaries” being supposed to bring it along with them. He knows well that there is no one who believes anything of the sort; but we do indeed believe that the Gospel brings it with it, wherever it is permitted to do its work in full.

The Church is the Body of Christ, filled with His Spirit and His Life, so that it both represents Christ’s own foundation and performs His work. “Now ye are the body of Christ, and members in particular.”

We will mention no more than this one thing in this connection. It is this, namely, which cannot exist without “by the Holy Ghost.” It is the Spirit of God who works this miracle, that in the darkness of the world the Church is light; in the corruption of the world it is salt. It is the Spirit of God who makes the Church a gathering of laborers in the Lord’s great harvest, where each and every one is active in the great work of love: to gather in from the world’s perdition unto the Savior and His Kingdom.

Do you want Congregation? Does this question signify something evil? Or is there anyone who can answer: We have enough of that already? Ah no. We have not had enough of this yet. Nor can we here on earth ever have enough of it, or even too much. For the question plainly means this: Do you who in name and in form are a Church also desire the Church’s true Spirit and Life and Work, so that you are no longer merely paying members of an association which maintains church, pastor, and graveyard, but truly are spiritual members of the Body of Christ, active in His work of love upon the earth?

If anyone would answer: We already have this sufficiently, does that not mean that he is satisfied with the form without the life, and content with the letter without the Spirit?

For he who has begun to look into the truth of the divine rebuke: “I know thy works, that thou hast a name that thou livest, and art dead,” will surely acknowledge that it is most necessary to ask after the Church in spirit and in truth among us.

And when we think of how many there are among our churches and church-members who take firm comfort precisely from this cradle-song: All is in order, sleep in peace—then we are almost tempted to beg *Lutheraneren* and the Majority and J. R. S. at last to put an end to it.

The matter is this, that we stand in our ecclesiastical development directly before a great crisis, in which this is the question that must be answered: Is the old state-church order truly answering to the demand and need of our time and our land and our circumstances, or not? It may be all the same whether one holds the old church order to be sufficient or not in the old lands and the old times. We ask: Can the Lutheran Church solve her task here among us by continuing in the old tracks?

We say no. It is necessary that we awaken and recognize that there is need of a congregation. It is necessary that there be the participation of all in the work; that there be earnest renunciation of the world, living love, burning zeal for the salvation of souls—not merely among the ministers, but among all the members of the congregation.

Spiritual interest must be awakened, nourished, and developed in all. It must not be so that, for example, the heathen mission is a minister’s affair, which the minister presses upon the congregation as a demand for sacrifice once a year or twice; but that it becomes dear and precious to each individual member of the congregation, from the child to the grey‑haired elder, so that they all love the cause, live for it, pray for it, suffer for it, read about it, save for it, and offer for it. All, all at work for the cause of Jesus Christ, so that the minister’s task becomes that of encouraging, urging, guiding, and instructing. Then the congregation has its true character; then the work is congregational.

Oh, how much greater power there would be if we became a congregation!

Has J. R. S. heard of Hermannsburg, where Harms was made an instrument to bring forth a congregation? Is it everywhere thus among us? If not, why not?

When we have then seen a very little grain of what the New Testament means by congregation, do we not then have the right to speak and testify concerning this matter? Is it pride to ask the people whether they do not desire to have it as the New Testament presents the Christian community on earth?

Surely it cannot be more prideful than to ask people whether they will be Christians. That does not mean that we can make them Christians; but it means that God can. So also with the congregation.

By the force of the old church order it has here among our Norwegian people been asked again and again: Will you have a minister? Will you have your children baptized? Will you have the Supper administered? Will you have the Word of God preached? Will you have the dead laid in the earth?

This was not pride, nor judgmentalism; it did not even touch the conscience directly. But when all this is taken together into a single question, and we do not merely ask whether something ought to be done for the people, but whether the people themselves will join in and do something for the power and forward movement of the Kingdom of God—then it is to be called pride and judgmentalism. Was it not the two-edged sword that touched the sleeping mind and the dead heart? Or why this bitterness?

What in politics is called popular liberty is in the Church called congregation. It signifies the participation of all in the work and the development, and in the sorrows and joys that follow therefrom. Is it truly wrong to agitate for that cause, even in the form of a question?

\subsubsection{Will ye have congregation?}

It has been called an irritating question; we would rather call it an uplifting question.

It is a good that is offered, and a gift which none can take except the one who receives it.

But every good gift and every perfect gift is from above, and cometh down from the Father of lights, with whom is no variableness, neither shadow of turning.

Therefore congregation also is a gift of God, a good and perfect gift, which God gives to those who have room for it.

For the congregation was not, before the Holy Spirit was poured out. The day the Spirit came upon the disciples in the upper room, those disciples became congregation.

If anyone says that they have congregation, but they do not have the Spirit, what profit is there in that? Is it God’s congregation?

If there is to be congregation among us, then we must receive the Spirit from God, so that we may become one body and one Spirit, a dwelling of God in the Spirit; so that if the Spirit is God’s gift, then the congregation is likewise.

Is it necessary for us to receive the Spirit? Or is it ever possible for anyone here on earth to have the Spirit in such a manner and to such a degree that he has no need to receive Him? All who have ever had the Spirit know this: that we have only so much Spirit as we receive. But God giveth the Holy Spirit to them that ask Him.

Of all gifts this gift is the greatest. Why, then, should it be irritating to ask people whether they will have it?

But those who answer: Yes, we gladly desire Spirit and congregation, ought to consider what this entails. For the Spirit of God brings with Him all good—righteousness, peace, joy, and all blessed fruits; but the Spirit of God also brings with Him conflict with the flesh, enmity with the world, the raging wrath of the devil and of hell. The Spirit of God is a Spirit who drives to the following of Christ upon the path of suffering and of the cross, and who constrains, by the power of God’s love, to seek after that which is lost and to labor for the salvation of souls.

Therefore the congregation is no fellowship of sloth and ease. It is the living, contending, and suffering fellowship of love.

As the Head, so the body; as the body, so the members. If Christ has loved, labored, sacrificed, and suffered, then His body and bride must be prepared for the same. To love Him, to labor with Him, to sacrifice oneself for Him, to suffer with Him—this is the congregation’s condition and calling.

Is it then so foolish to ask: Will you have congregation? Or is it so certain that we may content ourselves with the sleep-inducing speech: We have congregation already; there is no need for anyone to come to us and speak of congregation. It has been the custom of the sleeping congregation in all ages to comfort itself with the thought that all is in order. But it has always been a false comfort, and therefore the Scripture says: When they shall say, Peace and safety! then sudden destruction cometh upon them.

Wisely, the question is justified now as at all times. And we dare to add: it is all the more justified, inasmuch as there is all the greater occasion to obtain congregation. The outward hindrances which barred the congregation’s unfolding and manifestation have been swept away; God has set an open door before His people. There is a time of visitation among us, when the Lord is to be found by those who seek Him, and is near to those who call upon Him. It seems to us that never has the question been more timely:

Will you have congregation?

\subsubsection{What is congregation?}

It is indeed a delight to be permitted to ask this. For the whole New Testament aims precisely at this: that there should come to be congregation.


God sent the Son, that He might present unto Himself a glorious Church, without spot or wrinkle or any such thing, but holy and blameless.

God sent the Spirit to create the Church, that Christ’s suffering and resurrection might have their perfect fruit in her.

Christ’s Apostles labored and suffered for this cause, that there should be a Church of Christ in every place where the Gospel made its advance.

Therefore in the New Testament everything concerns the great mystery, which is Christ and the Church. And therefore it might seem that every child in Christendom ought to know what the Church is.

And it is simple enough, as is all God’s Gospel, if only we were not blinded in our unclean hearts and sinful thoughts.

It is the old story: we will not acknowledge our frailty; therefore we simply insist that we already are all that we ought to be, long since. We “have kept the commandments from our youth up”; we were good Christians before and have no need to become so; we are already Church, and require no change in that matter.

But if we are already Church, then it is incomprehensible what the Word of God means by Church. The one does not agree with the other; and since we will not correct ourselves according to God’s Word, then God’s Word must needs correct itself according to us; and we behave like poor cobblers who, when the shoe does not fit the foot, would make the foot fit the shoe.

How men have twisted and turned and distorted the Scripture and the theological conscience, to make fit together what does not fit!

These are deep matters and a great mystery, it is true; and yet it is “revealed unto babes.” The Church is the Body of Christ.

That this is a divine mystery revealed in the Gospel is altogether plain. And yet there is no Christian soul who does not understand that the Body of Christ is there, and only there, where the Spirit and life of Christ are.

This therefore means that the Church is a people of believing men, born of the Spirit of God, in order that the Spirit of God may also dwell in them.

But faith both comes from “hearing” and is sustained by it. The Word of God and the Sacraments are the means for the Church’s sustenance of life; therefore they are used in the Church by great and small alike.


From the Spirit of God comes the love of God. Therefore the congregation is a people in the self‑sacrifice of love no less than in the devotion and rest of faith.

Therefore there is no congregation without gifts of grace with which to work and to serve; for love could not be, if it were not allowed to act. But where love comes forth in work, there the gift of grace is given.

Much lies in this, that the congregation is a people in faith and love. Herein lies the chief difference between the world and the congregation. All worldly men cleave to the visible things and seek their own. Believing men cleave to the invisible things and seek the glory of God and the good of their neighbor for time and eternity.

And when we consider this and seek to measure reality by it, then we begin to understand what is meant by worldly and world‑conformed congregations. For if this is the nature of the world, to strive after earthly things and to seek its own, then indeed there is much worldliness in all our congregations. And if this is the true nature of the congregation, to strive after the invisible and eternal things and to live for the cause of God’s kingdom and the salvation of souls, then there is not very much of the congregation’s true being in our congregations.

Can it be otherwise? — We will answer with another question: Should it not be otherwise? We know that it can be otherwise, if the unconverted are converted, if the sleeping awake, if the awakened go into the vineyard and labor.

We know that it can be otherwise, if only it becomes otherwise with each one of us.

Therefore it is right to say that there can be no congregation without awakening. If sleeping congregations are to become awakened congregations, then they must be awakened. But this is unpleasant and painful, so that many worldly members of the congregation are deeply offended. And it often comes to pass that death stands life so hard against in the worldly congregation, that either life is stifled again, or else the awakened are driven out.

Woe unto the world because of offences!

When therefore just now our church people are being shaken in the sieve, and the question is asked in many places: Shall we awake and become a congregation of God? then it behooves each congregation to consider what belongs to its peace; for it goes ill with that congregation which “knoweth not the time of its visitation.”



\subsubsection{The Body of Christ}

When the Church is the body of Christ, this shows both how infinitely condescending He is who is Lord of all, and at the same time how infinitely exalted is the calling wherewith the Church is called. It also lays the very greatest responsibility upon all who bear the Church’s holy name.

For what, then, does Christ will with His body? What does the vine will with its branches? What is it that is promised?

It says itself that Christ wills to give His body Spirit and life and blessedness and glory. It says itself that the vine wills to give its branches sap and strength. To be within the Church and not receive Spirit and life from Christ—that is formalism and hypocrisy. If anyone would be a true member of the Church, he must be crucified with Christ in order to live with Him. Dead to the world, living in the power of Christ’s resurrection—such is the right position within the Church.

Yet with this the matter is surely not finished. The soul dwells in the body in order that it may give itself to expression and be active through the body. The invisible and ascended Jesus lives in His Church in order still to be revealed and to work in the world. The branches upon the vine are not only to live; they are also to bear fruit.

From this we gain some insight into the Church’s high calling, and why it is so serious a matter to walk worthy of this calling.

The Church is to reveal Christ in the world, so that souls may see Him, know Him, believe Him, and live by Him. The Church is to do Christ’s works in the world, so that His love and mercy may be felt through the whole course and conduct of His Church. It is not merely a question of being a living branch, but of being a fruitful branch.

A number of people—especially pastors and theologians—so easily absolve themselves from this calling and responsibility. They say: yes, of course the Church reveals Christ in the world; that is why it has a pastor and the preaching of God’s Word and the administration of the sacraments. When the Church has these things, then its calling and task are fulfilled; and this costs so much per year, paid in ready money or in usable goods.

Ah yes, it was simple enough—far too simple, low, fleshly, and heathenish! It is altogether true that without Word and Sacraments the congregation cannot fulfill its calling. But there is a better way for Christians to fulfill their duty of witness than to hire a man to do it in their stead. The congregation does not have a pastor in order that it itself may escape from its Christian duty and lay it upon the pastor. Rather, the congregation has a pastor so that the pastor may encourage, admonish, and spur the congregation on to true and living Christianity in heart, word, and deed.

Alas, it is perhaps not entirely unheard of that people have a pastor in order that they may sleep securely, sin securely, and die in false peace. And it happens, sadly, that there are pastors who do not see the people’s danger and therefore strengthen them in sleep and sin by a dead preaching; but thereby neither the task of the pastor nor that of the congregation is fulfilled.

The whole congregation, and every single member of it, is bound to walk worthily of Christ’s Gospel; and the whole congregation, and each of its members, is called to manifest Christ in the world by proclaiming his excellencies, who called them out of darkness into his marvelous light.

If, therefore, there are found among us organizations which call themselves congregations and dare to take God’s Word and Sacraments into use, then the responsibility of such organizations is as great and as holy as the name by which they have called themselves. Woe unto them if they have taken the name and do not answer to it. Worst of all, if they not only themselves lay claim to being a congregation, but even forbid others to be a congregation in that place where, under the name of congregation, they also claim the congregation’s right.

This has often been attempted among our people.

This, then, is the chief question: Do our congregations, as a general rule, correspond to the calling that lies in the name, the Body of Christ?

Is Christ manifested in the world through the testimony and walk of the congregations?

Do they do the works of Christ and nothing other than the works of Christ?

Do they bear all such fruits, and only such fruits, as are well-pleasing to God?

Or is there need of some change?

We indeed hear a great clamor from those who grumble and say: Here all is in order; here no change is needed, neither with us nor with our congregations.

But we also hear a sigh unto the Lord from the breast of those who long for congregation in Spirit and in truth: Pour out the Spirit upon us and upon our congregations, that we might become what we ought to be.

