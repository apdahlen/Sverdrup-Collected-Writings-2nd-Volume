
\begin{center}
\includegraphics[width=0.9\textwidth]{OpenImage.png}
\end{center}

\subsection{The Unity of the Believers}

Source: “Luthersk Tidsskrift,” 1907, pp. 65–74. See the note above, p. 344. — Ed.

This section appears on pages 371–381  of the original volume. — Present Ed.

\bigskip

\begin{quote}
“Ye are all the children of God by faith in Christ Jesus.” Gal. 3:26.

“Ye are all one in Christ Jesus.” Gal. 3:28.
\end{quote}

When we, with our human understanding, look out over the divided and torn Christendom, it appears senseless and hopeless to speak of the unity of the believers. It does not seem to be present; nor is there any prospect of ever attaining it here upon earth. It would be a superfluous waste of time here to enumerate the manifold proofs for this opinion of human reason and unbelief concerning the matter. We have heard them so often already, and they lie openly before our natural perception.

But thereby the matter is not settled; for here there is no question of a carnal matter, which can be judged by a fleshly mind. Here spiritual perception is required, which discerns spiritual things. And this is certain: that if the believers come to a clear spiritual recognition of this chief matter in Christianity, much will thereby be accomplished toward removing the stumbling blocks which disagreement and division within Christendom occasion for many.

The unity of the believers is not brought about by the believers’ assembling at some place or other and resolving to be in agreement, or to organize an outward society, or by any other outward endeavor.

The unity of the believers is brought forth by faith itself. Before meetings and agreements, before constitutions and resolutions, the unity of the believers is already present in and through their faith. Just as the human race is one by the “bonds of blood,” so the believers are bound together into a unity by nobler and finer, and therefore also more intimate and stronger, bonds.

It goes without saying that this analogy is justified; for it is God who “hath made of one blood all the race of mankind to dwell upon the whole compass of the earth.” And it is the same God who has made it so that the believers have one Spirit and one life everywhere they are found. The new race is the raising up of the old race from the Fall, and its salvation from sin and from its inherent corruption.

When, therefore, for our edification, we would contemplate the unity of the new race, there meets us first and foremost this established fact: that all who belong to the new race have one Father.

\textbf{The same Father}

There cannot be too much weight laid upon this matter. If it stands at all times a living reality before the consciousness of believers that they are “born of God,” then they become both firmer in faith and stronger in love; their fellowship with God and their love toward the brethren grow more intimate, the more clearly it stands before them that God is their Father.

It is the chief distinguishing mark of the “elect race” that it is a race born of God. This separates it from the world; this binds it together in the Spirit. It is precisely this which is the right and true glory of election, that it reveals itself in the relation to God and in kinship with Him.

There is a brief narrative in the First Book of Samuel which perhaps has not received from all the attention it ought to have. It stands in chapter 10, verses 10–12:

“And when they came thither to Gibeah, behold, a company of prophets met him; and the Spirit of God came upon him, and he prophesied among them. And it came to pass, when all that knew him beforetime saw that, behold, he prophesied among the prophets, then the people said one to another, What is this that is come unto the son of Kish? Is Saul also among the prophets? And one of the same place answered and said, And who, then, is their father? Therefore it became a proverb, Is Saul also among the prophets?”

Could it be said in a more arresting way that for a prophet it does not depend on whether he has Kish for his father? What matters is to have God for one’s Father. He who is born of God speaks by the Spirit of God; and at all times and in all places it is a miracle of God when “even a Saul is among the prophets,” that is, when one who according to his natural birth is nothing but a corrupted child of sin, by being born of God has become a child of grace, who is driven by the Spirit of God, his Father, in word and in deed.

“We have not all one Father? Hath not one God created us?” says the prophet Malachi. And he does not speak of the first creation, according to which we are all children of Adam; but he speaks of the new creation, by which God creates new men, the true children of Abraham through faith.

This new race of God’s children are the true men. For the filial status with God does not lift believers up into a new superhuman dignity, so that they may look down upon other men with contempt; no, it lifts them precisely into true human dignity, so that they again become what men were created to be—the image of God. Far from making them proud and arrogant, insensitive and egoistic, the Spirit of adoption fills them with the love of God; and as they themselves have begun to understand how great a thing it is to be true men, it becomes their highest delight and the longing of their hearts to lead others forward into the same joy and glory which they themselves have experienced.

Only see what is written of this in Acts 10:25–26: “And as Peter was coming in, Cornelius met him, and fell down at his feet, and worshipped him. But Peter took him up, saying, Stand up; I myself also am a man.” It is not a matter of crushing down, but of raising up, that concerns Peter. So far from boasting that he was more than Cornelius because he was a Jew and a believer, he precisely lays aside the Jewish boast, because in Christ he was a true man, a child of the heavenly Father.

This new race of God’s children thus bears within its very adoption this calling and this impulse, that it must seek to raise up all fallen men and gather them into that fellowship and kinship of God into which it itself has entered. Love, compassion, mercy are therefore the words that designate its relation to men, because they are all created by the same God and all by nature are the same Father’s lost sons and daughters, whom it concerns us to call back to the Father’s house.

In this, that they are children of one Father, the new race is nevertheless far—far in advance of the old. For in the old race the mutual kinship does indeed become more and more distant as time goes on and the number of generations increases. True as it is that we are bound together by “the ties of blood,” we nevertheless know both from the history of the family and from daily experience that it is only the very nearest relatives who have any awareness of this bond; and it is not long before even those who belong to the same people stand as strangers over against one another.

But it is not so with the children of God. At all times they are, in the strictest sense, one another’s brethren. Here neither family nor people, tribe nor race, land nor continent makes any difference. If we meet a Christian man or woman from Europe or Asia, Africa or Australia, at once the Christian brotherhood is there; for we have indeed one and the same Father, Him who is in the heavens. The believers’ register of descent is so short: it is only child and Father—and then no more. For Jesus says: Call no man your father upon the earth; for one is your Father, He who is in the heavens.

Thus neither time nor place can abolish this unity of the believers, that they have one Father. Not even the circumstance that they have been led home to the Father in heaven by different instruments ought to make any breach in this honor and joy of the believers. It is perhaps here most often that something intrudes which would disturb the unity. But whether we are Lutherans or Wesleyans or Zwinglians or Calvinists, we nevertheless have only one Father, He who is in the heavens—if indeed we truly are God’s children. Not that it is right to despise the instruments which God has used for our salvation; but neither is it right to set these instruments in the way of our fellowship with God and God’s children. They help us on the way; they do not hinder us from arriving, if only the journey proceeds rightly. And if we arrive at the Father’s heart by the one Mediator Jesus Christ, and by the one Spirit become the one and the same Father’s children, then the difference which exists between the various divisions of the Church will not be able to make the children strangers to one another.

It is therefore not said that the difference is without significance; but of that we shall not speak here. Only this must here be emphasized with the utmost strength: that all true believers are bound together into a living unity, since they all are of one Father.

But we cannot stop there in our contemplation. The Word of God and Christian experience show us that the unity of believers is not only in the Father, but also in His Son, inasmuch as they have only one Savior, Jesus Christ.

\textbf{The same Savior}

Here is the center of Christianity; its true essence may in the briefest manner be designated as faith in Christ. It is precisely for this reason that our religion is called Christianity: that it revolves around Him who came into the world to make sinners saved. Therefore this also is the answer to the question of the soul’s anguish: What shall I do to be saved? “Believe on the Lord Jesus, and thou shalt be saved.” For faith in Christ is not a cold philosophy that leaves a man in his misery; it is a way of salvation—yes, a salvation—that makes the dead alive, and that consists precisely in this one thing: to believe on the Lord Jesus Christ. Without Jesus there is for us no access to the Father; without Jesus there is for us no work of the Spirit. In the matter of our salvation everything rests upon Him who went forth from the Father and came into the world, who left the world again and went to the Father.

It therefore cannot be otherwise than that the unity of believers, in its innermost essence, is also unity in the faith of God’s Son—indeed, in God’s Son Himself, Jesus Christ. So intimate is the union of believers in Christ that it is not only said of them in John 17:20–23 that they shall be one in the Father and the Son, but in Galatians 3:28 it is said even more strongly: Ye are all one in Christ Jesus. Not merely one essence or one nature with the Father and the Son, but only one personhood, inasmuch as Jesus is the Head of His body, which is the Church.

For Jesus is our Mediator and Reconciler, and faith in this crucified Savior is such a surrender to Him that we let go of everything of our own—our will and cognition, our thought and feeling, our own sin and our own death—and lay hold of Him, Him alone, with all that He is and has, so that He becomes ours and we His for time and eternity. The believers’ union with Jesus is as intimate as Jesus’ union with the Father, as He Himself says: “that they all may be one; as thou, Father, art in me, and I in thee.”

And this means so much for the unity of the believers, namely, that it does not come into being by their joining together, loving one another, praying with one another, and laboring together; but rather their unity goes before all these things and forms the foundation and presupposition for them. For the believers’ fellowship, association, mutual love, and cooperation all rest upon this, that through faith they are united with Jesus Christ as the members with the Head unto one life and one body.

Let us therefore not go astray, friends, so that we imagine we can bring about or fabricate the unity of the believers by outward, church-political undertakings. It comes only through faith in Jesus, and by no other means. Herein lies, in truth, the deepest root of the great and destructive Catholic error, in that the Catholic Church places connection with the Church or the Pope in place of connection with Jesus Christ, when the question is how the unity of the believers is to be produced. In this way one departs from the necessity of personal conversion and personal faith; one merely unites oneself with the congregation or the Church, and then the Church provides for the union with Christ through the priests and bishops and the Pope.

Oh, how tempting for the carnal mind and the natural man is this Catholic conception! It steals in so unnoticed even among the most earnest Christians, if they do not continually stand on guard against it. And wherever it gains entrance, there it weakens faith, arrests prayer, extinguishes the Spirit, and transforms the congregation into an outward society of men with an outward use of the means of grace, without Spirit and life.

Therefore stand watch, and keep yourself always near to Jesus himself in simple faith, in constant and inward fellowship in prayer, so that you are first and foremost one with him; then through him you come into connection—yes, into union—with the believers, not through the believers into connection with Jesus.

But it is not to be wondered at that there are so few who are willing to walk this simple, biblical, and spiritual way. It costs so much less for our flesh when we may walk the way of Catholicism. Human nature shrinks from the way of faith, because it is the way of the Cross.

The direct encounter with the Savior is hard for flesh and blood, when we are made to stand at the foot of the Cross as poor sinners, who with tears of repentance must confess our sin and receive forgiveness of unmerited grace. And then thereafter to live and walk in faith, so that we hold fast to the Invisible as though we saw Him—this often falls so hard upon us. To surrender oneself and one’s own honor and rest solely upon Jesus and His merit; to renounce one’s own will and follow Jesus in the bloody footsteps; to lose the pleasant feeling that we are benefactors of the congregation by joining ourselves to it and supporting it, and in return must feel that we are poor beggars at the door of grace—yes, this is indeed heavy for the great, proud, vain old Adam. Yes, it truly is so; yet the death of the flesh is the life of the Spirit, and to be cut off from the old, sin‑defiled race and grafted into the pure, holy Jesus, and to have one’s life from Him and in Him, this is in truth a greater honor than all the glory of the world. To cast off the old, unclean, defiled garment and put on the Lord Jesus—what would not an earnest soul give for such an exchange! Mortal indeed it is to be rid of the false sheen and the useless self‑deception, and to come to the genuine, unadulterated being that accompanies living faith in Jesus—what is it not worth!

Let us therefore walk this direct way! It is not only the foremost; it is the only one; no other leads to the goal, which is the blessed life of the soul in time and in eternity. If we go in faith to Jesus and lay hold of Him as He is, then we have in Him the Reconciler, the Teacher, and the King. We are reconciled with the Father and filled with His love; we learn to walk the narrow way of life and to conduct ourselves in love; we receive a Head and a Lord who leads us with the tenderest love and grants us the most honorable calling that can be bestowed upon a human being—to wage the battle of light against darkness and to labor for the restoration of a fallen race to the glory of God. And in this way, and in this way only, do we attain a unity among believers that is not dependent upon outward things, nor broken by faults and frailties, offenses and scandals among the faithful. So long as they yet cling to Jesus, the unity is there; and only the one who falls away from the Lord comes out of that chain which binds the friends of Jesus together.

Therefore the matter concerns faith and the inward vitality of faith; for where the connection with the Head, Jesus, abides, there the mutual union and coherence of the members also abides. And in order that this connection with Jesus may be secure and independent of human strength and will, there has been given to the children of God, as the third strong bond of unity, the one Spirit, God’s own Holy Spirit, the Spirit of faith and of love, that their unity might be the work of the triune God.

\textbf{The same Spirit}

The Father, the Son, and the Holy Spirit unite all believers, so that they constitute the one true fellowship upon earth, a fellowship that is not dissolved throughout all eternity.

There could be no fellowship with the Father and the Son except through the Spirit. There is no faith, no love, no hope, except through the Spirit. All that Christians gather together under the word “spiritual life” must and can be wrought only by God’s own Spirit.

Sin took the Spirit away from mankind. “In his corruption man is flesh,” says the Scripture in Genesis 6:3. Therefore the Spirit of God can be given only on the ground of the atonement of our sins through the death of Jesus Christ. The atonement is the condition for the outpouring of the Spirit; and it is the Jesus who was lifted up upon the cross and at the cross, who sends forth the Spirit in order to draw all unto himself. And all who do not resist the Spirit’s calling voice, the blessed invitation through the Gospel, come to the cross of Christ and believe in the crucified One; and the Spirit who called them and drew them takes up his dwelling in the broken and opened hearts, and fills them with the love and mind of the crucified Savior. There the “middle wall” is broken down, there the “enmity” is slain, there all believers are gathered and united into one body through the one Spirit.

The Spirit manifests himself in individual believers as the principle of life and the impulse to labor. He works as a unifying power that binds Head and members together; he is felt as the gift of willingness, which places all Christ’s disciples in his willing host, ready to serve him, because he “has loosed their bonds” (Psalm 116:16).

The Holy Spirit is the principle of election, which separates the chosen race from the world. Therefore resistance to the Spirit is so dangerous, and blasphemy against the Spirit unforgivable; for where the heart is closed to the Spirit, no salvation is possible.

Where, on the other hand, the broken heart opens itself to the Spirit of God and the love of Christ, there a dwelling of God comes into being, a member of Christ, which is led and governed by His will and therefore accomplishes what He wills to have done.

And all these individual souls together become a spiritual house and a holy body of Jesus Christ.

Therefore it is said that all the children of God are one — that is, one person in Jesus Christ; for just as the head rules the whole body and uses it to express and carry out its will, so Jesus, by the Spirit, is mighty and active in all His believers, to lead them and to work through them.

Naturally, the unity of believers is not a matter that lies under their dominion and control, so that they may deal with it as they please. It belongs to their life and being, and is a power of God that prevails in them so long as their faith is living. Another matter it is that they can quench the Spirit, fall away from the faith, and thus also come out of the unity.

Yet the Spirit of God in the redeemed soul is nonetheless freedom, not bondage. Therefore Scripture calls the fellowship of believers not only the body of Christ, but also the bride of Jesus. And this latter image emphasizes no less than the former the unity of believers with Jesus and with one another; but at the same time it also highlights the freedom which the Spirit of love gives. There is always a selfhood in believers which makes their life in Christ a life of freedom, a life that is lived only by those who continually surrender themselves to Him in the obedience of the Spirit. They can break the bond — and lose the life.

The unity of believers, since it is such an effect of the triune God, will show itself in a powerful and recognizable manner. The chosen race has its marks and its tasks, and it is undoubtedly justified to include them here, even if they can be mentioned only in the greatest brevity.

It comes from faith; but faith is a matter of the heart; from this it follows that it is said in Acts 4:32: “And the multitude of them that believed were of one heart and of one soul.” Precisely because they all believed in the same Savior, therefore they were so heartily bound together among themselves. For one faith brings with it one understanding, one will, and one feeling in them all. And if there arises in any one of them a disturbance or confusion in their spiritual life, it is felt throughout the whole fellowship, and there immediately begins an activity on the part of the others to remedy such a need.

The smallest wound is perceptible in a body so finely sensitive, and at once there arises an effort to have the wound healed. It therefore cannot be otherwise than that errors and missteps which threaten the truth or the purity are sought corrected as soon as may be. To have Cain’s evil mind, which would not be his brother’s keeper, is impossible to unite with the unity of the believers or with the Spirit of God. In a good and true sense every believer is his brother’s keeper; for he who preserves his brother from error and fall preserves him in the life of fellowship with Jesus, and he who turns a brother from the error of his way saves a soul from death and hides a multitude of sins.

Therefore the unity of the believers is marked by their inward and heartfelt care for one another in all things. Great or small makes no difference; the living interest they have in one another is of divine origin, and the measure of their concern is not reckoned by a human standard.

And no different is it with the tasks of the kingdom of God, which have been entrusted to His congregation to fulfill. The kingdom of the Son of Man has within itself a divine necessity and impulse to embrace all peoples and to make them all partakers of the blessing which was promised to Abraham and to his seed. The unity of the believers shows itself herein, that they all share this nature, that they seek the salvation of the lost, whether the unredeemed are near at hand or far away. At home and abroad the unity of the believers is visible and recognizable by this zeal for the salvation of souls, this struggle for the kingdom of light and of life upon the earth. With one heart and with one tongue the fellowship of believers praises God—Father, Son, and Holy Spirit—for salvation, and confesses faith in the Triune God. With one heart and with one strength they labor for the same great cause of the kingdom: the proclamation of the Gospel to all nations.

Yet by this it is not said that they are not also many and do not rejoice heartily in becoming more. It has rightly been said that God is the God of diversity and takes delight in diversity. His creation is manifold, and yet the same mighty life pulses through the whole of it. His believers are manifold and endowed with manifold gifts; yet they are all permeated by the same Spirit and the same spiritual life. Let us therefore beware of becoming narrow in our hearts and in our judgments, so that we condemn as dead those whom the Lord nevertheless acknowledges, because they do not appear exactly as we think they ought. We remember the Apostle’s exhortation in 1 Corinthians 12:15–17, and this exhortation is not superfluous in our own day. Enlarged hearts are needed among God’s children, that they may be so much the more like Him with the great Father-heart, who has spared nothing and spares nothing in order to draw all into the embrace of love.

Therefore, brothers and sisters in the Lord, let us give the Spirit room in our hearts—ever greater room—so that the Father and the Son may come in to us and dwell and abide with us, so that, united with the Triune God and with one another, we may become a harmonious power in the world of sin and death, to practice the work of faith and love, and thus prepare the way for the kingdom that is coming, that all flesh shall see the salvation of God.