
\section{First Section}

Under this Section are grouped some of the articles which are particularly suited to clarify the principial side of the Christian congregation, or, in other words, to supply the answer to the question standing over the first piece: “What is at stake?” This piece is therefore also to be regarded in a special sense as an introduction to this volume, inasmuch as it both presents the question from a principial point of view and at the same time intimates its practical significance for the solution of our ecclesiastical task as an Evangelical Lutheran Free Church in this land. It also appeared at a time when the “Conference” was passing through severe inner conflicts, while at the same time it had to endure violent attacks from without. But it was precisely in this period that the free-church direction within this church body was purified and matured to a clear consciousness of its calling and its responsibility toward our people and its ecclesiastical task. — The second piece, “The Holiness of the Congregation,” is a small series of articles from the period immediately prior to “the Union,” and for this very reason acquires increased interest as a testimony that the author never lost sight of the main matter, not even in the rushing current of the time of union. — The third and fourth pieces, “The Liberation of the Congregation” and “The Vivification of the Congregation,” appeared at the time when the failed attempt at union, by the unmerited grace of God, set as its fruit “The Lutheran Free Church.” The statements by Professor Sverdrup included in this Section were thus without exception published at what we may with full right designate as important turning points in the history of our Church. At every decisive stage of development he stepped forward with his admonishing and warning word: Forget not that it is the Congregation that is at stake. — Ed.

\pagebreak

\subsection{What Is at Stake?}

Source: Quarterly Journal for the Norwegian Lutheran Church in America. Edited by Prof. Sverdrup and Oftedal. Third Year (Vol. III), 1877. Pages 110–112. — Ed.

This section appears on pages 1–4 of the original volume. — Present Ed.

\bigskip

The conflict that is being waged between the Synod and the Conference is beginning to grow more animated once again. No one can fail to see this—some with joy, some with heartfelt sorrow; some rejoice over the conflict, because they believe it will bring about the destruction of God’s Church; some mourn over it for the very same reason. Some take pleasure in the conflict, because in it they discern the tokens of a better day for the young Free Church; some lament it, because it must end in the downfall of clerical dominion. We do not concern ourselves with these moods and feelings; we know that they must accompany ecclesiastical development everywhere, and have accompanied it from the Church’s earliest days. We do not believe that the conflict can cease so long as the Church is a church in conflict; but we ask this: what is it that is being contended over, what is it that is being fought for? We ask ourselves whether we are contending rightly; for the manner in which our opponents conduct their struggle, they themselves bear the responsibility, and we are not set to judge them. What is at stake? If it is an unnecessary strife, then we will put an end to it; if it is a necessary strife, then it is also a sinful strife, and we will repent of it. Our settled conviction is this: that in this conflict what is at stake— is the congregation.

There has not yet, for any extended period of time, existed anywhere a Lutheran Free Church. From the earliest days of the Church there have been only very few and very small Christian free churches, persecuted and despised. We therefore stand on a ground both difficult and perilous. Young and inexperienced, so weak and fragile, we are to seek the true form of the Lutheran Free Church. Is it any wonder that there are difficulties and dissension? A great work can never be accomplished without many and severe struggles. But the greatest conflict must arise between priestly power and congregational freedom. The lordly conception, inherited through centuries, which church government and bishops and priests have exercised within the Church, cannot yield to a true and inward cooperation between congregation and office without costing a struggle. Equilibrium cannot be restored again without perilous rending. In these rendings we live, and we do not deceive ourselves when we expect that they will continue yet for some time.

It is our conviction that all that we gather together under the term Wisconsinism\footnote{On the meaning of this expression, see the present work, Volume I, page 108. — Ed.} is an attempt, within the Free Church, to preserve the priestly power of the Norwegian State Church by means of a straitjacket of doctrinal propositions, all of which aim at one goal: to make the congregation into a will-less mass in the hand of the priest. Against the free, living congregation, the doctrine of world-justification is the most violent assault; against the congregation’s self-edification, the Wisconsin doctrine of lay activity is directed; against the congregation’s unhindered and unmediated access to the mercy seat in the blood of Jesus, the doctrine of absolution is aimed; against the congregation’s right to test the spirits by God’s Word, the assertion that the pure doctrine is God’s Word stands in direct conflict. It is because these doctrines and propositions stand in the way of the free congregation’s development among us that we exert so much force in combating them.

It is the congregation that is at stake: what it is, and what right and authority it possesses. And because this is our conviction, therefore we cannot relent or enter into compromise. We gladly desire to confer concerning these matters with our opponents; we gladly wish to test whether our conviction is well grounded. We also desire opportunity to contend orally for the same cause for which we write. But this we believe is the advantage which the “open declaration”\footnote{“Open Declaration” was published under the date January 20, 1874, signed by Professors Even Oftedal and A. Weenaas. Its purpose was, in brief and pointed terms, to set forth “what we contend against, and how we will contend,” and its publication at the time occasioned a violent opposition. A couple of years later, Professor Weenaas withdrew the “Open Declaration,” insofar as he had “co-signed” it, in a postscript to the second edition of his book Wisconsinism, Illustrated by Historical Hostile Acts. Quite shortly thereafter he published in The Standard a “concise account of my withdrawal of the ‘Open Declaration,’” and again in the same paper a couple of weeks later a rather more detailed “Explanation of my withdrawal of the ‘Open Declaration.’” In this latter article he declares that he “now, as before, assents to what is essential in the ‘Open Declaration,’” and that the withdrawal is only “a withdrawal of my signature from the ‘Open Declaration’ in the form in which it was published.” — Ed.} has brought us in the struggle: that it has become ever more clear that that about which the struggle turns, that which is at stake, is nothing else—nothing less—than the congregation, the free, living, believing, and witnessing congregation.

And concerning how we may labor best for the congregation, how we may promote its good, how the blood and death of Jesus may rightly be preached for the salvation of men—concerning these things it is that we desire to confer with all those among our countrymen in America who stand upon the ground of the Lutheran Confession.