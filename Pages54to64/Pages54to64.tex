%FIXME: Recast without the KJV resister.

\begin{center}
\includegraphics[width=0.9\textwidth]{OpenImage.png}
\end{center}

\section{The Free Church Body}


Source: “Lutheraneren og Missionsbladet” 1877 Nos. 20, 22; 1878 Nos. 2, 3, 7, 10, 23, 25. Separate reprints in “Gammelt og Nyt,” first booklet, 1897. — Ed.

This section appears on pages 54–100 of the original volume. — Present Ed.

\bigskip


\subsection{The Congregation}

It is already a long time since the Norwegian church bodies began their activity in America, and there have already been many experiences during that time, both bitter and joyful; and the people have learned much in this school, which is not easily forgotten. For that which has cost weeping and tears, that which has many a time torn apart the tenderest bonds—at times between pastor and congregation, at times between brethren within the same congregation, at times between husband and wife, parents and children—that is written into a man’s heart in such a manner and with such bitter pain that it is not erased so long as there is yet life and breath.

But if these are exceedingly costly things, which no earnest man sets aside or passes over lightly, then there is a seal yet more costly which has been set upon God’s congregation: it is the blood that has been smeared upon its doorposts, the precious blood of Jesus Christ, by which it has been purchased for God to be His free, pure, undefiled Bride. And God has therefore not left His congregation without guidance in His holy Word; but the people whom He has obtained for Himself, these He also instructs concerning the way in which they are to walk.

Therefore, when we are to consider these matters, we must first trace back to God’s own Word, and thereafter we shall judge, insofar as it is possible for us, also the experiences that have already been made in America and the results to which the development has come. We shall begin with that which is first—and which is also last—in the church body: the congregation.


God Himself is the Founder of the Church; He is her Father, He is her Staff and Support; the Son is her Redeemer, her King, her Head, her Bridegroom; the Spirit is her Life, her Freedom, her Advocate, her Pledge. The Church herself is a gathering of human beings whom God has justified through faith in His Son, born again unto the life of the Spirit and granted the right of sonship and the life of children, an inheritance in heaven with the Son and the eternal life of love in the Spirit.

But when we thus speak of the Church as the fellowship in the Holy Spirit, the communion of saints, and of the glory of this fellowship, then we are speaking of invisible and eternal things, which are the object of faith, not of sight. Yet nothing can be the object of faith which has not been revealed. No one can believe in God if God were only invisible. But because God has been revealed in the flesh, because the only-begotten Son, who is in the bosom of the Father, has made Him known, therefore faith is possible. The invisible, upon which faith relies, is also visible through its revelation. The Apostle Paul even says expressly that God’s invisible being, His eternal power and Godhead, are perceived from the creation of the world. So it is also with the communion of saints.

The Church: she is invisible, because her life is hidden with Christ in God; she is visible, because her life, God’s own Holy Spirit, is given to her through visible means, the Word and the Sacraments; visible, because she consists of truly visible human beings; visible, because she possesses recognizable marks and indeed is a shining light in the darkness of the world, a city set upon a hill that cannot be hidden.

The Church is therefore both invisible and visible, just as God Himself is invisible and visible (see John 14:9: He that hath seen Me hath seen the Father). Just as no one could believe in God if He were not revealed, so no one can believe in the communion of saints if it is not revealed. God has revealed Himself in His Son, who Himself is the Word of God; the Church reveals herself in believing human beings, who live by the Word and the Sacraments a spiritual life, rich in the fruits of mercy. And just as God’s revelation in the Son is in humiliation, in lowliness, in poverty, within the bounds of time and space, so also the Church’s revelation is in frailty, in inconspicuousness, in contempt, bound to definite times and definite places. Yet we must hold fast that just as it is the one and the same true God who is invisible and revealed, so it is also the one and the same Church of God that is invisible and visible; for thus says the Word of God: Ye are the light of the world; a city that is set on a hill cannot be hidden.

But we add at once that, as there is a likeness between the revelation of God and the revelation of the Church, so there is also a difference. For God has revealed Himself in the Son in a perfect manner, both through the perfect Word and the perfect act of love; the Church, on the other hand, can reveal itself in a perfect manner only through the Word and the Sacraments, and even there it will often fail to do what it nevertheless, by the grace of God, is able to do.

By contrast, its act of love is never perfect. It has sin in its flesh, and even in its best activity this will cleave to the Church’s work; therefore the Church, like every individual Christian, must daily pray: “forgive us our debt!” Thus the visible Church will often present a spectacle which is not only frail and lowly, but also defiled by sin. Its glory is not merely hidden; it is, alas, often darkened by hideous stains. Among these stains are all the “dead members” of the Church; yet even the living members must many a time complain of themselves that the power of darkness is great in their flesh.

Holy Scripture speaks of the Church in two ways. It speaks of one Church and of many churches. It is of the one, indivisible Church that Jesus Christ speaks when He says to Peter: Thou art Peter, and upon this rock I will build my Church (Matt. 16:18). It is of this Church that Paul writes to the Ephesians: And gave Him to be Head over all things to the Church, which is His body, the fullness of Him that filleth all in all (Eph. 1:23). And again: Christ loved the Church, and gave Himself for it; that He might sanctify it, having cleansed it by the washing of water with the Word; that He might present it to Himself a glorious Church, not having spot or wrinkle (Eph. 5:25–27; compare Col. 1:18; Acts 20:28).

But the Word of God also speaks of many congregations: a congregation in Jerusalem, a congregation in Rome, congregations in Macedonia, congregations in Asia Minor. And each one, taken by itself, is called the Church of God (1 Cor. 1:2). Are these two different things, two kinds of congregations? God forbid; for then they could not be called by the same name. It is the one congregation which shows itself in different places, as Acts 20:28 shows most clearly. It is one and the same Body of Christ which reveals itself in different places. God sent the Son once, among one people, in one land; but the Son has sent His messengers to preach everywhere and in all places His own Word of repentance and forgiveness of sins, in order to gather in every place those who are grafted into the true Vine, those who become members of His Body. All that is said of the one holy Church is also said of each single congregation in every single place. The whole congregation, which we are accustomed to call “the Church”—a word which Holy Scripture never uses—is invisible and visible; the individual congregation is precisely the same: its holiness is hidden and, alas, often darkened by sin; yet its light also shines through the Word, through the Sacraments, through believing people’s self‑sacrificing love toward poor, lost souls. The Church is therefore not a higher kind of society than the congregation, but it is the gathering of all true congregations; it is in every place where there is a congregation, and it is in no other place. That Scripture uses one and the same word everywhere is proof enough that it makes no distinction between Church and congregation. — Our Head, Christ, was revealed once to bear the sin of all; His Body, living and active by the Spirit, is revealed everywhere the Word is preached, and at all times. The Kingdom spreads until it fills the whole earth.

We will with all our might hold fast to this: that it is one and the same congregation which is invisible and visible; and we believe that this is of the utmost practical importance. We know well that the mockers say: a thing cannot be both visible and invisible; it must be two different things. A fool also says in his heart: There is no God. A fool also says: God is invisible; He cannot be manifested in the flesh. But the Word of God, which says, “Our life is hid with Christ in God,” the same Word also says: “Let your light so shine before men, that they may see your good works.” Why do we emphasize this? Because this talk of a visible and an invisible congregation has, alas, become the cause of no small lukewarmness and indifference among us.

There is, for example, a man who belongs to no congregation; according to his own imagination he belongs to the fellowship of believers, but in a visible congregation he will not be, for it is so defiled, so unclean, so mixed. Thus he remains outside; it is not the visible congregation that matters—he does not go to church, he does not partake of the Sacrament; it is all unclean through its muddled state. Thus a soul is lost, and thus God’s congregation loses a laborer, because he would have the invisible congregation and its glory, but not the dust and shame of the visible.

There is a man who is a pastor in a congregation. He labors with the notion that the congregation which has called him as pastor is merely the visible congregation; the invisible congregation is another matter. His visible congregation is only an outward circle, outside of which perhaps one or another may be received into the invisible congregation by faith. Thus he becomes indifferent as to how the congregation appears; thus the congregation becomes indifferent toward itself; thus the salt through indifference and mockery loses its power—wherewith then shall it be salted?

No, this spirit of mockery and lukewarmness, which we all carry with us from the state church, must be cast out. We must give heed to this: that each individual congregation is God’s congregation; each individual congregation which is bought with Jesus’ blood; each individual congregation is one of the seven golden lampstands which John saw. Each of them shall be the manifestation of Christ’s body in the world. If this were to become truly alive to us, what earnestness and holy stirring would not come upon us all who are in the congregation, and what a heavy responsibility would fall upon us for the wretched condition which our sins have given to God’s congregation.

We know that we cannot search hearts and reins, and assuredly many will come in who do not have on the wedding garment; but should it then become so easy a matter for anyone to run about the shops in the city and change congregation by gathering the shop assistants’ names in a book, because he bought clothes in one place and flour in another and tobacco in a third? Many will assuredly come at the invitation, “Come, all!” and many will afterward be a shame to the congregation; but then perhaps one and another would pause and reflect, if it were so that to enter the congregation were the same as to say: “From now on I will be a witness of Jesus Christ; He has indeed bought me with His blood.”

No one can prevent hypocrisy; but a living preaching of the congregation’s true nature can, by God’s grace, nevertheless prevent much levity and indifference, both among pastors and among members of the congregation.

It was altogether impossible to point to any “more convenient time” for the formation of a truly Free Church than that in which the Norwegians came to America. It is superfluous to remind the reader that the Norwegian Church, from the beginning of this century down to our own day, has experienced a season of awakening such as, so far as history knows, it has never had before. And however dark and dead it may unfortunately still be in many places among our people, there is yet scarcely a single family in the whole Norwegian nation that has not in one way or another come into contact with “the Awakening” or with “the awakened.” Some have been seized by the Spirit of God, and through deep distress over sin have learned to know the Saviour as their Saviour; others have gone along with the stream, and without themselves coming to any personal experience of the life in God, they have nevertheless acknowledged the Christianity of the awakened as a living fruit of the work of the Spirit, and many a time have they silently sighed within themselves: “Oh that I were as one of them! Let my soul die the death of the righteous, and let my last end be like his.” Though they have lacked full sincerity to experience both the depths of sin and the depths of grace, yet the Awakening has not passed them by altogether untouched. Some have also followed the awakened in hypocrisy, and these have unfortunately many times become such blemish of shame as have offended the children of God and have caused the name of God and the work of His Spirit to be blasphemed by the children of unbelief. Some have conceived a thorough, almost devilish hatred of this work of God, and they have not only mocked when the shame of the hypocrites was laid bare, but they have also called the movement toward God and the joy in the Lord fanaticism and madness. But almost no one has been able to gain complete footing in peace, wholly untouched by this mighty movement. Nearly all have in one way or another been either for it or against it.

But it is then an evident fact that such a time of decision as this was a peculiar hour of the Lord, which He Himself had chosen, that a free congregation might be formed which was permeated by a living consciousness of what faith in Christ and the Christian life are. It was an hour of the Lord in which a choice and a decision took place in so many hearts; and in such a time the setting apart could also occur which was necessary in order that the free congregation might be formed of those who would voluntarily confess Christ. Mass Christianity and the mass congregation were judged by the Awakening; and when the free congregation was formed, those who in their hearts hated and persecuted all living Christianity would naturally withdraw themselves. And thereby the self-governing congregation became able truly to be governed by the Spirit of Christ, even though it might number many dead members; for even these had yet, in a certain sense, bowed themselves under the Word of God, which many of them gladly wished might bow their hearts to a full knowledge of the truth, that they too might become children of God.

Neither can we be blind to this, that it was a special grace of the Lord that He, precisely at the time when, according to His counsel, the free Norwegian congregations were to be formed, led us out to this new land, where we had to begin entirely anew. We came to a land where the freedom of the congregation was respected by the state, where it stood open to every man, unhindered by the laws of the country and unimpaired in his civil rights, to be within or without the congregation. We came to a land to which we brought nothing with us save the Word of God and our sound confession; where there was no old church property to contend over, no church government from which to seek permission, no clerical estate that could at once begin to preserve the old privileges of rank, so far as it might be done within the free church.

The Norwegians stood in great poverty in the foreign land; but before them lay the prairie, and there was required only the Lord’s blessing and strenuous labor, that there might arise a flourishing garden both in temporal and in spiritual things. We thank the Lord that both have accompanied us unto this day; and if it may now at least to some degree be said that there exists a free church body among the Norwegians, then this is owing to the Lord’s wondrous governance, that at the same time as the fire of revival passed through Norway, He led so many Norwegian men and women, who were seized by His Spirit, across the sea to a land of freedom, where the congregation was permitted to grow in peace with that growth which the Lord granted by His grace.

It has been for the good of the Norwegian-American congregation that the Word of the Lord in the old land had awakened souls both to living love and to conscious resistance against Christ; for for a free church to begin in the lukewarmness and sleep in which the state church finds its greatest peace and comfort—that is the most dreadful thing of all.

It thus came almost as an uncomprehended gift to the congregations that were founded in America, that at the very establishment of the congregation there had to prevail so full a voluntariness as the conditions of human conditions nearly demand. It cannot indeed be denied that, despite the great separation wrought by the “Awakening” between “living Christians” and “dead children of the world,” and despite all the altered circumstances, yet in America also “custom and usage” worked very powerfully toward leading people to enter the congregation. It is likewise true that no Christian can think without sorrow of the many, many Norwegians who in the foreign land have altogether forsaken their Church, and that therefore those who earnestly desired to promote the welfare of the congregation were glad to have as many of their countrymen as possible within it. It must also be admitted that, both on the part of many pastors and of congregational members, there was an anxious fear concerning the pastor’s salary and the expenses, which led them to employ various unsatisfactory means in order to induce people to enter the congregation. And finally, the whole lonely condition in the foreign land naturally came in as a powerful incentive, urging men in every way to unite themselves together.

Yet it must be said that, as a rule, a full voluntariness prevailed in the founding of congregations. It is the Lord’s command that we shall “compel them to come in,” and it is therefore not our task to investigate whether those who enter have on the wedding garment or not. The Lord Himself will examine that in His own time. But the “compulsion” of which the Lord speaks is not any sort of outward constraint; rather it is as when a kindly man urges a foreign traveler to lodge in his house by showing such friendliness that it becomes impossible to refuse his invitation. We must say that just as important as it is that there be full freedom in the founding of a congregation, so necessary it is that there be a truly earnest “compelling to come in” through the Word of God. It must be the Word that gathers a congregation.

Even among the Norwegian emigrants, who indeed all were once received into God’s Church through baptism, it would have been far better if one had never organized any congregation or gathered congregational members in any other way or by any other means than through the preaching of the Word of God, publicly and privately. For even baptized persons are many times, alas, in such a condition that it is better both for themselves and for the congregation that they “stand without” and are the object of the congregation’s influence in that way, than that they be within and perhaps there sleep the deepest sleep, precisely because they are “in the congregation.” Many times, too, overly hasty admissions into the congregation have soon brought about manifest scandal within it, and then one has had to resort to church discipline and exclusion; and as necessary as this means is, we nevertheless know, alas, that its application is bound up with such great dangers both for the congregation that disciplines and for the one who is disciplined, that in many cases and in many congregations it would be better if one could avoid it.

We therefore lay the greatest weight upon this: that the congregation must be gathered by the Word of God, and that the Word’s urging, drawing, and entreating is the only thing that shall be used to bring people into the congregation. If God’s Spirit is permitted to work this purpose in a man, that he desires admission into the congregation, then he is a true increase of the congregation, even if he has not yet come to the peace of faith with God; without this, he is only a diminution of the congregation’s true strength. We will here only add that there is far from everywhere that faithfulness in “compelling them to come in” which there ought to be. There is a dreadful lack of this, that all pastors and all congregational members in this respect can say with Paul, “The love of Christ constraineth us,” to bear forth the testimony of Christ’s death before all without distinction. May the Lord send us, in all our church bodies and in all our congregations, many true crying voices that compel them to come in, so that the Lord’s house may be filled.

Nevertheless, it cannot be concealed that there are a considerable number of congregations in America which have been established upon a foundation altogether different from the preaching of God’s Word. There are even, within the small circle where we are known, congregations whose very existence we fear is owing to party-spirit and hatred, and to purely temporal motives. And there are many members entered upon the rosters who would never have been there, had not purely outward, purely temporal advantages been employed as a lure to draw them in. There has not everywhere been the proper regard for the freedom of the people, nor yet the proper reverence for the purity of the congregation. It is therefore natural that such congregations will reap as they have sown. They have at times had a rapid growth, because passion and fanaticism are indeed powerful forces for swiftly driving a congregation up to a certain height; but they carry the germ of death within themselves, and unless the Spirit of the Lord breathes into them a warmth other than that of passion, the artificial fire of fanaticism will soon devour them.

The first thing, therefore, which is of importance for a congregation, is that it be founded in a right manner. And we must rejoice over the infinite grace of the Lord, who has so wondrously prepared the way for the establishment of the free congregation, so that it may truly be said that there are many congregations in all communions in America which have been spared many, many adversities, because there were few irregularities at the founding. It was indeed to be expected that the transition from a state church to a free church, even in the most favorable cases, must bring with it great difficulties; and we ought not to marvel if some must suffer much on account of their lack of understanding; yet there is in truth great cause to thank the Lord, because He has spared us as much as He has done.

That which has constituted the difficulty has naturally been this, that some have wished to receive too many, others too few. The Wisconsinites, who began by being Grundtvigians and therefore ascribed to Baptism a false significance, and who ended by preaching the justification of the world\footnote{Universal Justification — Present Ed.} and therefore ascribed to Absolution a false significance, have by their very doctrinal position been driven, quite naturally, to an extreme laxity in the founding of congregations. For it is precisely the same thing whether one places a Catholic significance in Baptism or in Absolution: one is thereby also necessarily driven, in practice, to follow a Catholic conception of the congregation. One may therefore be quite certain that where the justification of the world is preached, there the consciousness of the congregation lacks life and power to a high degree. We hope, however, that this is not often the case.

On the other hand, the Elling Society\footnote{The synod founded by the well-known layman Elling Eielsen, “The Evangelical Lutheran Church.” See also Collected Works I, 223. — Ed.} did indeed, in its time, go to the opposite extreme. The “old constitution” was assuredly not in full accord with the order of the Kingdom of God in this respect. There were indeed not especially great difficulties connected with being received into a congregation there, more than elsewhere; but the fault was this, that those who were received into the congregation were, by the very act of reception, stamped as “converted, or on the way of conversion,” and this naturally had a harmful influence upon the rightful standing of the congregation. For we all know how quickly the natural man is inclined to take comfort from all such outward things. And there is scarcely any doubt that several congregations, precisely through this delusion that they were Christians more so than other congregations, have lost no small measure of seriousness and sincerity.

With these providential dealings of God and these peculiar dangers before our eyes, we therefore believe that the best founding of a congregation is that which at one and the same time is a sifting out from all those who either live in complete indifference or in outright denial, and likewise a gathering of all those who, by the Word of God and by nothing else, are freely constrained to enter.

Let it indeed take time to have a congregation built up; but let it also be well understood that it is neither priest nor congregation that shall judge whether those who enter have on the wedding garment.

If in a place there has long been no preaching, then let a powerful and earnest awakening sermon be heard, perhaps for a long time, before a congregation is organized. Let the Word have time to exercise its work, both unto brokenness and unto resistance, if so it must be, before a congregation is founded.

And remember, that it is not necessary that there must be a priest to preach in such a place; but if you dwell there, and the Lord has given you to know your Savior, then begin at once, wherever you may be, to gather people for edifying meetings and the reading of God’s Word. This is the first beginning toward the formation of a congregation.

There are congregations which have quickly gone under because this was neglected before the congregation was founded.

But above all, let us be careful that it is not the bitterness of party spirit which gathers the congregation instead of the love of Christ.

